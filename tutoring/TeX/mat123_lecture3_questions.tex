\documentclass{article}
\usepackage{amsmath}
\usepackage{amssymb}
\usepackage[a4paper, top=25mm, bottom=25mm, left=25mm, right=25mm]{geometry}
\usepackage{pgfplots}
\usepackage{mathtools}
\pgfplotsset{compat=1.18}
\usepgfplotslibrary{fillbetween}
\usepackage{comment}

\begin{document}

\large

\begin{center}
{\huge MAT123 03.11.2025}
\end{center}

\hfill

{\Large \textbf{QUESTIONS}}

\noindent \textbf{Q1}: Evaluate $\displaystyle\lim_{x\to0^+}\left(\frac2\pi\arccos x\right)^{1/x}$.

\hfill

\noindent \textbf{Q2}: Evaluate $\displaystyle\lim_{x\to0^+}\left(e^x-1\right)^{1/\ln x}$.

\hfill

\noindent \textbf{Q3}: Find the equation of the line that is tangent to the curve $(x^2+y^2)^3 = (x-y)^3$ at $(1,-1)$.

\hfill

\noindent \textbf{Q4}: Find the maximum possible total surface area of a cylinder inscribed in a hemisphere of radius $1$.

\hfill

\noindent \textbf{Q5}: Find the closest point(s) on the curve $y = x^2$ to the point $(0, 1)$.

\hfill

\noindent \textbf{Q6}: Evaluate $\displaystyle\lim_{x\to0}\frac{\cos^2x-\cos\left(x\sqrt2\right)}{x^4}$.

\hfill

\noindent \textbf{Q7}: Determine the constants so that $\displaystyle\lim_{x\to\infty}x^3\left(a+\frac bx +\arctan x\right)= c$.

\hfill

\noindent \textbf{Q8}: Find the absolute extreme values of $f(x)=\left|x^2-x-12\right|$ on $[-4, 5]$.

\hfill

\noindent \textbf{Q9:} Find the first derivative of $f(x)=\ln(\ln x)+\tan^3\left(\dfrac{x+1}{x-1}\right)+\pi^{\sin^3{x}}$.

\hfill

\noindent \textbf{Q10:} What value of $a$ makes $f(x)=x^2+\dfrac ax$ have

\textbf{a.} a local minimum at $x=2$?

\textbf{b.} a point of inflection at $x=1$?

\hfill

\hfill

{\Large \textbf{ANSWERS}}

\hfill

\noindent \textbf{Q1:} $e^{-\dfrac2\pi}$ \hspace{50pt}\textbf{Q2:} $e$ \hspace{50pt} \textbf{Q3:} $y=x-2$ \hspace{50pt} \textbf{Q4:} $\pi\left(1+\sqrt2\right)$

\hfill

\noindent \textbf{Q5:} $\left(\dfrac{\sqrt2}2,\dfrac12\right),\:\left(-\dfrac{\sqrt2}2,\dfrac12\right)$

\hfill

\noindent \textbf{Q6:} $\dfrac16$

\hfill

\noindent \textbf{Q7:} $a=-\dfrac\pi2,\:b=1,\:c=\dfrac13$

\hfill

\noindent \textbf{Q8:} Absolute minimum is $0$ at $x=-3, -4$; absolute maximum is $\dfrac{49}4$ at $x=\dfrac12$

\hfill

\noindent \textbf{Q9:} $\dfrac1{x\ln x}-6\tan^2\left(\dfrac{x+1}{x-1}\right)\cdot \sec^2\left(\dfrac{x+1}{x-1}\right)\cdot\dfrac1{(x-1)^2}+\pi^{\sin^3 x}\left[3\sin^2x\cdot\cos x\cdot\ln\pi\right]$

\hfill

\noindent \textbf{Q10:} \textbf{a.} 16, \textbf{b.} -1

\end{document}