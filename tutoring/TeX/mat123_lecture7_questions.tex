\documentclass{article}
\usepackage{amsmath}
\usepackage{amssymb}
\usepackage[a4paper, top=25mm, bottom=25mm, left=25mm, right=25mm]{geometry}
\usepackage{pgfplots}
\usepackage{mathtools}
\pgfplotsset{compat=1.18}
\usepgfplotslibrary{fillbetween}
\usepackage{comment}
\usepackage[utf8]{inputenc}
\usepackage[T1]{fontenc}
\usepackage{parskip}
\usepackage{cancel}

\begin{document}

\large

\begin{center}
{\huge MAT123 26.12.2025}
\end{center}

\vspace{1em}

{\setlength{\parindent}{1em} \Large
\indent \textbf{QUESTIONS}
}

\textbf{Q1.} Determine if the following sequences converge or diverge.

{\setlength{\parindent}{1em}
\indent \textbf{a.} $a_n=\dfrac{1}{\sqrt{n^2-1}-\sqrt{n^2+n}}$ \ \
\indent \textbf{b.} $b_n=\dfrac{\sin^2n}{2^n}$ \ \
\indent \textbf{c.} $c_n=\dfrac{n^2}{2n-1}\,\sin\dfrac1n$ \\[1em]
\indent \textbf{d.} $\displaystyle a_n=\int_1^n\frac1{x^p}\,dx, \quad p>1$ \ \
\indent \textbf{e.} $b_n=\dfrac{n!}{n^n}$}

\vspace{1em}

\textbf{Q2.} Determine whether each of the following series converges or diverges.

{\setlength{\parindent}{1em}
\indent \textbf{a.} $\displaystyle\sum_{n=1}^\infty\dfrac{(-2)^{n+1}+3^n}{4^n}$ \
\indent \textbf{b.} $\displaystyle\sum_{n=1}^\infty\dfrac{n(n+1)}{(n+2)(n+3)}$ \
\indent \textbf{c.} $\displaystyle\sum_{n=0}^\infty\dfrac{\cos(n\pi)}{5^n}$ \\[1em]
\indent \textbf{d.} $\displaystyle\sum_{n=1}^\infty\ln\sqrt{\dfrac{n+1}n}$ \
\indent \textbf{e.} $\displaystyle\sum_{n=3}^\infty\dfrac1{n\,\ln n\,\ln(\ln n)}$ \
\indent \textbf{f.} $\displaystyle\sum_{n=1}^\infty\sqrt{\frac{n+1}{n^2+2}}$}

\vspace{1em}

\textbf{Q3.} Find the radius, center, and interval of convergence of $\displaystyle\sum_{n=1}^\infty\left(\frac{n}{n+1}\right)^{n^2}x^n$.

\vspace{1em}

\textbf{Q4.} Find the Maclaurin series for the function $\dfrac{x^2}{1+x}$.

\newpage

{\setlength{\parindent}{1em} \Large
\indent \textbf{ANSWERS}
}

\textbf{Q1.} To determine whether the sequence converges or diverges, we compute its limit as $n\to\infty$.

\vspace{1em}

\textbf{a.} Rationalize the denominator; i.e., multiply and divide by the conjugate of the denominator.

\[
a_n=\frac{1}{\sqrt{n^2-1}-\sqrt{n^2+n}}
\cdot
\frac{\sqrt{n^2-1}+\sqrt{n^2+n}}{\sqrt{n^2-1}+\sqrt{n^2+n}}
=\frac{\sqrt{n^2-1}+\sqrt{n^2+n}}{(n^2-1)-(n^2+n)}
\]

Simplify the denominator.

\[(n^2-1)-(n^2+n)=-n-1\implies a_n=\frac{\sqrt{n^2-1}+\sqrt{n^2+n}}{-n-1}\]

Factor out \(n\) from the square roots.

\[\sqrt{n^2-1}=|n|\sqrt{1-\frac{1}{n^2}}, \quad\sqrt{n^2+n}=|n|\sqrt{1+\frac{1}{n}}\]

$|n|$ simplifies to $n$ because $n>0$. Hence,

\[a_n=\frac{n\left(\sqrt{1-\frac{1}{n^2}}+\sqrt{1+\frac{1}{n}}\right)}{-n-1}\]

Divide numerator and denominator by \(n\):

\[a_n=\frac{\sqrt{1-\frac{1}{n^2}}+\sqrt{1+\frac{1}{n}}}
{-\left(1+\frac{1}{n}\right)}\]

Take the limit.

\[
\lim_{n\to\infty} a_n=\frac{1+1}{-1}=-2\qquad\left[\text{because}\lim_{n\to\infty}\frac1{n^2}=\lim_{n\to\infty}\frac1n=0\right]
\]

Therefore, $\boxed{\lim_{n\to\infty} a_n=-2}$

\vspace{1em}

\textbf{b.} Use bounds on the numerator. For all real numbers \(n\),

\[
-1\leq \sin n\leq 1\implies0 \le \sin^2 n \le 1.
\]

Thus,

\[
0 \le \frac{\sin^2 n}{2^n} \le \frac{1}{2^n}.
\]

Since $\displaystyle\lim_{n\to\infty}0=\lim_{n\to\infty}\dfrac1{2^n}=0$, by the Squeeze Theorem, the sequence $\dfrac{\sin^2n}{2^n}$ also converges to

\[\boxed{0}\]

\vspace{1em}

\textbf{c.} Rewrite the expression to isolate known limits.

\[
c_n=\left(\frac{n^2}{2n-1}\right)\sin\frac{1}{n}
\]

Rewrite the rational factor:

\[
\frac{n^2}{2n-1}=\frac{n}{2-\frac{1}{n}}\implies
c_n=\frac{n}{2-\frac{1}{n}}\,\sin\frac{1}{n}
\]

Use a standard trigonometric limit. Recall that

\[
\lim_{x\to 0} \frac{\sin x}{x}=1
\]

Let \(x=\dfrac{1}{n}\). Then

\[
\sin\frac{1}{n}=\frac{1}{n}\cdot\frac{\sin\frac{1}{n}}{\frac{1}{n}} \implies
c_n=\frac{n}{2-\frac{1}{n}} \cdot \frac{1}{n}
\cdot \frac{\sin\frac{1}{n}}{\frac{1}{n}}
=\frac{1}{2-\frac{1}{n}} \cdot \frac{\sin\frac{1}{n}}{\frac{1}{n}}
\]

Take the limit.

\[
\lim_{n\to\infty}\left(\frac1{2-\frac{1}{n}}\right)=\frac12,
\qquad
\lim_{n\to\infty}\frac{\sin\frac{1}{n}}{\frac{1}{n}}=1\implies \lim_{n\to\infty}\left[\left(2-\frac{1}{n}\right)\,\frac{\sin\frac{1}{n}}{\frac{1}{n}}\right]=\frac12\cdot1=\boxed{\frac12}
\]

\vspace{1em}

\textbf{d.} Evaluate the integral.

\[\lim_{n\to\infty}\int_1^n\frac1{x^p}\,dx=\lim_{n\to\infty}\left.\frac{x^{1-p}}{1-p}\right|_1^n=\lim_{n\to\infty}\left(\underbrace{\frac{n^{1-p}}{1-p}}_0-\frac{1^{1-p}}{1-p}\right)=\lim_{n\to\infty}\frac1{p-1}\]

\[\boxed{\text{The limit of the sequence is }\frac1{p-1}.}\]
\vspace{1em}

\newpage

\textbf{e.} Write the sequence as a product.

\[b_n=\frac{n!}{n^n}=\frac{1\cdot 2\cdot 3\cdots n}{n\cdot n\cdot n\cdots n}\]

Estimate the product. For \(k \le n\), we have \(\frac{k}{n} \le 1\). In particular, for
\(k=1,2,\dots,n\),

\[
\frac{k}{n} \le \frac{n}{n}=1.
\]

Hence,

\[
b_n=\frac{1\cdot2\cdot3\,\ldots \,n}{n\cdot n\cdot n\,\ldots\, n}\leq\frac1n\cdot1\cdot1\,\ldots\,1 = \frac1n.\implies b_n\leq\frac1n
\]

$n!$ and $n^n$ are both positive for $n>0$. Therefore,

\[
0 \le b_n \le \frac{1}{n}.
\]

Since $\displaystyle\lim_{n\to\infty}0=\lim_{n\to\infty}\dfrac1{n}=0$, by the Squeeze Theorem, the sequence $\dfrac{n!}{n^n}$ also converges to

\[\boxed0\]

\vspace{1em}

\textbf{Q2.} \textbf{a.} Split the series.

\[\sum_{n=1}^\infty \frac{(-2)^{n+1}+3^n}{4^n}=\sum_{n=1}^\infty \frac{(-2)^{n+1}}{4^n}+\sum_{n=1}^\infty \frac{3^n}{4^n}.\]

Simplify each term. For the first series,

\[
\sum_{n=1}^\infty\frac{(-2)^{n+1}}{4^n}=\sum_{n=1}^\infty\frac{(-2)^n\cdot(-2)}{4^n}=-2\sum_{n=1}^\infty\left(\frac{-1}{2}\right)^n
\]

For the second series,

\[
\sum_{n=1}^\infty \frac{3^n}{4^n}
=
\sum_{n=1}^\infty \left(\frac{3}{4}\right)^n.
\]

Both series are geometric with common ratios
\[
r_1=-\frac{1}{2}, \qquad r_2=\frac{3}{4}
\]

and satisfy \(|r_1|<1\) and \(|r_2|<1\). Remember the geometric series formula.

\[\sum_{n=1}^\infty ar^n=\frac r{1-r}\]

First series:
\[-2\sum_{n=1}^\infty \left(-\frac{1}{2}\right)^n=-2\left(\frac{-\frac12}{1-\left(-\frac{1}{2}\right)}\right)=\frac{2}{3}.
\]

Second series:
\[\sum_{n=1}^\infty \left(\frac{3}{4}\right)^n=\frac{\frac34}{1-\frac{3}{4}}=3.\]

Add the results.

\[
\sum_{n=1}^\infty \frac{(-2)^{n+1}+3^n}{4^n}
=
\frac{2}{3}+3
=
\frac{11}{3}.
\]

\[
\boxed{\text{The series converges and its sum is } \frac{11}{3}.}
\]

\vspace{1em}

\textbf{b.} Divide the numerator and denominator by \(n^2\):

\[
a_n
=
\frac{1+\frac{1}{n}}{\left(1+\frac{2}{n}\right)\left(1+\frac{3}{n}\right)}.
\]

As \(n \to \infty\), $\displaystyle\frac{2}{n},\frac{3}{n}\to 0$. So

\[
\lim_{n\to\infty} a_n
=
\frac{1}{(1)(1)}
=
1.
\]

Apply the Divergence Test. Since $\displaystyle\lim_{n\to\infty} a_n =1\neq 0$, the Divergence Test implies that the series

\[
\sum_{n=1}^\infty \frac{n(n+1)}{(n+2)(n+3)}
\]

$\boxed{\text{diverges}}$.

\newpage

\textbf{c.} Let $\displaystyle a_n=\frac{\cos(n\pi)}{5^n}$. Apply the Ratio Test:
\[
\lim_{n\to\infty}\left|\frac{a_{n+1}}{a_n}\right|
=
\lim_{n\to\infty}\left|\frac{\cos((n+1)\pi)}{5^{n+1}}\cdot\frac{5^n}{\cos(n\pi)}\right|
=
\lim_{n\to\infty}\frac{|\cos((n+1)\pi)|}{|\cos(n\pi)|}\cdot\frac{1}{5}
\]

Since \(\cos(n\pi)=(-1)^n\), we have \(|\cos(n\pi)|=1\) for all \(n\). Therefore,
\[
\lim_{n\to\infty}\left|\frac{a_{n+1}}{a_n}\right|=\frac{1}{5}<1.
\]

By the Ratio Test, the series converges absolutely. Therefore, the series $\boxed{\text{converges}}$.

\textbf{d.} First rewrite the general term:
\[
\ln\sqrt{\frac{n+1}{n}}
=
\frac{1}{2}\ln\left(\frac{n+1}{n}\right)
=
\frac{1}{2}\bigl(\ln(n+1)-\ln n\bigr).
\]

Examine the partial sums. Let
\[
S_N=\sum_{n=1}^N \ln\sqrt{\frac{n+1}{n}}
=
\frac{1}{2}\sum_{n=1}^N \bigl(\ln(n+1)-\ln n\bigr).
\]

Take the limit.

\begin{align*}\lim_{N\to\infty}S_N&=\frac12\bigg[(\cancel{\ln2}-\ln1)+\cancel{\ln3}-\cancel{\ln2})+(\cancel{\ln4}-\cancel{\ln3})\\&\quad+\ldots+(\cancel{\ln N}-\cancel{\ln(N-1)})+(\ln(N+1)-\cancel{\ln N})\bigg]=\frac12(\ln(N+1)-\ln1)\end{align*}

\[\lim_{N\to\infty}S_N=\frac12\ln(N+1)=\infty\]

Since the partial sums diverge to infinity, the series
\[
\sum_{n=1}^\infty \ln\sqrt{\frac{n+1}{n}}
\]
$\boxed{\text{diverges.}}$

\vspace{1em}

\textbf{e.} We determine whether the series converges or diverges using the Integral Test. Verify the conditions of the Integral Test.

For \(x \ge 3\), the function $f(x)=\dfrac{1}{x\,\ln x\,\ln(\ln x)} $ is positive, continuous, and decreasing. Thus, the Integral Test applies.

Evaluate the corresponding improper integral. Consider
\[
\int_3^\infty \frac{1}{x\,\ln x\,\ln(\ln x)}\,dx.
\]

Use the substitution
\[
u=\ln(\ln x),
\quad
du=\frac{1}{x\ln x}\,dx.
\]

For $x=3,\:u=\ln\ln3$. For $x\to\infty,\:u\to\infty$.

Then the integral becomes
\[
\int_3^\infty \frac{1}{x\,\ln x\,\ln(\ln x)}\,dx=\int_{\ln\ln3}^\infty \frac{1}{u}\,du.
\]

This is an improper, where we need to take the limit.

\[
\int_{\ln\ln3}^\infty \frac{1}{u}\,du=\lim_{R\to\infty}\int_{\ln\ln3}^R \frac{1}{u}\,du=\lim_{R\to\infty}\ln|u|\bigg|_{\ln\ln3}^R=\lim_{R\to\infty}(\ln|R|-\ln\ln\ln3)=\infty
\]

Since the corresponding improper integral diverges, the series
\[
\sum_{n=3}^\infty \frac{1}{n\,\ln n\,\ln(\ln n)}
\]
also $\boxed{\text{diverges}}$ by the Integral Test.

\vspace{1em}

\textbf{f.} We determine whether the series converges or diverges using the Limit
Comparison Test. Consider the series
\[
\sum_{n=1}^\infty \frac{1}{\sqrt{n}},
\]
which is a \(p\)-series with \(p=\tfrac12<1\) and therefore diverges. Let
\[a_n=\sqrt{\frac{n+1}{n^2+2}},\qquad b_n=\frac{1}{\sqrt{n}}.\]

Then
\[\frac{a_n}{b_n}=\sqrt{\frac{n+1}{n^2+2}} \cdot \sqrt{n}=\sqrt{\frac{n(n+1)}{n^2+2}}.\]

Divide numerator and denominator inside the square root by \(n^2\):

\[\frac{a_n}{b_n}=\sqrt{\frac{1+\frac{1}{n}}{1+\frac{2}{n^2}}}.\]

Take the limit.

\[
\lim_{n\to\infty}\frac{1}{n}= 0,
\qquad\lim_{n\to\infty}\frac{1}{n^2}=0\qquad\implies\qquad\lim_{n\to\infty} \frac{a_n}{b_n}=\sqrt{\frac{1}{1}}=1
\]

Since $\displaystyle 0 < \lim_{n\to\infty} \frac{a_n}{b_n} < \infty$ and \(\sum b_n\) diverges, the given series also $\boxed{\text{diverges}}$ by the Limit Comparison Test.

\newpage

\textbf{Q3.} The series is of the form
\[
\sum_{n=1}^\infty a_n x^n,
\qquad
a_n=\left(\frac{n}{n+1}\right)^{n^2}.
\]
Hence, the center of the power series is $0$.

Use the Root Test to find the radius of convergence. Consider

\[
\lim_{n\to\infty} \sqrt[n]{|a_nx^n|}
=
\lim_{n\to\infty} \sqrt[n]{\left(\frac n{n+1}\right)^{n^2}}|x|
=
\lim_{n\to\infty} \left(\frac{n}{n+1}\right)^n |x|.
\]

Calculate $\displaystyle\lim_{n\to\infty}\left(\dfrac{n}{n+1}\right)^n$.

\[\lim_{n\to\infty}\left(\dfrac{n}{n+1}\right)^n=\lim_{n\to\infty}\left(\frac{n+1}n\right)^{-n}=\lim_{n\to\infty}\frac1{\left(1+\frac1n\right)^n}=\frac1{\displaystyle\underbrace{\lim_{n\to\infty}\left(1+\frac1n\right)^n}_{\text{standard limit}}}=\frac1e\]

We obtain
\[
\lim_{n\to\infty} \sqrt[n]{|a_n x^n|}
=
\frac{|x|}{e}.
\]

By the Root Test, the series converges if

\[\frac{|x|}{e} < 1\implies |x| < e.\]

Thus, the radius of convergence is $R=e$. Test the endpoints.

\textit{Case 1}: \(x=e\). The series becomes

\[
\sum_{n=1}^\infty \left(\frac{n}{n+1}\right)^{n^2} e^n
=
\sum_{n=1}^\infty \left[e\left(\frac{n}{n+1}\right)^n\right]^n.
\]

Now, we need to compute $\displaystyle\lim_{n\to\infty}\left(\frac{n}{n+1}\right)^{n^2}e^n$ for the Divergence Test. Relate this series to the corresponding function and calculate

\[L=\lim_{x\to\infty}\left(\frac{x}{x+1}\right)^{x^2}e^x.\]

First rewrite the expression using exponentials:

\[
\left(\frac{x}{x+1}\right)^{x^2} e^x
=
e^{\textstyle\left(x^2 \ln\!\left(\frac{x}{x+1}\right)+x\right)}.
\]

Thus it suffices to compute

\[
\lim_{x\to\infty} \left[x^2 \ln\!\left(\frac{x}{x+1}\right)+x\right].
\]

Rewrite.

\[\lim_{x\to\infty}
\frac{\frac1x+\ln\left(\frac{x}{1+x}\right)}{1/x^2}.
\]

This limit is of the indeterminate form \(\frac{0}{0}\), so we apply
L’Hôpital’s Rule. Differentiate the numerator and denominator.

Numerator:

\[
\frac{d}{dx}\!\left(\frac{1}{x}+\ln\!\left(\frac{x}{x+1}\right)\right)
=
-\frac{1}{x^2}
+
\left(\frac{x+1}{x}\cdot\frac{1\cdot(x+1)-x\cdot(1)}{(x+1)^2}\right).
\]

Denominator:

\[
\frac{d}{dx}\!\left(\frac{1}{x^2}\right)
=
-\frac{2}{x^3}.
\]

Thus the limit becomes

\[
\lim_{x\to\infty}
\frac{-\frac{1}{x^2}+\frac{x+1}{x}\cdot\frac{1}{(x+1)^2}}{-\frac{2}{x^3}}=\lim_{x\to\infty}
\frac{-\frac{1}{x^2}+\frac{1}{x}\cdot\frac{1}{x+1}}{-\frac{2}{x^3}}
\]

Multiply numerator and denominator by \(x^3\):

\[
\lim_{x\to\infty}
\frac{-x + \frac{x^2}{x+1}}{-2}=\lim_{x\to\infty}
\frac{-x + (x-\frac{x}{x+1})}{-2}=\lim_{x\to\infty}
\frac{x}{2(x+1)}=\frac12.
\]

Hence,

\[
\lim_{x\to\infty}
\left[x^2 \ln\!\left(\frac{x}{x+1}\right)+x\right]
=
\frac12.
\]

Finally,

\[
\lim_{x\to\infty}
\left(\frac{x}{x+1}\right)^{x^2} e^x
=
\sqrt e.
\]

Since the limit converges to $\sqrt e$, which is different than $0$, by the Divergence Test, the series at this point diverges.

\textit{Case 2}: \(x=-e\). The series becomes

\[
\sum_{n=1}^\infty(-1)^n \left(\frac{n}{n+1}\right)^{n^2} e^n
=
\sum_{n=1}^\infty(-1)^n \left[e\left(\frac{n}{n+1}\right)^n\right]^n.
\]

From earlier, we found out that $\displaystyle\lim_{x\to\infty}
\left(\frac{x}{x+1}\right)^{x^2} e^x
=
\sqrt e$. The limit we want to evaluate $\displaystyle\lim_{x\to\infty}
(-1)^n\left(\frac{x}{x+1}\right)^{x^2} e^x$ does not exist because the function oscillates between $-\sqrt e$ and $\sqrt e$ as $x\to\infty$. Therefore, the series at this point is also divergent.

The series converges for
\[
|x|<e
\]
and diverges at \(x=\pm e\).

\[
\boxed{
\begin{aligned}
\text{Center:} &\ 0,\\
\text{Radius of convergence:} &\ R=e,\\
\text{Interval of convergence:} &\ (-e,e).
\end{aligned}
}
\]

\vspace{1em}

\textbf{Q4.} We begin with the geometric series
\[
\frac{1}{1-x}
=
\sum_{n=0}^\infty x^n=1+x+x^2+\ldots,
\qquad |x|<1.
\]

Replace \(x\) by \(-x\) to obtain
\[
\frac{1}{1+x}
=
\sum_{n=0}^\infty (-1)^n x^n,
\qquad |x|<1.
\]

Now multiply both sides by \(x^2\):
\[
\frac{x^2}{1+x}
=
x^2 \sum_{n=0}^\infty (-1)^n x^n
=
\sum_{n=0}^\infty (-1)^n x^{n+2}.
\]

Therefore, the Maclaurin series for the function is
\[
\boxed{
\frac{x^2}{1+x}
=
\sum_{n=0}^\infty (-1)^n x^{n+2},
\qquad |x|<1.
}
\]

\end{document}