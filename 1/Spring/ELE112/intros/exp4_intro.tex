%% Comply with the comment where two consecutive percent symbols are written and then delete that comment.
%%
% Student Number: 2240357068
% Student Name: Baturay KAFKAS
% EEE @ Hacettepe University

% Last update: 10:32 27/04/25

% My resource collection: https://github.com/kfksbtry/uni
% Circuits were set up in LTspice.

\documentclass{article}

\usepackage{comment}
\usepackage{graphicx}
\usepackage[top=25mm, bottom=25mm, left=25mm, right=25mm]{geometry}
\usepackage{amsmath}
\usepackage{moresize}
\usepackage{float}
\usepackage{fancyhdr}
\usepackage{booktabs}

\pagestyle{fancy}
\fancyhf{}
\fancyhead[R]{Baturay KAFKAS 2240357068 Electrical \& Electronics Engineering}

\rfoot{\thepage}
\renewcommand{\headrulewidth}{0pt} 
\renewcommand{\footrulewidth}{0pt}

\setcounter{page}{37}

\begin{document}

\large

{\small \textbf{HACETTEPE UNIVERSITY}}

{\small \textbf{DEPARTMENT OF ELECTRICAL \& ELECTRONICS ENGINEERING}} 

{\small \textbf{ELE 112 INTRODUCTION TO ELECTRICAL ENGINEERING LABORATORY}} 

\vspace{4mm}

{\small \textbf{EXPERIMENT \#4}}

\vspace{4mm}

{\small \textbf{EXPERIMENTAL EVALUATION OF SOURCE TRANSFORMATIONS, VOLTAGE AND CURRENT DIVIDER CIRCUITS AND THE SUPERPOSITION PRINCIPLE IN DC CIRCUITS}}

\vspace{4mm}

{\textbf{Objective:}}
{Source transformations, voltage and current divider circuits and superposition.}

\vspace{4mm}

{\textbf{Theory:}}
{The details are given in the ELE110 Introduction to Electrical Engineering course notes.}

\vspace{4mm}

{\Large \textbf{2. EXPERIMENTAL WORK}}

\vspace{4mm}

{\textbf{2.1} Set up the circuit shown in \textit{Fig. 1}. Connect a voltmeter between a and b, and measure the voltage V$_\text{ab}$. Also measure the equivalent resistance between a and b (R$_\text{ab}$).}

\vspace{4mm}

{\textbf{2.2} Set up the circuit in \textit{Fig. 2}, and measure the voltage V$_\text{ab}$ and the current I$_\text{ab}$ for V$_\text{in}$ = 2V, 4V, 6V, 8V and 10V, respectively.}

\vspace{4mm}

{\textbf{2.3} Set up the circuit in \textit{Fig. 3}, and measure the current I$_\text{in}$,   I$_1$, and I$_2$ for V$_\text{in}$ = 0.66V, 1.32V, 1.98V, 3.3V, and 3.64V, respectively.}

\vspace{4mm}

{\textbf{2.4} Set up the circuit in \textit{Fig. 4}.

\ a) Connect current source only, and measure V$_\text{ab}$.

\ b) Connect voltage source only, and measure V$_\text{ab}$.

\ c) Connect both of the sources at the same time, and measure  V$_\text{ab}$.}

\vspace{4mm}

{\Large \textbf{3. RESULTS AND CONCLUSION}}

\vspace{4mm}

{\textbf{3.1} Compare your results from the preliminary work with the measurement you obtained in step 2.1.}

{\textbf{3.2} By using the measurements in step 2.2, sketch V$_\text{ab}$ vs. V$_\text{in}$. Compare this plot to the one in your preliminary work and comment.}

{\textbf{3.3} By using the measurements in step 2.3., Sketch I$_\text{in}$ vs I$_1$ and I$_\text{in}$ vs I$_2$. Compare these plots to the one in your preliminary work and comment.}

{\textbf{3.4} By using the measured values in step 2.4, prove the superposition principle.}

\vspace{4mm}

{\Large \textbf{EQUIPMENTS AND COMPONENTS}}

\vspace{4mm}

{DC power supply}

{10 mA. current source}

{AVO meter}

{Resistors: 100$\Omega$ (\#1), 120$\Omega$ (\#1), 220$\Omega$ (\#2), 330$\Omega$ (\#1)}

\end{document}