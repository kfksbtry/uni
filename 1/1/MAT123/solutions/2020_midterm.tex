\documentclass{article}
\usepackage{graphicx}
\usepackage{amsmath}
\usepackage{amssymb}
\usepackage[a4paper, top=25mm, bottom=25mm, left=25mm, right=25mm]{geometry}
\usepackage{pgfplots}
\pgfplotsset{compat=1.18}
\usepackage{mathtools}
\DeclarePairedDelimiter\ceil{\lceil}{\rceil}
\DeclarePairedDelimiter\floor{\lfloor}{\rfloor}

\begin{document}

\usepgfplotslibrary{fillbetween}
\large

\begin{center}
2020-2021 Fall\\MAT123 Midterm\\(30/11/2020)
\end{center}

\noindent 1) Evaluate $\displaystyle\lim_{x\to0^+} (\sqrt x)^{\ln(x+1)}$.

\hfill

\noindent 2) Show that the function $f(x)$ defined by

\[
 f(x) =
\begin{cases}
\displaystyle x\arctan\frac{1}{x}, & \text{if}\ x > 0 \\
0, & \text{if}\ x = 0 \\
\displaystyle \frac{x-\cos x}{x^2}, & \text{if}\ x < 0 \\
\end{cases}
\]

\noindent is not continuous at the point $x=0$.

\hfill

\noindent 3) Find an equation of the line which is tangent to the curve

\begin{equation*}
\cos y^2 + xy + 1= 0
\end{equation*}

\noindent at the point $\displaystyle \left(\sqrt{\frac{2}{\pi}}, -\sqrt{\frac{\pi}{2}} \right)$. Note that $y=f(x)$.

\hfill

\noindent 4) A block of ice in the shape of a cube originally having volume 3000 cm$^3$. When it is melting, the surface area is decreasing at the rate of 36 cm$^2$/h. At what rate does the length of each of its edges decrease at the time its volume is 216 cm$^3$? Assume that during melting, the block of ice maintains its cubical shape.

\hfill

\noindent 5) a) Using the Intermediate Value Theorem and Rolle’s theorem, show that the equation $e^x + x = 0$ has only one root (Note that if this root $c_1$, then $c_1 \in (-1, 0)$).

\hfill

\noindent b) Determine the interval of increase, decrease, and concavity of the function $f(x) = e^x+x$. By constructing a table, sketch the graph.

\hfill

\noindent 6) Determine (but do not evaluate) the integral corresponding to the area of the region bounded by the curves $y = -x^2 + 1$ and $y = |x|- 1$

\hfill

\noindent 7) Evaluate $\displaystyle \int x\ln x \, dx$.

\hfill

\newpage

\begin{center}
Solutions (Last update: 7/15/25 (15th of July) 8:25 PM)
\end{center}

\noindent 1) Let $L$ be the value of the limit. Since the expression is continuous for $x>0$, we can apply the logarithm function to each side of the equation. Then, we can swap the logarithm and the limit. Use the property of logarithms afterwards.

\begin{equation*}L=\lim_{x\to0^+} \left(\sqrt x\right)^{\ln\left(x+1\right)}\end{equation*}

\begin{equation*}\ln(L)=\ln\left[\lim_{x\to0^+} \left(\sqrt x\right)^{\ln\left(x+1\right)}\right] = \lim_{x\to0^+} \ln\left[\left(\sqrt x\right)^{\ln\left(x+1\right)}\right] = \lim_{x\to0^+} \ln\left[\left(\sqrt x\right)^{\ln\left(x+1\right)}\right] \end{equation*}
\begin{equation*}\ln\left(L\right) = \lim_{x\to0^+} \left[\ln\left(x+1\right)\cdot\ln\left(\sqrt x\right)\right] \quad\left[0\cdot\infty\right]\end{equation*}

\hfill

\noindent Make it so the limit is in the form $\displaystyle \frac00$ or $\displaystyle \frac\infty\infty$ in order to apply the L'Hôpital's rule.

\begin{align*}\ln\left(L\right) &= \lim_{x\to0^+} \left[\frac{\ln\left(\sqrt x\right)}{\frac1{\ln\left(x+1\right)}}\right] \quad\left[\frac\infty\infty\right]\\\\&\overset{\text{L'H.}}{=}\lim_{x\to0^+} \left[\frac{\frac1{\sqrt x}\cdot\frac1{2\sqrt x}}{-\frac1{\ln^2\left(x+1\right)}\cdot \frac1{x+1}}\right] = \lim_{x\to0^+} \left[-\frac{\ln^2(x+1)\cdot(x+1)}{2x}\right]\\\\&=\lim_{x\to0^+} \left[-\frac{\ln^2(x+1)}{2x}\right] \cdot\lim_{x\to0^+} \left(x+1\right)=\lim_{x\to0^+} \left[-\frac{\ln^2(x+1)}{2x}\right]\quad\left[\frac00\right]\\\\&\overset{\text{L'H.}}{=}\lim_{x\to0^+} \left[-\frac{2\ln(x+1) \cdot\frac1{x+1}}2\right]=\lim_{x\to0^+} \left[\frac{\ln(x+1)}{x+1}\right] = \frac{\ln1}{1}=0\end{align*}

\hfill

\noindent $\ln(L)= 0$, so $\boxed{L = 1}$.

\hfill

\noindent 2) $\displaystyle \arctan\frac1x$ takes it values on $\displaystyle -\frac{\pi}2 \leq \arctan\frac1x \leq \frac{\pi}2$. Multiply each side by $x$, then we get $\displaystyle -\frac{x\pi}2 \leq x\arctan\frac1x \leq \frac{x\pi}2$. Take the limits of each side. By the squeeze theorem, we see that the limit of $\displaystyle x\arctan\frac1x$ at the point $x=0$ is $0$. This means that for $f(x)$, the limit from the right side also equals $0$.

\begin{equation*}\lim_{x\to 0}\left(-\frac{x\pi}2\right)\leq\lim_{x\to 0}\left(x\arctan\frac1x\right)\leq\lim_{x\to 0}\left(\frac{x\pi}2\right)\end{equation*}
\begin{equation*}0\leq\lim_{x\to 0}\left(x\arctan\frac1x\right)\leq0\end{equation*}
\begin{equation*}\therefore\lim_{x\to 0}\left(x\arctan\frac1x\right) = 0\end{equation*}

\hfill

\noindent From the left side, the limit is equal to as follows.

\begin{equation*}\lim_{x\to0^-}\frac{x-\cos x}{x^2}=\lim_{x\to0^-}(x-\cos x)\cdot\lim_{x\to0^-}\frac1{x^2} =-\infty\end{equation*}

\hfill

\noindent Continuity requires the equality of one-sided limits and the value of the function at that point. However, the one-sided limits are not equal; $0\neq-\infty$. Therefore, $f(x)$ is discontinuous at $x=0$.

\hfill

\noindent 3) Differentiate both sides implicitly.

\begin{equation*}\frac{d}{dx}\left(\cos y^2 + xy +1 \right) = \frac{d}{dx} \, 0\end{equation*}
\begin{equation*}-\sin y^2 \cdot 2y\frac{dy}{dx} + y + x\frac{dy}{dx}= 0\end{equation*}

\begin{equation*}\frac{dy}{dx}\left(-\sin y^2 \cdot 2y + x\right)= -y\end{equation*}

\begin{equation}\frac{dy}{dx}= \frac y{\sin y^2 \cdot 2y - x}\end{equation}

\hfill

\noindent Evaluate $\displaystyle \frac{dy}{dx}$ at the point.

\begin{equation}\frac{dy}{dx}\Bigg|_{\left(\sqrt{\frac{2}{\pi}}, -\sqrt{\frac{\pi}{2}} \right)} = \frac y{\sin y^2 \cdot 2y - x} = \frac{-\sqrt{\frac{\pi}{2}}}{\sin\left(\left(-\sqrt{\frac{\pi}{2}}\right)^2 \right)\cdot 2\left(-\sqrt{\frac{\pi}{2}}\right) -\sqrt{\frac{2}{\pi}}} =\frac{\sqrt{\frac{\pi}{2}}}{ \sqrt{2\pi} +\sqrt{\frac{2}{\pi}}}\end{equation}

\hfill

\noindent Recall: $y-y_0 = m(x-x_0)$, where $m$ is the slope. Substitute $m$ with $(2)$ and find the tangent line.

\begin{equation*}
\boxed{y+\sqrt{\frac{\pi}{2}} = \frac{\displaystyle\sqrt{\frac{\pi}{2}}}{ \displaystyle\sqrt{2\pi} +\sqrt{\frac{2}{\pi}}}\left(x-\sqrt{\frac{2}{\pi}}\right)}
\end{equation*}

\hfill

\noindent 4) Let $S(t),\, V(t),\, a(t)$ represent the surface area, the volume, and the length of one side of the object, respectively, as a function of time. We may write the following.

\begin{equation*}
S(t) = 6a^2(t), \quad V(t) = a^3(t)
\end{equation*}

\hfill

\noindent Given that at $t=t_0$, $V(t_0) = 216, \,S'(t_0)=-36$. Using the relationship with the sides,

\begin{align*}
&V(t_0) = a^3(t_0)= 216\,\rightarrow\, a(t_0) = 6\\
&S'(t_0) = 12a(t_0)a'(t_0)=-36\\
&\therefore 12\cdot6\cdot a'(t_0)=-36 \,\rightarrow\, a'(t_0)=-\frac12
\end{align*}

\begin{equation*}
\boxed{a'(t_0) = -\frac12\,\text{cm/h}}
\end{equation*}

\newpage

\noindent 5)

\hfill

\noindent a) Let $f(x) = \mathrm{e}^x + x$. $f$ is continuous and differentiable for all $x\in\mathbb{R}$.

\begin{equation*}
    f(-1) = \mathrm{e}^{-1} -1 = \frac1{\mathrm{e}} - 1, \quad f(0) = \mathrm{e} - 0 = \mathrm{e}
\end{equation*}

\hfill

\noindent Since $f(-1) < 0$ and $f(0) > 0$ and $f$ is continuous on the interval $[-1, 0]$, by IVT, there is at least one point $x_1$ that satisfies $f(x_1) = 0$. Assume that there is another distinct root $x_2$. Rolle's theorem states that if $f$ is continuous on a particular interval with endpoints having the same function value, there exists a point $c$ on that interval such that $f'(c) = 0$ there.

\begin{equation*}
f'(c) = \mathrm{e}^c + 1 \geq 1 \quad [\mathrm{e}^c > 0]
\end{equation*}

\hfill

\noindent This yields a contradiction. Therefore, there is \textit{only} one root.

\hfill

\noindent b) The expression is defined $\forall x\in \mathbb{R}$. Let us find the limit at infinity and the limit at negative infinity.

\begin{equation*}\lim_{x\to \infty}(\mathrm{e}^x+x)=\infty\,\quad\lim_{x\to -\infty}(\mathrm{e}^x+x)=-\infty\end{equation*}

\hfill

\noindent There are no vertical or horizontal asymptotes. However, there is a slant asymptote. Attempt a long polynomial division and we will find that the slant asymptote is $y=x$. Verify with the following limit:

\begin{equation*}\lim_{x\to-\infty}\left[(\mathrm{e})-x\right]=\lim_{x\to-\infty}\mathrm{e}^x=0\end{equation*}

\hfill 

\noindent Take the first and second derivatives.

\begin{equation*}y' = \mathrm{e}^x + 1,\quad y'' = \mathrm{e}^x\end{equation*}

\hfill 

\noindent We see that there are no critical or inflection points either. Now, set up a table and see what the graph looks like.

\begin{center}
    \large
    \begin{tabular}{ |c| c| } 
    \hline
        $x$ & $(-\infty, \infty)$\\
        \hline
        $y$ & $(-\infty, \infty)$\\
        \hline
        $y'$ sign & + \\
        \hline
        $y''$ sign & + \\
        \hline
    \end{tabular}
\end{center}

\hfill

\begin{center}
\begin{tikzpicture}
  \begin{axis}[
    axis lines = center,
    xlabel = $x$, ylabel = $y$,
    domain=-4:4,
    samples=200,
    ymin=-4, ymax=4,
    xmin=-4, xmax=4,
    axis line style={->},
    scale=1.3,
    ]
    \addplot[blue, thick] {e^x+x};

    \draw[dashed, red] (axis cs:-5,-5) -- (axis cs:5,5);
  \end{axis}
\end{tikzpicture}
\end{center}

\noindent 6)

\begin{center}
\begin{tikzpicture}
  \begin{axis}[
    axis lines = middle,
    xlabel = $x$,
    ylabel = $y$,
    domain=-2:2,
    samples=200,
    xtick={-2,-1,1,2},
    ytick={2,-2},
    ymin=-2, ymax=2,
    xmin=-2, xmax=2,
    width=12cm,
    height=8cm,
  ]

    % Fill area between the curves
    \addplot [
      name path=upper,
      domain=-1:1,
      samples=200,
      draw=none
    ] {-x^2 + 1};

    \addplot [
      name path=lower,
      domain=-2:2,
      samples=200,
      draw=none
    ] {abs(x) - 1};

    \addplot [
      fill=blue!30,
      draw=none
    ] fill between[of=upper and lower, soft clip={domain=-2:2}];

    \addplot [
      thick,
      blue,
    ] {-x^2 + 1};

    \addplot [
      thick,
      blue,
    ] {abs(x) - 1};
    
    \draw[black] (axis cs:-1,0) -- (axis cs:1,0);
    \draw[black] (axis cs:0,1) -- (axis cs:0,-1);
    \node at (-0.1, 1.2) {$1$};
    \node at (-0.15, -1.2) {$-1$};

  \end{axis}
\end{tikzpicture}
\end{center}

\hfill

\noindent The area of the region is as follows.

\begin{align*}\mathrm{I} & = \int_{-1}^1\left[(-x^2+1) - (|x|-1)\right]\, dx = \int_{-1}^0(-x^2+x +2)\, dx +\int_{0}^1(-x^2-x+2)\, dx\end{align*}

\hfill

\noindent 7) We'll use integration by parts.

\[
\left.
\begin{array}{ll}
\displaystyle\ln x=u\,\rightarrow\,\frac1x\,dx=du  \\
\displaystyle x\,dx= dv \,\rightarrow\,\frac{x^2}2=v
\end{array}
\right\}\quad
\begin{array}{ll}
\mathrm{I}&=\displaystyle\int x\ln x\,dx = \frac{x^2}2\cdot\ln x-\int\frac{x^2}2\cdot \frac1x \,dx \\\\&=\displaystyle\frac{x^2}2\cdot\ln x-\int\frac{x}2\,dx=\boxed{\frac{x^2}2\cdot\ln x - \frac{x^2}4+c,\,c\in\mathbb{R}}
\end{array}
\]

\end{document} 