\documentclass{article}
\usepackage{amsmath}
\usepackage{amssymb}
\usepackage[a4paper, top=25mm, bottom=25mm, left=25mm, right=25mm]{geometry}
\usepackage{pgfplots}
\usepackage{mathtools}
\pgfplotsset{compat=1.18}

\begin{document}
\pagestyle{empty}
\large

\begin{center}
2011-2012 Spring \\MAT123-[Instructor] Final\\(29/05/2012)\\Time: 15:00 - 16:50\\Duration: 110 minutes
\end{center}

\noindent 1. Find the length of the curve $y^2=4(x+1)^3$ for $0\leq x\leq1,\: y>0$.

\hfill

\noindent 2. Given $f(x)=x+2x^2+x^3$, find $\left(f^{-1}\right)'(4)$.

\hfill

\noindent 3. Evaluate the following integrals.

\hfill

(a) $\displaystyle\int\cos^3x\sin^2x\,dx$ \ \ \ (b) $\displaystyle\int\frac{x^2}{\sqrt{16-x^2}}\,dx$ \ \ \ (c) $\displaystyle\int\frac{x^3-1}{x^3-x}\,dx$ \ \  (d) $\displaystyle\int x^{123}\ln x\,dx$

\hfill

\noindent 4. Evaluate $\displaystyle\lim_{x\to\infty}(\ln x)^{\textstyle\frac1x}$.

\hfill

\noindent 5.

\hfill

\noindent (a) Determine the series  $\displaystyle\sum_{n=1}^\infty{\frac5{3^n}}$ converges or diverges. Give reasons for your answer.

\hfill

\noindent (b) Determine the series $\displaystyle\sum_{n=1}^\infty\cos\left(\frac1{5^n}\right)$ converges or diverges. Give reasons for your answer.

\hfill

\noindent (c) Use the Ratio Test to determine if the series $\displaystyle\sum_{n=1}^\infty\frac{n!}{\mathrm{e}^{2n}}$ converges or diverges.

\hfill

\noindent (d) Use the Integral Test to determine if the series $\displaystyle\sum_{n=1}^\infty\frac n{\mathrm{e}^{n^2}}$ converges or diverges. Be sure to check that the conditions of the integral test are satisfied.

\hfill

\noindent (Bonus)

\hfill

\noindent (a) Geometrically, what does $f'(x)$ mean?

\hfill

\noindent (b) If $f(t)$ describes the displacement of an object in time $t$, what is $f'(t)$?

\newpage

\begin{center}
2011-2012 Spring Final (29/05/2012) Solutions\\
(Last update: 30/08/2025 00:25)
\end{center}

\noindent 1. $y$ is implicitly defined as a function of $x$. Differentiate each side and solve for $\displaystyle \frac{dy}{dx}$.

\[\frac d{dx}\left(y^2\right)=\frac d{dx}\left(4(x+1)^3\right)\implies 2y\frac{dy}{dx}=12(x+1)^2\implies\frac{dy}{dx}=\frac{6(x+1)^2}y\]

\hfill

\noindent Since we're interested in the upper part of the curve (i.e., $y>0$), $y=2(x+1)^{3/2}$.

\[\frac{dy}{dx}=\frac{6(x+1)^2}{2(x+1)^{3/2}}=3\sqrt{x+1}\]

\hfill

\noindent The length of a curve defined by $y=f(x)$ whose derivative is continuous on the interval $a\leq x\leq b$ can be evaluated using the integral

\[S=\int_a^b\sqrt{1+\left(\frac{dy}{dx}\right)^2}\,dx.\]

\hfill

\noindent Set $\displaystyle a=0,\:b=1,\:\frac{dy}{dx}=3\sqrt{x+1}$ and find the length.

\[S=\int_0^1\sqrt{1+\left(3\sqrt{x+1}\right)^2}\,dx=\int_0^1\sqrt{9x+10}\,dx\]

\hfill

\noindent Let $u=9x+10$, then $du=9\,dx$.

\begin{align*}
S&=\int_0^1\sqrt{9x+10}\,dx=\int\frac19\sqrt{u}\,du=\frac19\cdot\frac23u^{3/2}+c=\frac2{27}\left(9x+10\right)^{3/2}\bigg|_0^1\\\\&=\boxed{\frac2{27}\left(19^{3/2}-10^{3/2}\right)}
\end{align*}

\hfill

\noindent 2. The derivative of $f^{-1}$ at a point can be calculated using the rule

\[\left(f^{-1}\right)'(x)=\frac1{f'\left(f^{-1}(x)\right)}\]

\hfill

\noindent Find the point where $f(x)=4$. We could intuitively say $f(1)=4$ because $f(1)=1+2\cdot1^3+1^3=4$. Therefore,

\[\left(f^{-1}\right)'\left(4\right)=\frac1{f'\left(f^{-1}(4)\right)}=\frac1{f'(1)}\]

\hfill

\noindent Calculate the derivative of $f$ at the point $x=1$.

\[f'(x)=1+4x+3x^2\implies f'(1)=1+4\cdot1+3\cdot1^2=8\]

\hfill

\noindent So,

\[\left(f^{-1}\right)'\left(4\right)=\boxed{\frac18}\]

\newpage

\noindent 3.

\hfill

\noindent (a)
\begin{align*}
\mathrm{I}&=\int\cos^3x\sin^2x\,dx\qquad\left[\sin^2+\cos^2x=1\right]\\\\&=\int\cos x\cdot\left(1-\sin^2x\right)\cdot\sin^2x\,dx
\end{align*}

\noindent Let $u=\sin x$, then $du=\cos x\,dx$.

\hfill

\begin{align*}
\mathrm{I}&=\int\cos x\cdot\left(1-\sin^2x\right)\cdot\sin^2x\,dx=\int\left(1-u^2\right) u^2\,du=\int\left(u^2-u^4\right)\,du=\frac{u^3}3-\frac{u^5}5+c\\\\&=\boxed{\frac{\sin^3x}3-\frac{\sin^5x}5+c}
\end{align*}

\hfill

\noindent (b) Let $x=4\sin u$, then $dx=4\cos u\,du$ for $\displaystyle -\frac\pi2<u<\frac\pi2$.

\begin{align*}
\mathrm{I}&=\int\frac{x^2}{\sqrt{16-x^2}}\,dx=\int\frac{16\sin^2u}{\sqrt{16-16\sin^2u}}\cdot4\cos u\,du=\int\frac{16\sin^2u\cos u}{\left|\cos u\right|}\,du\:\left[\cos u>0\right]\\\\&=\int16\sin^2u\,du=16\int\left(1-\cos^2u\right)\,du=16\int\frac{1-\cos2u}2\,du=8\left(u-\frac{\sin2u}2\right)+c\\\\&=8u-8\sin u\cos u+c
\end{align*}

\hfill

\noindent Recall: $x=4\sin u$. Then

\[x^2=16\sin^2u\implies x^2=16-16\cos^2u\implies\cos^2u=\frac{16-x^2}{16}\implies\cos u=\sqrt{1-\frac{x^2}{16}}\]

\[\sin u=\frac{x}4\implies u=\arcsin\frac x4\]

\hfill

\noindent Rewrite the integral.

\[\mathrm{I}=\boxed{8\arcsin\frac x4-2x\sqrt{1-\frac{x^2}{16}}+c,\quad c\in\mathbb{R}}\]

\newpage

\noindent (c) Use the method of partial fraction decomposition.

\begin{align*}
\mathrm{I}&=\int\frac{x^3-1}{x^3-x}\,dx=\int\frac{(x-1)\left(x^2+x+1\right)}{x(x-1)(x+1)}\,dx=\int\frac{x^2+x+1}{x^2+x}\,dx=\int\left(1+\frac1{x^2+x}\right)\,dx\\\\&=\int\,dx+\int\frac1{x(x+1)}\,dx=x+\int\left(\frac A{x}+\frac B{x+1}\right)\,dx
\end{align*}

\begin{align*}
A(x+1)+B(x)&=1\\
x(A+B)+A&=1\\
\therefore A+B=0\quad[\text{eliminate}\:x]\,&\rightarrow\,A=1\implies B=-1
\end{align*}

\[\mathrm{I}=x+\int\left(\frac Ax+\frac B{x+1}\right)dx=x+\int\left(\frac1x-\frac1{x+1}\right)dx=\boxed{x+\ln\left|x\right|-\ln\left|x+1\right|+c,\: c\in\mathbb{R}}\]

\hfill

\noindent (d) Use the method of integration by parts.

\[\left.\begin{array}{c}
\displaystyle u=\ln x\implies du=\frac1x\,dx\\[1em]
\displaystyle dv=x^{123}\,dx\implies v=\frac{x^{124}}{124}
\end{array}\right\}\rightarrow\int u\,dv=uv-\int v\,du\]

\[\mathrm{I}=\ln x\cdot\frac{x^{124}}{124}-\int\frac{x^{124}}{124}\cdot\frac1x\,dx=\boxed{\frac{\ln x\cdot x^{124}}{124}+\frac{x^{124}}{124^2}+c,\quad c\in\mathbb{R}}\]

\hfill

\noindent 4. Let $L$ be the value of the limit.

\[L=\lim_{x\to\infty}(\ln x)^{\textstyle\frac1x}\]

\hfill

\noindent Take the logarithm of both sides. We can take the logarithm inside the limit because the expression is continuous for $x>0$. After that, apply L'Hôpital's rule where $0/0$ or $\infty/\infty$ forms occur.

\begin{align*}
\ln(L)&=\ln\left(\lim_{x\to\infty}(\ln x)^{\textstyle\frac1x}\right)=\lim_{x\to\infty}\ln\left[\ln(x)^{\textstyle\frac1x}\right]=\lim_{x\to\infty}\frac{\ln(\ln x)}x\quad\left[\frac\infty\infty\right]\\\\&\overset{\text{L'H.}}{=}\lim_{x\to\infty}\frac{\displaystyle\frac1{\ln x}\cdot\frac1x}1=\lim_{x\to\infty}\frac1{x\ln x}=0
\end{align*}

\hfill

\noindent Since $\ln(L)=0$, $\boxed{L=1}$.

\newpage

\noindent 5.

\hfill

\noindent (a) Since the numerator is constant, we can take it out of the summation.

\[\sum_{n=1}^\infty\frac5{3^n}=5\cdot\sum_{n=1}^\infty\frac1{3^n}=5\sum_{n=1}^\infty\left(\frac13\right)^n\]

\hfill

\noindent This is a geometric series with $\displaystyle r=\frac13<1$. Therefore, the series converges.

\hfill

\noindent (b) Take the limit of the sequence at infinity. We can take the limit inside the trigonometric function because it is continuous everywhere.

\[\lim_{n\to\infty}\cos\left(\frac1{5^n}\right)=\cos\left(\lim_{n\to\infty}\frac1{5^n}\right)=\cos(0)=1\neq0\]

\hfill

\noindent By the $n$th Term Test for divergence, the series diverges.

\hfill

\noindent (c) Let $\displaystyle a_n=\frac{n!}{\mathrm{e^{2n}}}$. Then,

\[\lim_{n\to\infty}\left|\frac{a_{n+1}}{a_n}\right|=\lim_{n\to\infty}\left|\frac{(n+1)!}{\mathrm{e}^{2(n+1)}}\cdot\frac{\mathrm{e}^{2n}}{n!}\right|=\lim_{n\to\infty}\left|\frac{n+1}{\mathrm{e}^2}\right|=\infty>1\]

\hfill

\noindent By the Ratio Test, the series diverges.

\hfill

\noindent (d) Let $a_n=f(n)$, where $n\in\mathbb{N}$. The function $\displaystyle f(x)=\frac x{\mathrm{e}^{x^2}}$ is positive, continuous and decreasing for $x>1$.

\[\left.\begin{array}{c}
x>0\\
\mathrm{e}^{x^2}>0
\end{array}\right\}\:\text{for}\: x>1\implies\frac x{\mathrm{e}^{x^2}}>0\]

\hfill

\noindent $\mathrm{e}^{x^2}$ grows at a higher rate than $x$. Therefore, $f$ is decreasing. The expressions are continuous for $x>1$. We may now apply the Integral Test. Handle the improper integral with the limit.

\[\int_1^\infty\frac x{\mathrm{e}^{x^2}}\,dx=\lim_{R\to\infty}-\frac12\mathrm{e}^{-x^2}\bigg|_1^R=\lim_{R\to\infty}-\frac12\left(\mathrm{e}^{-R^2}-\mathrm{e}^{-1}\right)=\frac1{2\mathrm{e}}\quad\left(\text{converges}\right)\]

\hfill

\noindent By the Integral Test, the series converges.

\hfill

\noindent (Bonus)

\hfill

\noindent (a) Let $y=f(x)$ be a continuous function on a bounded interval, and let $f$ be differentiable on the same interval except possibly at the endpoints. Then $f'(x)$ gives the first derivative. $f'(x)$ gives the instantaneous rate of change of the function at a certain point and it gives the slope of the line that is tangent to the graph of the function at that point.

\hfill

\noindent (b) Given $f(t)$ describes the displacement, $f'(t)$ corresponds to the instantaneous velocity of the object.

\end{document}