\documentclass{article}
\usepackage{amsmath}
\usepackage{amssymb}
\usepackage[a4paper, top=25mm, bottom=25mm, left=25mm, right=25mm]{geometry}
\usepackage{pgfplots}
\usepackage{cancel}
\usepackage{mathtools}
\pgfplotsset{compat=1.18}

\begin{document}
\pagestyle{empty}
\large

\begin{center}
2012-2013 Fall \\MAT123-[Instructor02]-02, [Instructor05]-05, Final\\(23/01/2013)
\end{center}

\noindent 1. Determine whether each given sequence with $n$th term converges or diverges. Evaluate the limit of each convergent sequence. Explain all your work, and write clearly.

\hfill

(a) $\displaystyle a_n=(-1)^n\,n\sin\left(\frac1n\right)$ \ \ \ (b) $\displaystyle a_n=\mathrm{e}^{\cos\left(\textstyle\frac1n\right)}$

\hfill

\noindent 2. Determine whether the following series converge or diverge. Give reasons for your answers.

\hfill

(a) $\displaystyle\sum_{n=1}^\infty\frac n{\left(3+n^2\right)^{3/4}}$ \ \ \ (b) $\displaystyle\sum_{n=2}^\infty\frac1{\sqrt n}\ln\left(\frac{n+1}{n-1}\right)$ \ \ \ (c) $\displaystyle\sum_{n=1}^\infty\frac{(2n)!}{5^n\left(n!\right)^2}$ \ \ \ (d) $\displaystyle\sum_{n=1}^\infty\frac1{n^2}\mathrm{e}^{1/n}$

\hfill

\noindent 3. Integrate the following functions and write each step in detail.

\hfill

(a) $\displaystyle\int\frac{dx}{\mathrm{e}^x+1}$ \ \ \ (b) $\displaystyle\int x\arcsin x\,dx$

\hfill

\noindent 4. Find the length of the curve $\displaystyle y=\int_0^x\sqrt{\cos\left(4t\right)}\,dt$ for $0\leq x\leq \pi/8$.

\hfill

\noindent 5. For the function $\displaystyle f(x)=\frac x{x^2-4}$,

\hfill

(a) Find all the asymptotes of $f$.

\hfill

(b) Find the intervals of increase or decrease.

\hfill

(c) Find the local maximum and minimum values, if any.

\hfill

(d) Find the intervals of concavity and the inflection points, if any.

\hfill

(e) Sketch the graph of $f$.

\newpage

\begin{center}
2012-2013 Fall Final (23/01/2013) Solutions\\
(Last update: 30/08/2025 00:29)
\end{center}

\noindent 1.

\hfill

\noindent (a) Evaluate the limit of the non-alternating part at infinity.

\[\lim_{n\to\infty}n\sin\left(\frac1n\right)\overset{n=\frac1u}{=}\lim_{u\to0}\frac{\sin u}u=1\]

\hfill

\noindent However, for odd values of $n$, the limit at infinity becomes negative. On the other hand, for even values of $n$, the limit at infinity becomes positive. Therefore, the limit at infinity does not exist. So, the sequence diverges.

\hfill

\noindent (b) The exponential function $\mathrm{e}^x$ and the trigonometric function $\cos x$ are continuous everywhere.

\[\lim_{n\to\infty}a_n=\lim_{n\to\infty}\mathrm{e}^{\displaystyle\cos\left(\frac1n\right)}=\mathrm{e}^{\displaystyle\cos\left(\lim_{n\to\infty}\frac1n\right)}=\mathrm{e}^{\cos0}=\mathrm{e}^1=\mathrm{e}\]

\hfill

\noindent The sequence converges to $\boxed{\mathrm{e}}$.

\hfill

\noindent 2.

\hfill

\noindent (a) Let $a_n=f(n)$. Define $\displaystyle f(x)=\frac x{\left(3+x^2\right)^{3/4}}$. The function is continuous for $x>1$ because the numerator and the denominator are continuous for $x>1$ and $\left(3+x^2\right)^{3/4}\neq0,\:\forall x\in\mathbb{R}$.

\[\left.\begin{array}{c}
x>0\\
\left(3+x^2\right)^{3/4}>0
\end{array}\right\}\:\text{for}\:x>1\implies\frac x{\left(3+x^2\right)^{3/4}}>0\]

\hfill

\noindent For $x>1$, $x<x^{3/2}=\left(x^2\right)^{3/4}<\left(3+x^2\right)^{3/4}$. The denominator grows faster than the numerator. Therefore, the function is decreasing.

\hfill

\noindent We may now apply the Integral Test. Handle the improper integral by taking the limit.

\[\int_1^\infty\frac x{\left(3+x^2\right)^{3/4}}\,dx=\lim_{R\to\infty}2\left(3+x^2\right)^{1/4}\bigg|_1^R=2\lim_{R\to\infty}\left[(3+R^2)^{1/4}-\left(3+1^2\right)^{1/4}\right]=\infty\]

\hfill

\noindent Since the integral diverges, the series also diverges.

\hfill

\noindent (b) $\ln(1+x)<x$ for $x>-1$. Therefore,

\[\frac1{\sqrt n}\ln\left(\frac{n+1}{n-1}\right)=\frac1{\sqrt n}\ln\left(1+\frac2{n-1}\right)<\frac1{\sqrt n}\cdot\frac2{n-1}\]

\hfill

\noindent Since $n\geq2$, we have the inequality $\displaystyle 2n-2\geq n\implies\frac2n\geq\frac1{n-1}$.

\hfill

\[\frac1{\sqrt n}\cdot\frac2{n-1}\leq\frac1{\sqrt n}\cdot\frac4n=\frac4{n^{3/2}}\]

\hfill

\noindent Now, let $\displaystyle a_n=\frac1{\sqrt n}\ln\left(\frac{n+1}{n-1}\right)$ and $\displaystyle b_n=\frac4{n^{3/2}}$. Apply the Direct Comparison Test.

\[0<a_n<b_n\implies0<\frac1{\sqrt n}\ln\left(\frac{n+1}{n-1}\right)<\frac4{n^{3/2}}\]

\hfill

\noindent $b_n$ converges by the $p$-series test because $3/2>1$. Since $b_n$ converges, by the Direct Comparison Test, $a_n$ also converges.

\noindent (c) Apply the Ratio Test. Let $\displaystyle a_n=\frac{(2n)!}{5^n\left(n!\right)^2}$.

\begin{align*}\lim_{n\to\infty}\left|\frac{a_{n+1}}{a_n}\right|&=\lim_{n\to\infty}\left|\frac{(2(n+1))!}{5^{n+1}\left((n+1)!\right)^2}\cdot\frac{5^n\left(n!\right)^2}{(2n)!}\right|\\\\&=\lim_{n\to\infty}\left|\frac{(2n+2)\cdot(2n+1)\cdot((2n)!)\cdot(n!)^2\cdot5^n}{5^n\cdot5\cdot(n+1)^2\cdot(n!)^2\cdot((2n)!)}\right|\\\\&=\lim_{n\to\infty}\left|\frac{(2n+2)(2n+1)}{5(n+1)^2}\right|=\lim_{n\to\infty}\left|\frac{4n^2+6n+2}{5n^2+10n+5}\right|\end{align*}

\hfill

\noindent Now, take the corresponding function and evaluate the limit using L'Hôpital's rule.

\[\lim_{x\to\infty}\left|\frac{4x^2+6x+2}{5x^2+10x+5}\right|\overset{\text{L'H.}}{=}\lim_{x\to\infty}\left|\frac{8x+6}{10x+10}\right|\overset{\text{L'H.}}{=}\lim_{x\to\infty}\left|\frac{8}{10}\right|=\frac45<1\]

\hfill

\noindent By the Ratio Test, the series converges absolutely. Since the series converges absolutely, the series converges.

\hfill

\noindent (d) Let $\displaystyle a_n=\frac1{n^2}\mathrm{e}^{1/n}$. Define $\displaystyle f(x)=\frac1{x^2}\mathrm{e}^{1/x}$. The function is continuous for $x>1$ because the expressions $\displaystyle\frac1{x^2}$ and $\mathrm{e}^{1/x}$ are continuous for $x>1$ and $x^2\neq0,\:\forall x>1$.

\[\left.\begin{array}{c}
x^2>0\\
\mathrm{e}^{1/x}>0
\end{array}\right\}\:\text{for}\:x>1\implies\frac{\mathrm{e}^{1/x}}{x^2}>0\]

\hfill

\noindent For $x>1$, $\mathrm{e}^{1/x}$ tends to $1$ and $x^2$ is increasing. Therefore, the function is decreasing for $x>1$.

\hfill

\noindent We may now apply the Integral Test. Handle the improper integral by taking the limit.

\[\int_1^\infty\frac{\mathrm{e}^{1/x}}{x^2}\,dx-=\lim_{R\to\infty}-\mathrm{e}^{1/x}\bigg|_1^\infty=\lim_{R\to\infty}\left(-\mathrm{e}^{1/R}+\mathrm{e}^{1}\right)=\mathrm{e}-1\quad(\text{convergent})\]

\hfill

\noindent Since the integral converges, by the Integral Test, the series also converges.

\newpage

\noindent 3.

\hfill

\noindent (a) Add and subtract $\mathrm{e}^x$ in the numerator.

\[\int\frac{dx}{\mathrm{e}^x+1}=\int\frac{1+\mathrm{e}^x-\mathrm{e}^x}{\mathrm{e}^x+1}\,dx=\int\frac{\mathrm{e}^x+1}{\mathrm{e}^x+1}\,dx-\int\frac{\mathrm{e}^x}{\mathrm{e}^x+1}\,dx\]

\hfill

\noindent The result of the integral on the left is $x$. To calculate the integral on the right, use the $u$-substitution. Let $u=\mathrm{e}^x+1$, then $du=\mathrm{e}^x\,dx$.

\[x-\int\frac{du}{u}=x-\ln|u|+c=\boxed{x-\ln\left|\mathrm{e}^x+1\right|+c=x-\ln\left(\mathrm{e}^x+1\right)+c,\:c\in\mathbb{R}}\:\left[\mathrm{e}^x+1>0\right]\]

\hfill

\noindent (b) Solve the integral using the method of integration by parts.

\[\left.\begin{array}{c}
\displaystyle u=\arcsin x\implies du=\frac1{\sqrt{1-x^2}}\,dx\\[1em]
\displaystyle dv=x\,dx\implies v=\frac{x^2}2
\end{array}\right\}\rightarrow\int u\,dv=uv-\int v\,du\]

\hfill

\begin{equation}\int x\arcsin x\,dx=\frac{x^2}2\arcsin x-\int\frac{x^2}{2\sqrt{1-x^2}}\,dx\end{equation}

\hfill

\noindent Now, we need to find the result of the integral on the right. Let $x=\sin u$ for $\displaystyle\frac\pi2<u<\frac\pi2$. Then $dx=\cos u\,du$.

\begin{align*}\int\frac{x^2}{2\sqrt{1-x^2}}\,dx&=\int\frac{\sin^2u}{2\sqrt{1-\sin^2u}}\cdot\cos u\,du=\int\frac{\sin^2u\cos u}{2\sqrt{\cos^2u}}\,du\quad\left[\sin^2 u+\cos^2u=1\right]\\\\&=\int\frac{\sin^2{u}\cos u}{2\left|\cos u\right|}\,du\quad\left[\cos u > 0\right]\\\\&=\frac12\int\sin^2u\,du=\frac12\int\left(1-\cos^2u\right)\,du=\frac12\int\left(\frac{1-\cos2u}2\right)\,du\\\\&=\frac u4-\frac{\sin2u}8+c=\frac u4-\frac{\sin u\cos u}4+c,\quad c\in\mathbb{R}\end{align*}

\hfill

\noindent Recall the equation $x=\sin u$.

\[x=\sin u\implies\arcsin x=u\]
\[x=\sin u\implies x^2=\sin^2u=1-\cos^2u\implies\cos u=\sqrt{1-x^2}\]

\hfill

\noindent Rewrite the result of the last integral.

\[\int\frac{x^2}{2\sqrt{1-x^2}}\,dx=\frac14\left(\arcsin x-x\sqrt{1-x^2}\right)\]

\hfill

\noindent Rewrite (1).

\[\int x\arcsin x\,dx=\boxed{\frac{x^2}2\arcsin x-\frac14\arcsin x+\frac14x\sqrt{1-x^2}+c,\quad c\in\mathbb{R}}\]

\hfill

\noindent 4. The length of a curve defined by $y=f(x)$ whose derivative is continuous on the interval $a\leq x\leq b$ can be evaluated using the integral.

\[S=\int_a^b\sqrt{1+\left(\frac{dy}{dx}\right)^2}\,dx.\]

\hfill

\noindent Find $\displaystyle\frac{dy}{dx}$.

\[\frac{dy}{dx}=\frac d{dx}\int_0^x\sqrt{\cos(4t)}\,dt\]

\hfill

\noindent By the Fundamental Theorem of Calculus, $\displaystyle\frac{dy}{dx}$ can be rewritten as

\hfill

\[\frac{dy}{dx}=\sqrt{\cos(4x)}\]

\hfill

\noindent Set $\displaystyle a=0,\:b=\frac\pi8,\:\frac{dy}{dx}=\sqrt{\cos(4x)}$ and then find the length.

\begin{align*}S&=\int_0^{\pi/8}\sqrt{1+\left(\sqrt{\cos(4x)}\right)^2}\,dx=\int_0^{\pi/8}\sqrt{1+\cos(4x)}\,dx\quad\left[\cos(4x)=2\cos^2(2x)-1\right]\\\\&=\int_0^{\pi/8}\sqrt{2\cos^2(2x)}\,dx=\sqrt2\int_0^{\pi/8}\left|\cos(2x)\right|\,dx\quad\left[\cos(2x)>0\right]\\\\&=\sqrt2\int_0^{\pi/8}\cos(2x)\,dx=\frac{\sqrt2}2\sin(2x)\bigg|_0^{\pi/8}=\frac{\sqrt2}2\left(\sin\frac\pi4-\sin0\right)=\boxed{\frac12}\end{align*}

\noindent 5.

\hfill

\noindent (a) Find the horizontal asymptotes.
\[\lim_{x\to\pm\infty}\frac{1}{x^2-4}=0\]

\hfill

\noindent Find the vertical asymptotes. The expression is undefined for $x=\pm2$.

\[\lim_{x\to2^+}\frac{1}{x^2-4}=\lim_{x\to-2^+}\frac{1}{x^2-4}=\infty\]
\[\lim_{x\to2^-}\frac{1}{x^2-4}=\lim_{x\to-2^-}\frac{1}{x^2-4}=-\infty\]

\[\boxed{\text{The horizontal asymptote is } y= 0. \text{ The vertical asymptotes are } x=\pm2.}\]

\hfill

\noindent (b) Compute the first derivative and set it to $0$ to find the critical points. Apply the product rule appropriately.

\[f'(x)=\frac{1\cdot\left(x^2-4\right)-x\cdot(2x)}{\left(x^2-4\right)^2}=-\frac{x^2+4}{\left(x^2-4\right)^2}\]

\noindent $f$ is increasing where $f'(x)>0$ and decreasing where $f'(x)<0$. Therefore,

\[\boxed{\text{f is decreasing everywhere except at the undefined points.}}\]

\hfill

\noindent (c) $\boxed{\text{No local maximum or minimum values exist.}}$

\hfill

\noindent (d) Compute the second derivative.

\[f''(x)=-\frac{2x\cdot\left(x^2-4\right)^2-\left(x^2+4\right)\cdot2\cdot(x^2-4)\cdot(2x)}{\left(x^2-4\right)^4}=\frac{2x^3+24x}{\left(x^2-4\right)^3}\]

\[
\boxed{
\begin{array}{c}
\displaystyle \text{An inflection point occurs at }x=0.\\$f$ \text{ is concave up for }-2<x<0\:\cup\: x>2.\:$f$ \text{ is concave down for }x<-2\:\cup\:0<x<2.
\end{array}}\]

\hfill

\noindent (e)
\begin{center}
\begin{tikzpicture}
\begin{axis}[
    domain=-6:6,
    ymin=-4, ymax=4,
    samples=500,
    axis lines=middle,
    xlabel={$x$},
    ylabel={$y$},
    restrict y to domain=-4:4,
    enlargelimits=true,
    grid=none,
    xtick={-6,-4,-2,0,2,4,6},
    ytick={-6,-4,-2,0,2,4,6},
    unbounded coords=jump,
    clip=true,
    scale=2
]
\addplot[blue, thick] {x/(x^2 - 4)};
\draw[dashed, red] (axis cs: -6,0.02) -- (axis cs: 6,0.02);
\draw[dashed, red] (axis cs: 2,-4) -- (axis cs: 2,4);
\draw[dashed, red] (axis cs: -2,-4) -- (axis cs: -2,4);

\end{axis}
\end{tikzpicture}
\end{center}

\end{document}