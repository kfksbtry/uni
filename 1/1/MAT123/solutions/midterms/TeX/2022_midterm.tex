\documentclass{article}
\usepackage{amsmath}
\usepackage{amssymb}
\usepackage[a4paper, top=25mm, bottom=25mm, left=25mm, right=25mm]{geometry}
\usepackage{pgfplots}
\usepackage{mathtools}
\pgfplotsset{compat=1.18}

\begin{document}
\pagestyle{empty}
\large

\begin{center}
2022-2023 Fall \\MAT123-02,05 Midterm\\(23/11/2022)
\end{center}

\noindent 1. Evaluate the following limits without using L'Hôpital's rule.

\hfill

(a) $\displaystyle \lim_{x\to 0} \frac{\left|\sin x\right|}{x}$ \ \ \ (b) $\displaystyle \lim_{x\to 0^+}x\mathrm{e}^{\displaystyle \cos(1/x)}$ \ \ \ (c) $\displaystyle \lim_{x\to 0}{\frac{\sqrt{1+\tan x} -\sqrt{1-\sin x} }{x^3}}$

\hfill

\noindent 2. A balloon is released at point $A$ rises vertically with a constant speed of 5 m/s. Point $B$ is level with and 100 m distant from the point $A$. How fast is the angle of elevation of the balloon at $B$ changing when the balloon is 200 m above $A$?

\hfill

\noindent 3.

\hfill

\noindent (a) State IVT and MVT.

\hfill

\noindent (b) By IVT and MVT, show that the equation $x^{123}+2x^{85} + 3x^{17} + 4x-1 = 0$ has exactly one solution.

\hfill

\noindent 4. Find the tangent line to the curve defined implicitly by the equation

\[\sin\left(y^2\mathrm{e}^{2x}\right)+\sqrt{\pi}y=x^2+\pi\]

\noindent at $\left(0, \sqrt\pi\right).$

\hfill

\noindent 5. Evaluate $\displaystyle \lim_{x \to \infty} \left(\frac{\ln x}{x}\right)^{1/x}$.

\hfill

\noindent 6. Sketch the graph of $\displaystyle f(x) = \frac{x^2-2}{(x-1)^2}$.

\newpage

\begin{center}
2022-2023 Fall Midterm (23/11/2022) Solutions\\
(Last update: 29/08/2025 22:59)
\end{center}

\noindent 1.

\hfill

\noindent (a) Find the one-sided limits.
\[\lim_{x\to 0^+} \frac{\left|\sin x\right|}{x}=\lim_{x\to 0^+}\frac{\sin x}{x}=1,\quad\quad \lim_{x\to 0^-} \frac{\left|\sin x\right|}{x}=\lim_{x\to0^-}\left(-\frac{\sin x}{x}\right)=-\lim_{x\to0^-}\frac{\sin x}{x}=-1\]

\hfill

\noindent The one-sided limits are not equal. Therefore, the limit does not exist.

\hfill

\noindent (b) We have $-1\leq\cos(1/x)\leq1$ for all $x\in\mathbb R-\{0\}$. So, $\mathrm{e}^{-1}\leq\mathrm{e}^{\cos(1/x)}\leq\mathrm{e}^1$.
\[x\mathrm{e}^{-1}\leq x\mathrm{e}^{\cos(1/x)}\leq x\mathrm{e}\implies\lim_{x\to0^+} x\mathrm{e}^{-1}=\lim_{x\to0^+}x\mathrm{e}=0\implies\lim_{x\to0^+}x\mathrm{e}^{\cos(1/x)}=\boxed0\]

\noindent By the squeeze theorem, the limit is 0.

\hfill

\noindent (c)
\begin{align*}L&=\displaystyle\lim_{x\to0}\frac{\sqrt{1+\tan x}-\sqrt{1-\sin x}}{x^3}
\\\\&=\lim_{x\to0}\left(\frac{\sqrt{1+\tan x}-\sqrt{1-\sin x}}{x^3}\cdot\frac{\sqrt{1+\tan x}+\sqrt{1-\sin x}}{\sqrt{1+\tan x}+\sqrt{1-\sin x}}\right)\\\\&=\lim_{x\to0}\frac{(1+\tan x)-(1-\sin x)}{x^3\cdot\left(\sqrt{1+\tan x}+\sqrt{1-\sin x}\right)}=\lim_{x\to0}\frac{\tan x+\sin x}{x^3\cdot\left(\sqrt{1+\tan x}+\sqrt{1-\sin x}\right)}\\\\&=\lim_{x\to0}\frac{\sin x+\sin x\cdot\cos x}{\cos x\cdot x^3\cdot\left(\sqrt{1+\tan x}+\sqrt{1-\sin x}\right)}\\\\&=\lim_{x\to0}\frac{1+\cos x}{\cos x\cdot x^2\cdot\left(\sqrt{1+\tan x}+\sqrt{1-\sin x}\right)}\cdot\lim_{x\to0}\frac{\sin x}{x}\quad\left[\lim_{x\to0}\frac{\sin x}x=1\right]\\\\&=\lim_{x\to0}\left[\frac{1+\cos x}{\cos x\cdot x^2\cdot\left(\sqrt{1+\tan x}+\sqrt{1-\sin x}\right)}\cdot\frac{1-\cos x}{1-\cos x}\right]\\\\&=\lim_{x\to0}\left[\frac{1-\cos^2x}{\cos x\cdot x^2\cdot(1-\cos x)\cdot\left(\sqrt{1+\tan x}+\sqrt{1-\sin x}\right)}\right]\quad\left[\sin^2x+\cos^2x=1\right]\\\\&=\lim_{x\to0}\frac{\sin^2x}{x^2}\cdot\lim_{x\to0}\frac1{\cos x\cdot(1-\cos x)\cdot\left(\sqrt{1+\tan x}+\sqrt{1-\sin x}\right)}\\\\&=\lim_{x\to0}\frac1{\cos x}\cdot\lim_{x\to0}\frac1{1-\cos x}\cdot\lim_{x\to0}\frac1{\sqrt{1+\tan x}+\sqrt{1-\sin x}}=1\cdot\lim_{x\to0}\frac1{1-\cos x}\cdot\frac12 =\boxed{\infty}\end{align*}

\hfill

\noindent 2. Let $g(t),\,f(t)$ represent the distance between point $A$ and point $B$, and the distance between point $A$ and the balloon, respectively. We may express the angle as follows.

\[\theta(t)=\arctan{\frac{f(t)}{g(t)}}\]

\hfill

\noindent The first derivative of $\theta$ gives the rate of change of the angle. Apply the chain rule accordingly.

\[\theta'(t)=\frac1{\displaystyle 1+\frac{f^2(t)}{g^2(t)}}\cdot\frac{f'(t)\cdot g(t) - f(t) \cdot g'(t)}{g^2(t)}=\frac{f'(t)\cdot g(t) - f(t) \cdot g'(t)}{g^2(t)+f^2(t)}\]

\hfill

\noindent At $t=t_0$, $f(t_0)= 200,\,g(t_0) =100,\,f'(t_0)=5,\,g'(t_0) = 0$. The reason why $g'(t_0) = 0$ is that the distance between the points does not change over time. Calculate $\theta'(t_0)$.

\[\theta'(t_0)=\frac{5\cdot 100 - 200 \cdot 0}{100^2+200^2}=\frac{500}{50000}=\boxed{\frac1{100}\,\mathrm{rad/s}} \]

\hfill

\noindent 3.

\hfill

\noindent (a) Let $f(x)$ be continuous on $[a,b]$ and differentiable on $(a,b)$. The IVT states that since $f$ is continuous on the interval, $f$ takes any value on $[f(a), f(b)]$. The MVT states that since $f$ is differentiable on the interval provided the continuity, there is at least one point such that the slope of the line that passes through the endpoints is equal to the slope of the line that is tangent to that point.

\hfill

\noindent (b) Let $f(x) = x^{123}+2x^{85} + 3x^{17} + 4x-1$. Since this is a polynomial expression, $f$ is continuous and differentiable everywhere. Arbitrarily choose $x=-1$ and $x=1$ to make calculations easy. By IVT, $f$ takes any value on $[f(-1), f(1)]$.

\[f(-1) = -11,\quad f(1) =9\]

\hfill

\noindent $f$ must have at least one root $x_1$ on $[-1, 1]$ by IVT. Now, we need to prove that there is \textit{only} one. We assume that there is another distinct root $x_2$. Since $f(x_1) = f(x_2) = 0$, at some point $c$, the first derivative of the function at this point is $0$.

\[f'(c) = 123x^{122}+170x^{84} + 51x^{16} + 4 \geq 0+ 0+ 0+ 4 = 4\]

\hfill

\noindent $f'(c)>0$. However, this is a contradiction. Therefore, there is \textit{only} one root.

\hfill

\noindent 4. Differentiate both sides.

\[\frac d{dx}\left[\sin(y^2\mathrm{e}^{2x})+\sqrt\pi y\right]=\frac{d}{dx}\left(x^2+\pi\right)\]
\[\cos\left(y^2\mathrm{e}^{2x}\right)\cdot\left(2y{\frac{dy}{dx}\mathrm{e}^{2x}}+y^2\mathrm{e}^{2x}\cdot2\right)+\sqrt\pi\frac{dy}{dx}=2x\]
\[2y{\mathrm{e}^{2x}}\cos\left(y^2\mathrm{e}^{2x}\right)\frac{dy}{dx}+2y^2\mathrm{e}^{2x}\cos\left(y^2\mathrm{e}^{2x}\right)+\sqrt\pi\frac{dy}{dx}=2x\]
\[\frac{dy}{dx}\left[2y\mathrm{e}^{2x}\cos\left(y^2\mathrm{e}^{2x}\right)+\sqrt\pi\right]=2x-2y^2{\mathrm{e}^{2x}}\cos\left(y^2\mathrm{e}^{2x}\right)\]

\begin{equation}\frac{dy}{dx}=\frac{2x-2y^2{\mathrm{e}^{2x}}\cos\left(y^2\mathrm{e}^{2x}\right)}{2y\mathrm{e}^{2x}\cos\left(y^2\mathrm{e}^{2x}\right)+\sqrt\pi}\end{equation}

\hfill

\noindent Evaluate (1) at $\left(0, \sqrt\pi\right)$.

\[\left.\frac{dy}{dx}\right|_{\left(0, \sqrt\pi\right)}=\frac{2\cdot0-2\left(\sqrt\pi\right)^2{\mathrm{e}^{0}}\cos\left(\left(\sqrt\pi\right)^2\mathrm{e}^{0}\right)}{2\sqrt\pi\mathrm{e}^{0}\cos\left(\left(\sqrt\pi\right)^2\mathrm{e}^{0}\right)+\sqrt\pi}=\frac{2\pi}{-\sqrt\pi}=-2\sqrt\pi\]

\hfill

\noindent Use the straight line formula. $y-y_0=m(x-x_0)$, where $m=\left.\dfrac{dy}{dx}\right|_{\left(0,\sqrt\pi\right)}$

\[\boxed{y=\sqrt\pi(1-2x)}\]

\hfill

\noindent 5. Let $L$ be the value of the limit.

\[L=\lim_{x\to\infty}\left(\frac{\ln x}x\right)^{1/x}\qquad\left[\infty^0\right]\]
\[\ln(L)=\ln\left[\lim_{x\to\infty}\left(\frac{\ln x}x\right)^{1/x}\right]\]

\hfill

\noindent Take the logarithm inside the limit because the expression is continuous for $x>0$.

\begin{align*}\ln(L)&=\lim_{x\to\infty}\ln\left[\left(\frac{\ln x}x\right)^{1/x}\right]=\lim_{x\to \infty}\left[\frac{\ln\left(\frac{\ln x}x\right)}x\right]\quad\left[\frac\infty\infty\right]\\\\&\overset{\text{L'H.}}{=}\lim_{x\to\infty} \frac{\dfrac1{\frac{\ln x}{x}}\cdot\dfrac{\frac1x\cdot x-\ln x\cdot1}{x^2}}{1}=\lim_{x\to\infty}\left({\frac x{\ln x}\cdot\frac{1-\ln x}{x^2}}\right)=\lim_{x\to\infty}\frac{1-\ln x}{x\ln x}\\\\&=\lim_{x\to\infty}\frac1{x\ln x}-\lim_{x\to\infty}\frac{\ln x}{x\ln x}=0-\lim_{x\to\infty}\frac1x=0\end{align*}

\hfill

\noindent If $\ln(L) = 0$, then $\boxed{L = 1}$.

\hfill

\noindent 6. First off, find the domain. The expression is undefined when the denominator is zero. Therefore, $(x-1)^2\neq0\,\rightarrow\,x\neq1$. The only vertical asymptote occurs at $x = 1$.

\[\mathcal{D}=\mathbb{R}-\{1\}\]

\hfill

\noindent Let us find the limit at infinity.

\[\lim_{x\to \infty}\frac{x^2-2}{(x-1)^2}\overset{\text{L'H.}}{=}\lim_{x\to \infty}\frac{2x}{2(x-1)}\overset{\text{L'H.}}{=}\lim_{x\to\infty}\frac{2}{2}=1\]

\noindent Similarly,

\[\lim_{x\to-\infty}\frac{x^2-2}{(x-1)^2}=1\]

\hfill

\noindent The horizontal asymptote occurs only at $y=0$.

\hfill

\noindent Take the first derivative by applying the quotient rule.

\[y'=\frac{(2x)\cdot(x-1)^2-\left(x^2-2\right)\cdot2(x-1)}{(x-1)^4}=\frac{4-2x}{(x-1)^3}\]

\hfill

\noindent $y'$ is undefined for $x=1$, and $y'=0$ for $x=2$. Since $1$ is not in the domain, the \textit{only} critical point is $x = 2$.

\hfill

\noindent Take the second derivative.

\[y''=\frac{(-2)\cdot(x-1)^3-(4-2x)\cdot3(x-1)^2}{(x-1)^6}=\frac{4x-10}{(x-1)^4}\]

\hfill

\noindent The only inflection point occurs at $x=\dfrac52$.

\hfill

\noindent Consider some values of the function. Eventually, set up a table and see what the graph looks like in certain intervals.

\[\,f\left(-\sqrt2\right)=f\left(\sqrt2\right)=0,\:f(0)=-2,\:f(2)=2,\:f(5/2)=17/9\]

\begin{center}
    \large
    \begin{tabular}{ |c| c c c c c c c| } 
    \hline
        $x$ & $\left(-\infty, -\sqrt2\right)$ & $\left(-\sqrt2, 0\right)$&$\left(0, 1\right)$ & $\left(1,\sqrt2\right)$ & $\left(\sqrt2, 2\right)$ & $\left(2, \frac52\right)$ & $\left(\frac52, \infty\right)$  \\
        \hline
        $y$ & $(0, 1)$ &$(-2,0)$ & $(-\infty, -2)$& $(-\infty, 0)$& $(0, 2)$& $\left(\frac{17}9, 2\right)$& $\left(1, \frac{17}9\right)$\\
        \hline
        $y'$ sign & - & - & - &+&+&-&- \\
        \hline
        $y''$ sign & - &- &-&-&-&-&+ \\
        \hline
    \end{tabular}
\end{center}

\hfill

\begin{center}
\begin{tikzpicture}
  \begin{axis}[
    axis lines = center,
    xlabel = $x$, ylabel = $y$,
    domain=-12:12,
    samples=3200,
    ymin=-5, ymax=2,
    xmin=-11, xmax=11,
    restrict y to domain=-5:5,
    enlargelimits=true,
    axis line style={->},
    scale=1.6,
    ]
    \addplot[blue, thick] {(x^2-2)/(x-1)^2};

    \draw[dashed, red] (axis cs:-12,1) -- (axis cs:-1.5,1);
    \draw[dashed, red] (axis cs:0,1) -- (axis cs:12,1);
    \draw[dashed, red] (axis cs:1,2.2) -- (axis cs:1,-5.2);
    
    \draw[dashed, black] (axis cs:2,0) -- (axis cs:2,2);
    %\draw[dashed, black] (axis cs:5/2,0) -- (axis cs:5/2,17/9);
    
    \draw[dashed, black] (axis cs:0,2) -- (axis cs:2,2);
    %\draw[dashed, black] (axis cs:0,17/9) -- (axis cs:5/2,17/9);

  \end{axis}
\end{tikzpicture}
\end{center}
\end{document}