\documentclass{article}
\usepackage{amsmath}
\usepackage{amssymb}
\usepackage[a4paper, top=25mm, bottom=25mm, left=25mm, right=25mm]{geometry}
\usepackage{pgfplots}
\usepackage{mathtools}
\pgfplotsset{compat=1.18}

\begin{document}
\pagestyle{empty}
\large

\begin{center}
2011-2012 Spring \\MAT123-[Instructor] Midterm I\\(06/04/2012)\\Time: 15:00 - 16:45\\Duration: 105 minutes
\end{center}

\noindent 1. Find the equation of the tangent line to the curve $x^2+2xy+y^2=4$ at the point $(3,-1)$.

\hfill

\noindent 2. Find the point on the parabola $y=\sqrt x$ which is the closest to the point $(2,0)$.

\hfill

\noindent 3. Evaluate the limit, if it exists, and explain your answer. Do not use L'Hôpital's rule.

\hfill

(a) $\displaystyle\lim_{x\to0}\sqrt{x^2+2x^3} \sin\left(\frac1x\right)$ \ \ \ (b) $\displaystyle \lim_{x\to3}\frac{x^2-9}{x^2-x-6}$ \ \ \ (c) $\displaystyle \lim_{x\to-\infty}\sqrt{x^2-4x}+x$

\hfill

(d) $\displaystyle\lim_{h\to0}\frac{(1+h)^{123}-1}{h}$

\hfill

\noindent 4. Find the derivatives of the following functions.

\hfill

(a) $f(x)=x^{\cos\left(x^3\right)}$ \ \ \ (b) $f(x) = \tan\left(\mathrm{e}^{2x}\sin(3x)\right)$

\hfill

(c) Find the second derivative $f''(x)$ of $\displaystyle f(x) =\ln\left(\frac{x^2}{x^2+4}\right)$.

\hfill

\noindent 5. For the function $\displaystyle f(x)=\frac{1}{x^2-4}$,

\hfill

(a) Find the vertical and horizontal asymptotes.

\hfill

(b) Find the intervals of increase or decrease.

\hfill

(c) Find the local maximum and minimum values, if any.

\hfill

(d) Find the intervals of concavity and the inflection points, if any.

\hfill

(e) Sketch the graph of $f$.

\newpage

\begin{center}
2011-2012 Spring Midterm I (06/04/2012) Solutions\\
(Last update: 29/08/2025 19:24)
\end{center}

\noindent 1. $y$ is implicitly defined as a function of $x$. Differentiate each side.

\[\frac{d}{dx}\left(x^2+2xy+y^2\right)=\frac{d}{dx}\,(4)\]
\[2x+2y\cdot1+2x\cdot\frac{dy}{dx}+2y\frac{dy}{dx}=0\]
\[\frac{dy}{dx}(2x+2y)=-2x-2y\]

\[\frac{dy}{dx}=-1\]

\hfill

\noindent Recall the equation of a straight line: $y-y_0=m(x-x_0)$, where $m$ is simply $\displaystyle \frac{dy}{dx}$. Therefore, the tangent line at $(3,-1)$ is as follows.

\[\boxed{y+1=-1(x-3)}\]

\hfill

\noindent Furthermore, since $\displaystyle \frac{dy}{dx}=-1$, the tangent line consists of every point of the upper line. In other words, the tangent line is the upper line itself. The equation $x^2+2xy+y^2=4$ forms two parallel straight lines in the $xy-$coordinate system.

\hfill

\noindent 2. Let $\left(x,\sqrt x\right)$ be a point on this parabola. The distance between the points can be expressed using the Pythagorean theorem as follows.

\[f(x)=(2-x)^2 + \left(\sqrt x - 0\right)^2=L^2\]

\hfill

\noindent Take the derivative of both sides and set $\displaystyle f'(x)=0$ to find the critical points.

\begin{equation*}
f'(x)=2(2-x)\cdot(-1) + 1 = 2x-3=0 \implies x=\frac{3}2
\end{equation*}

\hfill

\noindent Now, verify whether this is a local minimum by taking the second derivative.

\[f''(x)=(2x-3)'=2>0\]

\noindent Since this is a local minimum of $f$, the distance is closest at $\displaystyle x=\frac32$. The point we are looking for is

\[\boxed{\left(\frac32,\sqrt{\frac32}\right)}\]

\newpage

\noindent 3.

\hfill

\noindent (a) The inequality $\displaystyle-1 \leq \sin\left(\frac1x\right)\leq1$ holds for all $x\in\mathbb{R}$ except for $x=0$. For small $x$, we can multiply each side of the inequality by $\sqrt{x^2+2x^3}$. Using the squeeze theorem, the limit is equal to $0$.

\[-1\leq\sin\left(\frac1x\right)\leq1\]
\[-\sqrt{x^2+2x^3}\leq\sqrt{x^2+2x^3}\sin\left(\frac1x\right)\leq\sqrt{x^2+2x^3}\]

\[\lim_{x\to0}-\sqrt{x^2+2x^3}=\lim_{x\to0}\sqrt{x^2+2x^3}=0\implies\lim_{x\to0}\sqrt{x^2+2x^3}\sin\left(\frac1x\right)=\boxed0\]

\hfill

\noindent (b) Factor each side of the fraction and eliminate like terms.

\[\lim_{x\to3}\frac{x^2-9}{x^2-x-6}=\lim_{x\to3}\frac{(x-3)(x+3)}{(x-3)(x+2)}=\lim_{x\to3}\frac{x+3}{x+2}=\boxed{\frac65}\]

\hfill

\noindent (c) The expression is in the form $\infty-\infty$. Expand the expression by multiplying by its conjugate to eliminate the indetermination.

\begin{align*}
\lim_{x\to-\infty}\sqrt{x^2-4x}+x&=\lim_{x\to-\infty}\left[\left(\sqrt{x^2-4x} + x\right) \cdot\frac{\sqrt{x^2-4x}-x}{\sqrt{x^2-4x}-x}\right]=\lim_{x\to-\infty}\frac{x^2-4x-x^2}{\sqrt{x^2-4x}-x}\\\\&=\lim_{x\to-\infty}\frac{-4x}{\sqrt{x^2-4x}-x}=\lim_{x\to-\infty}\frac{-4x}{\displaystyle |x|\sqrt{1-\frac4x}-x}\\\\&=\lim_{x\to-\infty}\frac{-4}{\displaystyle-\sqrt{1-\frac4x}-1}=\lim_{x\to-\infty}\frac42=\boxed{2}\end{align*}

\hfill

\noindent (d) Recall the definition of the derivative of a function at a point. Let $f$ be a differentiable function, then

\[\lim_{h\to 0} \frac{f(x+h)-f(x)}{h} = f'(x)\]

\hfill

\noindent If we set $f(x) = x^{123}$, then we can differentiate $f$ at $x=1$.

\[\lim_{h\to0}\frac{(1+h)^{123}-1}{h}=f'(1)=123\cdot(1)^{122} = \boxed{123}\]

\newpage

\noindent 4.

\hfill

\noindent (a) Take the logarithm of each side to compute the derivative easily.

\[f(x)=x^{\cos\left(x^3\right)}\]
\[\ln(f(x))=\ln\left[x^{\cos\left(x^3\right)}\right]=\cos\left(x^3\right)\cdot\ln x\]
\[\frac{d}{dx}\left[\ln(f(x))\right] = \frac{d}{dx}\left[\cos\left(x^3\right)\cdot\ln x\right]\]
\[\frac{1}{f(x)}\cdot f'(x) = -\sin\left(x^3\right)\cdot 3x^2 \cdot \ln x + \cos\left(x^3\right)\cdot\frac1x\]

\[\boxed{f'(x) = x^{\cos\left(x^3\right)}\cdot\left[-\sin\left(x^3\right)\cdot 3x^2 \cdot \ln x + \cos\left(x^3\right)\cdot\frac1x\right]}\]

\hfill

\noindent (b) Apply the chain rule accordingly.

\begin{equation*}f(x) = \tan\left(\mathrm{e}^{2x}\sin(3x)\right)\end{equation*}
\begin{equation*}\boxed{f'(x) = \sec^2\left(\mathrm{e}^{2x}\sin(3x)\right)\cdot\left[\mathrm{e}^{2x}\cdot 2\cdot\sin(3x) + \mathrm{e}^{2x}\cdot\cos(3x)\cdot3\right]}\end{equation*}

\noindent (c) Compute the first and second derivatives, respectively, applying the chain rule and the quotient rule accordingly.

\begin{equation*}f(x) = \ln\left(\frac{x^2}{x^2+4}\right)\end{equation*}
\begin{equation*}f'(x)= \frac{x^2+4}{x^2}\cdot\frac{2x\cdot(x^2+4)-x^2\cdot2x}{\left(x^2+4\right)^2}=\frac{8}{x^3+4x}\end{equation*}
\begin{equation*}f''(x)= -\frac{8}{\left(x^3+4x\right)^2}\cdot\left(3x^2+4\right)=\boxed{-\frac{24x^2+32}{\left(x^3+4x\right)^2}}\end{equation*}

\hfill

\noindent 5.

\hfill

\noindent (a) Find the horizontal asymptotes.
\[\lim_{x\to\pm\infty}\frac{1}{x^2-4}=0\]

\hfill

\noindent Find the vertical asymptotes. The expression is undefined for $x=\pm2$.

\[\lim_{x\to2^+}\frac{1}{x^2-4}=\lim_{x\to-2^-}\frac{1}{x^2-4}=\infty\]
\[\lim_{x\to2^-}\frac{1}{x^2-4}=\lim_{x\to-2^+}\frac{1}{x^2-4}=-\infty\]

\[\boxed{\text{The horizontal asymptote is } y= 0. \text{ The vertical asymptotes are } x=\pm2.}\]

\hfill

\noindent (b) Compute the first derivative and set it to $0$ to find the critical points.

\[f'(x)=-\frac{1}{\left(x^2-4\right)^2}\cdot2x\]

\noindent $f$ is increasing where $f'(x)>0$ and decreasing where $f'(x)<0$.

\[\boxed{\text{f is decreasing for } x>0, \text{increasing for } x<0.}\]

\hfill

\noindent (c) The \textit{only} critical point occurs at $x=0$. Compute the second derivative to check whether this is a local minimum or local maximum.

\begin{equation*}f''(x)=-\frac{2\cdot\left(x^2-4\right)^2-2x\cdot2\cdot(x^2-4)\cdot(2x)}{\left(x^2-4\right)^4}=\frac{8+6x^2}{\left(x^2-4\right)^3}\end{equation*}

\hfill

\noindent $\displaystyle f''(0)=\frac{8+6\cdot0^2}{\left(0^2-4\right)^3}=-\frac18<0$. Therefore, $(0, f(0))$  is a local maximum.

\[
\boxed{
\begin{array}{c}
\displaystyle \text{No local minimums exist.}\\\displaystyle \text{The }\textit{only} \text{ local maximum occurs at } x=0,\text{ which is}\left(0,-\frac14\right).
\end{array}}\]

\hfill

\noindent (d) $8+6x^2 \geq 0$. Therefore, no inflection points. $f$ is concave up if $f''(x)>0$, concave down if $f''(x)<0$.

\[
\boxed{
\begin{array}{c}
\displaystyle \text{No inflection points exist. }\\$f$ \text{ is concave up for } x>2 \text{ and } x<-2.\:$f$ \text{ is concave down for } |x|<2.
\end{array}}\]

\hfill

\noindent (e)
\begin{center}
\begin{tikzpicture}
\begin{axis}[
    domain=-6:6,
    ymin=-4, ymax=4,
    samples=500,
    axis lines=middle,
    xlabel={$x$},
    ylabel={$y$},
    restrict y to domain=-4:4,
    enlargelimits=true,
    grid=none,
    xtick={-6,-4,-2,0,2,4,6},
    ytick={-6,-4,-2,0,2,4,6},
    unbounded coords=jump,
    clip=true,
    scale=2
]
\addplot[blue, thick] {1/(x^2 - 4)};
\draw[dashed, red] (axis cs: -6,0.02) -- (axis cs: 6,0.02);
\draw[dashed, red] (axis cs: 2,-4) -- (axis cs: 2,4);
\draw[dashed, red] (axis cs: -2,-4) -- (axis cs: -2,4);

\end{axis}
\end{tikzpicture}
\end{center}

\end{document}