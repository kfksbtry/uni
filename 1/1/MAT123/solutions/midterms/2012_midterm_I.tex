\documentclass{article}
\usepackage{amsmath}
\usepackage{amssymb}
\usepackage[a4paper, top=25mm, bottom=25mm, left=25mm, right=25mm]{geometry}
\usepackage{pgfplots}
\usepackage{mathtools}
\pgfplotsset{compat=1.18}

\begin{document}
\pagestyle{empty}
\large

\begin{center}
2011-2012 Fall \\MAT123-[Instructor02]-02, [Instructor05]-05 Midterm I\\(15/11/2012)\\Time: 13:00 - 15:00\\Duration: 120 minutes
\end{center}

\noindent 1. Evaluate the limits, if they exist, and explain your answer. Don't use L'Hôpital's rule.

\hfill

(a) $\displaystyle \lim_{x\to 1} \frac{x^2+x-2}{\sqrt x-1}$ \ \ \ (b) $\displaystyle \lim_{x\to 3} \frac{\sin(x^2-9)}{x-3}$ \ \ \ (c) $\displaystyle \lim_{x\to -\infty}\left(\sqrt{x^2-x+1}-\sqrt{x^2-2x}\right)$

\hfill

\noindent 2. Find the derivatives of the following functions.

\hfill

(a) $\displaystyle f(x) = \tan^3\left(4\sin^2(3x)\right)$ \ \ \ (b) $\displaystyle f(x) = \left(\cos x^2\right)^x$

\hfill

(c) Find $f'(0)$ of $\displaystyle f(x)=\ln\left(\frac{3^x}{3^x+1}\right)$

\hfill

\noindent 3. Evaluate $\displaystyle \lim_{x\to0}\left(\mathrm{e}^x-x\right)^{\frac1x}$.

\hfill

\noindent 4.

\hfill

(a) Let $F(x)$ be a one-to-one function with inverse $F^{-1}$. Define a new function

\begin{equation*}
p(x) = 1-2F\left(\frac x3\right)
\end{equation*}

Find a formula for $p^{-1}$ in terms of $F^{-1}$.

\hfill

(b) Find the derivative of the inverse of the function $f(x) = \arctan x + \mathrm{e}^{123x}$ at $x=1$.

That is, find $\left(f^{-1}\right)'(1)$.

\hfill

\noindent 5. Find the equation of the tangent line to the curve $x^2y^2-36x=37$ at $(-1, 1)$.

\hfill

\noindent 6. The length of a hypotenuse of a right triangle is constant at $5$ cm, and the
length of one of its sides is decreasing at a rate of $2$ cm/sec. Find the rate of change of the area of the triangle when this side is $3$ cm long.

\newpage

\begin{center}
2011-2012 Fall Midterm I (15/11/2012) Solutions\\
(Last update: 7/22/25 (22nd of July) 8:27 PM)
\end{center}

\noindent 1.

\hfill

\noindent (a) Multiply each side by the conjugate of the denominator to eliminate the indetermination.

\begin{align*}
\lim_{x\to1}\frac{x^2+x-2}{\sqrt x -1}&=\lim_{x\to1}\left[\frac{(x+2)(x-1)}{\sqrt x -1}\cdot \frac{\sqrt x + 1}{\sqrt x +1}\right]=\lim_{x\to1}\frac{(x+2)(x-1)\left(\sqrt x + 1\right)}{x-1}\\\\&=\lim_{x\to1}\left[(x+2)\left(\sqrt x + 1\right)\right] =3\cdot 2 = \boxed6
\end{align*}

\hfill

\noindent (b)
\begin{equation*}
\lim_{x\to 3}\frac{\sin(x^2-9)}{x-3}=\lim_{x\to 3}\left[\frac{\sin(x^2-9)}{x-3}\cdot\frac{x+3}{x+3}\right]=\lim_{x\to3}\left[\frac{\sin(x^2-9)}{x^2-9}\right]\cdot\lim_{x\to 3}(x+3)
\end{equation*}

\hfill

\noindent The value $\displaystyle \lim_{x\to0}\frac{\sin u}{u}$ can be evaluated by using the squeeze theorem, and it could be expected that we knew the value of this limit prior to the examination. Set $u=x^2-9$. So, the left-hand limit is $1$.

\begin{equation*}
=\lim_{x\to3}\left[\frac{\sin(x^2-9)}{x^2-9}\right]\cdot\lim_{x\to 3}(x+3) = 1\cdot 6 = \boxed6
\end{equation*}

\hfill

\noindent (c) This is an indeterminate ($\infty - \infty$) form. We expand the expression by using its conjugate and divide each side of the fraction by $x$ to eliminate the determination.

\begin{align*}&\lim_{x\to -\infty}\left(\sqrt{x^2-x+1}-\sqrt{x^2-2x}\right)\\\\&=\lim_{x\to -\infty}\left[\sqrt{x^2-x+1}-\sqrt{x^2-2x}\cdot\frac{\sqrt{x^2-x+1}+\sqrt{x^2-2x}}{\sqrt{x^2-x+1}+\sqrt{x^2-2x}}\right]\\\\&=\lim_{x\to-\infty}\frac{x^2-x+1-\left(x^2-2x\right)}{\sqrt{x^2-x+1}+\sqrt{x^2-2x}}=\lim_{x\to-\infty}\left(\frac{x+1}{\sqrt{x^2-x+1}+\sqrt{x^2-2x}}\cdot\frac{x}{x}\right)\\\\&=\lim_{x\to-\infty}\frac{\displaystyle\frac{x+1}x}{\displaystyle \frac{\sqrt{x^2-x+1}+\sqrt{x^2-2x}}x}=\lim_{x\to-\infty}\frac{\displaystyle1+\frac{1}x}{\displaystyle\sqrt{1-\frac1x+\frac1{x^2}}+\sqrt{1-\frac2x}}\\\\&=\frac{1+0}{\sqrt{1-0+0}+\sqrt{1-0}}=\boxed{\frac12}\end{align*}

\newpage

\noindent 2.

\hfill

\noindent (a) Apply the chain rule accordingly.
\begin{equation*}
\boxed{f'(x) = 3\tan^2\left(4\sin^2(3x)\right)\cdot \sec^2\left(4\sin^2(3x)\right)\cdot8\sin(3x)\cdot \cos(3x)\cdot 3}
\end{equation*}

\hfill

\noindent (b) Take the logarithm of both sides to differentiate easily.
\begin{equation*}\ln\left(f(x)\right)=\ln\left[\left(\cos x^2\right)^x\right]=x\ln\left(\cos x^2\right)\end{equation*}

\begin{equation*}\frac{d}{dx}\,\ln\left(f(x)\right)=\frac{d}{dx}\left[x\ln\left(\cos x^2\right)\right]\end{equation*}

\begin{equation*}\frac1{f(x)}\cdot f'(x)=1\cdot\left[\ln\left(\cos x^2\right)\right] + x\cdot\frac1{\cos x^2}\cdot\left(-\sin x^2\right)\cdot 2x\end{equation*}

\begin{equation*}f'(x)=f(x)\left[\ln\left(\cos x^2\right)-2x^2\cdot\tan x^2\right]\end{equation*}

\begin{equation*}\boxed{f'(x)=\left(\cos x^2\right)^x\cdot\left[\ln\left(\cos x^2\right)-2x^2\cdot\tan x^2\right]}\end{equation*}

\hfill

\noindent (c) Rewrite the right-hand side using the property of logarithms. Take the first derivative afterwards.

\begin{equation*}f(x)=\ln\left(\frac{3^x}{3^x+1}\right)=\ln\left(3^x\right)-\ln\left(3^x+1\right)\end{equation*}

\begin{equation*}f'(x)=\frac1{3^x}\cdot 3^x\cdot \ln(3) -\frac1{3^x+1}\cdot3^x\cdot\ln(3)=\ln(3)\cdot\left[1-\frac{3^x}{3^x+1}\right]\end{equation*}

\begin{equation*}f'(0)=\ln(3)\cdot\left[1-\frac{3^0}{3^0+1}\right]=\boxed{\frac{\ln3}2}\end{equation*}

\hfill

\noindent 3. Let $L$ be the value of the limit.

\begin{equation*}L = \lim_{x\to0}\left(\mathrm{e}^x-x\right)^{\frac1x}\end{equation*}
\begin{equation*}\ln(L) = \ln\left[\lim_{x\to0}\left(\mathrm{e}^x-x\right)^{\frac1x}\right]\end{equation*}

\hfill

\noindent The expression is defined for $x\neq0$. Therefore, we can take the logarithm function inside the limit. After that, apply L'Hôpital's rule to eliminate the indeterminate form.

\begin{align*}\ln(L)&=\lim_{x\to0}\ln\left[\left(\mathrm{e}^x-x\right)^{\frac1x}\right]=\lim_{x\to0}\frac{\ln\left(\mathrm{e}^x-x\right)}x\quad\left[\frac00\right]\\\\&\overset{\text{L'H.}}{=}\lim_{x\to0}\frac{\displaystyle\frac{1}{\mathrm{e}^x-x}\cdot\left(\mathrm{e}^x-1\right)}{1}=\lim_{x\to0}\frac{\mathrm{e}^x-1}{\mathrm{e}^x-x}=\frac{\mathrm{e}^0-1}{\mathrm{e}^0-0}=0\end{align*}

\hfill

\noindent If $\ln(L)=0$, then $\boxed{L=1}$.

\newpage

\noindent 4.

\hfill

\noindent (a) \[p(x)=1-2F\left(\frac x3\right)\]
\[p(x)-1=-2F\left(\frac x3\right)\]
\[\frac{1-p(x)}{2}=F\left(\frac x3\right)\]
\[F^{-1}\left(\frac{1-p(x)}{2}\right)=\frac x3\]
\[3F^{-1}\left(\frac{1-p(x)}{2}\right)=x\]
\[3F^{-1}\left(\frac{1-p\left(p^{-1}(x)\right)}{2}\right)=p^{-1}(x)\]

\[\boxed{p^{-1}(x)=3F^{-1}\left(\frac{1-x}{2}\right)}\]

\hfill

\noindent (b) Find a root so that $f(x_0) = \arctan(x_0)+\mathrm{e}^{123\cdot x_0} = 1$. We can intuitively say that the root is small because both $\arctan x$ and $\mathrm{e}^x$ are strictly increasing everywhere. Try $x=0$.

\begin{equation*}f(0) = \arctan 0 + \mathrm{e}^{123\cdot 0} = 0 + 1 = 1\end{equation*}

\hfill

\noindent Therefore, $f^{-1}(1)=0$. The derivative of an inverse function at a given point is

\begin{equation*}
\left(f^{-1}\right)'(1) = \frac1{f'(f^{-1}(1))}
\end{equation*}

\hfill

\noindent Calculate $f'(x)$ and then, $(f^{-1})(1)$.

\begin{equation*}
    f'(x)=\frac1{1+x^2}+123\mathrm{e}^{122x}
\end{equation*}

\begin{equation*}
    \left(f^{-1}\right)'(1)=\frac{1}{f'(0)}=\left( \frac1{1+0^2}+123\mathrm{e}^{122\cdot0} \right)^{-1}=\boxed{\frac1{124}}
\end{equation*}

\hfill

\noindent 5) Consider $y=f(x)$. Differentiate both sides implicitly.

\begin{equation*}\frac{d}{dx}\left(x^2y^2-36x\right) = \frac{d}{dx}\,37\end{equation*}

\begin{equation*}2xy^2+ x^2\cdot2y\cdot \frac{dy}{dx}-36 = 0\end{equation*}

\begin{equation*}x^2\cdot2y\cdot \frac{dy}{dx}=36 - 2xy^2\end{equation*}

\begin{equation*}\frac{dy}{dx}=\frac{36-2xy^2}{x^2\cdot2y}\end{equation*}

\hfill

\noindent Calculate $\displaystyle \frac{dy}{dx}$ at $(-1,1)$.

\hfill

\begin{equation}\frac{dy}{dx}\Bigg|_{(-1,1)}=\frac{36-2(-1)\cdot1^2}{(-1)^2\cdot2\cdot1} = 19\end{equation}

\hfill

\noindent Using the straight line formula, we find the tangent line. Recall: $y-y_0=m(x-x_0)$, where $m$ can be substituted with $(1)$.

\begin{equation*}\boxed{y-1=19(x+1)}\end{equation*}

\hfill

\noindent 6. Let $x(t),\,y(t),\,l(t)$ represent the lengths of the sides as functions of time. We can set up the following equation for the area of the right triangle.

\begin{equation*}A(t) = \frac{x(t)\cdot y(t)}{2}\end{equation*}

\hfill

\noindent Take the derivative of both sides.

\begin{equation}A'(t) =\frac12\left(x'(t)\cdot y(t) + x(t)\cdot y'(t) \right)\end{equation}

\hfill

\noindent We also know that, by the Pythagorean theorem,

\begin{equation*}l^2(t) = x^2(t)+y^2(t)\end{equation*}

\hfill

\noindent Take the derivative of both sides.

\begin{equation*}2l(t)l'(t)= 2x(t)x'(t)+2y(t)y'(t)\end{equation*}

\hfill

\noindent Since the length of the hypotenuse is constant, $l'(t) = 0$. Therefore,

\begin{equation}x(t)x'(t)=-y(t)y'(t)\end{equation}

\hfill

\noindent At $t=t_0$, we have $l(t_0) = 5,\,x(t_0)=3,\,x'(t_0)=-2$, and by the Pythagorean theorem, $y(t_0)=\sqrt{5^2-3^2}=4$. Calculate $y'(t_0)$ from $(3)$.

\begin{equation*}
y'(t_0)=-\frac{3\cdot (-2)}4 = \frac32
\end{equation*}

\hfill

\noindent Plug the necessary values into $(2)$ to find the rate of change of the area.

\begin{equation*}
A'(t_0) =\frac12\left((-2)\cdot4+3\cdot\frac32\right)=\boxed{-\frac74\,\text{cm}^2/\text{s}}
\end{equation*}

\end{document}