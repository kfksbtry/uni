\documentclass{article}
\usepackage{amsmath}
\usepackage{amssymb}
\usepackage[a4paper, top=25mm, bottom=25mm, left=25mm, right=25mm]{geometry}
\usepackage{pgfplots}
\usepackage{mathtools}
\pgfplotsset{compat=1.18}

\begin{document}
\pagestyle{empty}
\large

\begin{center}
2023-2024 Fall \\MAT123-02,05 Midterm\\(01/11/2023)
\end{center}

\noindent 1. Evaluate the following limits without using L'Hôpital's rule.

\hfill

(a) $\displaystyle \lim_{t\to 0} \frac{\tan^{-1} t}{\sin^{-1} t}$ \ \ \ (b) $\displaystyle \lim_{x\to 0^+} x\mathrm{e}^{\displaystyle \sin(1/x)}$

\hfill

\noindent 2. A spherical balloon is being filled with air in such a way that its radius is increasing at the constant rate of 2 cm/s. At what rate is the volume of the balloon increasing at the instant when its surface has area $4\pi$ cm$^2$?

\hfill

\noindent 3. (a) State MVT.

\hfill

\noindent (b) Using MVT, show that for all $x>0$,

\[
1+x < \mathrm{e}^x < 1+x\mathrm{e}^x
\]

\hfill

\noindent 4. Find the tangent line to the curve defined implicitly by the equation

\[
y^2x^x + xy = 2
\]

\hfill

\noindent at the point $(1,1)$. Note that $y=f(x)$.

\hfill

\noindent 5. Find the number $A$ so that $\displaystyle \lim_{x \to \infty} \left(\frac{x+A}{x-2A}\right)^{x}= 5$.

\hfill

\noindent 6. Sketch the graph of $\displaystyle f(x) = (\ln x)^2$.

\newpage

\begin{center}
2023-2024 Fall Midterm (01/11/2023) Solutions\\
(Last update: 8/21/25 (21st of August) 4:36 PM)
\end{center}

\noindent 1)

\hfill

\noindent (a)
\begin{align*}
\lim_{t\to 0} \frac{\tan^{-1} t}{\sin^{-1} t}&= \lim_{t\to 0}\left( \frac{\tan^{-1} t}{\sin^{-1}t} \cdot\frac tt\right) = \lim_{t\to 0}\left( \frac{\tan^{-1} t}{t} \cdot\frac1{\frac {\sin^{-1}t}t}\right)\\\\&=\lim_{t\to0}\frac{\tan^{-1}t}t\cdot\frac1{\displaystyle\lim_{t\to 0}\frac{\sin^{-1}t}t}=1\cdot1=\boxed1
\end{align*}

\noindent The limits above are well-known limits. If we're supposed to write these limits in the form of $\sin x$ and $x$, follow these steps.

\[\lim_{t\to0}\frac{\tan^{-1} t}{t}\overset{u=\tan^{-1} t}{=}\lim_{u\to0}\frac u{\tan u} = \lim_{u\to0}\frac{\cos u}{\frac{\sin u}u}=\frac{\displaystyle \lim_{u\to0}\cos u}{\displaystyle\lim_{u\to0}\frac{\sin u}u} = \frac11 = 1\]

\[\lim_{t\to0}\frac{\sin^{-1} t}{t}\overset{v=\sin^{-1} t}{=}\lim_{v\to0}\frac v{\sin v} =\frac1{\displaystyle\lim_{u\to0}\frac{\sin v}v} = \frac11 = 1\]

\hfill

\noindent (b)
\[-1\leq\sin(1/x) \leq 1\]
\[\mathrm{e}^{-1}\leq\mathrm{e}^{\sin(1/x)} \leq \mathrm{e}^1\]
\[x\mathrm{e}^{-1}\leq x\mathrm{e}^{\sin(1/x)} \leq x\mathrm{e}\]
\[\lim_{x\to0^+}x\mathrm{e}^{-1}\leq \lim_{x\to0^+}x\mathrm{e}^{\sin(1/x)} \leq\lim_{x\to0^+}x\mathrm{e}\]
\[0\leq \lim_{x\to0^+}x\mathrm{e}^{\sin(1/x)} \leq0\]

\hfill

\noindent By the squeeze theorem, the limit is $\boxed{0}$.

\hfill

\noindent 2) Let $V(t),\,S(t),\,r(t)$ represent the volume, surface area and radius, respectively. $r'(t) = 2$ for all $t$. The rate of change of volume at $t=t_0$ is

\[V'(t_0) = 4\pi r^2(t_0)r'(t_0) \quad\left[V(t) = \frac43 \pi r^3(t)\right] \]

\hfill

\noindent We also have $S(t_0) = 4\pi$. Using the surface area formula $S(t) = 4\pi r^2(t)$, we find that at $t=t_0$, the radius of the balloon is 1. Therefore, the rate of change of volume can now be evaluated.

\[V'(t_0)=4\pi\cdot1^2\cdot2=\boxed{8\pi\,\text{cm}^3\text{/s}}\]

\newpage

\noindent 3)

\hfill

\noindent (a) The MVT states that if a function $f$ is continuous on $(a,b)$ and differentiable on $[a,b]$, there is at least one point such that the slope of the line that passes through the endpoints is equal to the slope of the line that is tangent to that point.

\hfill

\noindent (b) Let $f(x)=\mathrm{e}^x$. $f$ is continuous on $[0,x]$ and differentiable on $(0,x)$.  There exists at least one point $c$ on $(0,x)$ such that

\[
f'(c) = \mathrm{e}^c=\frac{\mathrm{e}^x-1}x=\frac{f(x)-f(0)}{x-0}
\]

\hfill

\noindent From the inequality $0<c<x$,

\[\mathrm{e}^0<\mathrm{e}^c<\mathrm{e}^x\]
\[1<\frac{\mathrm{e}^x-1}x<\mathrm{e}^x\]
\[x<\mathrm{e}^x-1 <x\mathrm{e}^x\]
\[1+x<\mathrm{e}^x <1+x\mathrm{e}^x\]

\hfill

\noindent 4) Let us find the derivative of $x^x$ with respect to x.

\[y=x^x\]
\[\ln(y)=\ln(x^x) = x\ln x\]
\[\frac1y\cdot y'=1\cdot\ln x + x\cdot \frac1x = \ln x +1\]
\[y'=x^x(\ln x+1)\]

\hfill

\noindent We can now differentiate both sides of the very first equation.

\[\frac d{dx}\left(y^2x^x + xy\right) = \frac d{dx}\,2\]
\[2y\cdot y'\cdot x^x + y^2\cdot x^x(\ln x+1)+1\cdot y +x\cdot y'=0\]
\[y'\cdot\left(2y\cdot x^x + x\right)=-y-y^2\cdot x^x(\ln x+1)\]

\[y'=-\frac{y+y^2\cdot x^x(\ln x+1)}{2y\cdot x^x+x}\]

\hfill

\noindent Evaluating $y'$ at $(1,1)$ gives $\displaystyle-\frac23$. Using the straight line formula, we get

\[
\boxed{y-1 =-\frac23(x-1)}
\]

\newpage

\noindent 5) Take the logarithm of both sides of the equation and apply L'Hôpital's rule.

\begin{align*}
\ln(5)&=\ln\left[\lim_{x\to\infty}\left(\frac{\displaystyle1+\frac Ax}{\displaystyle1-\frac {2A}x}\right)^{x}\right]=\lim_{x\to\infty}\ln\left[\left(\frac{\displaystyle1+\frac Ax}{\displaystyle1-\frac {2A}x}\right)^{x}\right]=\lim_{x\to\infty}\left[x\ln\left(\frac{\displaystyle1+\frac Ax}{\displaystyle1-\frac {2A}x}\right)\right]\\\\&=\lim_{x\to\infty}\left\{x\left[\ln\left(1+\frac Ax\right)-\ln\left(1-\frac {2A}x\right)\right]\right\}\quad\left[\infty\cdot0\right]\\\\&=\lim_{x\to\infty}\frac{\displaystyle\ln\left(1+\frac Ax\right)-\ln\left(1-\frac {2A}x\right)}{\displaystyle\frac1x}\quad\left[\frac00\right]\\\\&\overset{\text{L'H.}}{=}\lim_{x\to\infty}\left[\frac{\displaystyle\left(\frac1{1+\frac Ax}\right)\cdot\left(-\frac{A}{x^2}\right)-\left(\frac1{1-\frac{2A}x}\right)\cdot\left(\frac{2A}{x^2}\right)}{\displaystyle -\frac1{x^2}}\right]\\\\&=\lim_{x\to\infty}\left[\frac{\displaystyle\left(-\frac{A}{x^2}\right)\cdot\left(\frac1{1+\frac{A}x}+\frac{2}{1-\frac{2A}x}\right)}{\displaystyle-\frac{1}{x^2}}\right]\\\\&=A\lim_{x\to\infty}\frac1{\frac Ax + 1}+A\lim_{x\to\infty}\frac2{1-\frac{2A}x} = A \cdot 1 + A \cdot 2 = 3A
\end{align*}

\hfill

\noindent If $\ln(5) = 3A$, then $\boxed{A =\frac{\ln5}3}$.

\hfill

\noindent 6) First off, find the domain. The expression is defined \textit{only} for $x>0$. The only vertical asymptote occurs at $x=0$.

\[\mathcal{D} = \mathbb{R}^+\]

\hfill

\noindent The limits as $x\to\infty$ and as $x\to0^+$ are:

\[\lim_{x\to \infty}(\ln x)^2=\infty,\quad \lim_{x\to0^+}(\ln x)^2=\infty \]

\hfill

\noindent Take the first derivative to find the critical points.

\[y'=2\ln x\cdot\frac1x\]

\hfill

\noindent The \textit{only} critical point is $x=1$.

\hfill

\noindent Take the second derivative by applying the quotient rule.

\[y''=2\cdot\frac{\frac1x \cdot x-\ln x \cdot 1}{x^2}=\frac{2-2\ln x}{x^2}\]

\hfill

\noindent The \textit{only} inflection point is $x=\mathrm{e}$.

\hfill

\noindent Consider some values of the function. Eventually, set up a table and see what the graph looks like in certain intervals.

\[\,f\left(1\right)=0,\,f(\mathrm{e})=1\]

\begin{center}
    \large
    \begin{tabular}{ |c| c c c| } 
    \hline
        $x$ & $\left(0, 1\right)$ & $\left(1, \mathrm{e}\right)$&$\left(\mathrm{e}, \infty\right)$  \\
        \hline
        $y$ & $(0, \infty)$ &$(0,1)$ & $(1, \infty)$ \\
        \hline
        $y'$ sign & - & + & + \\
        \hline
        $y''$ sign & + & + & - \\
        \hline
    \end{tabular}
\end{center}

\hfill

\begin{center}
\begin{tikzpicture}
  \begin{axis}[
    axis lines = center,
    xlabel = $x$, ylabel = $y$,
    xtick ={1,2,...,10},
    domain=0:12,
    samples=800,
    ymin=-0.1, ymax=5.1,
    xmin=-0.1, xmax=10,
    restrict y to domain=0:5.5,
    enlargelimits=true,
    axis line style={->},
    clip=true,
    scale=1.5,
    ]
    \addplot[blue, thick] {ln(x)*ln(x)};

    \draw[dashed, red] (axis cs:0.05,-0.5) -- (axis cs:0.05,5.4);
    
    \draw[dashed, black] (axis cs:e,0) -- (axis cs:e,1);
    \draw[dashed, black] (axis cs:0,1) -- (axis cs:e,1);
    \node at (axis cs:e,-0.2) {e};
    
  \end{axis}
\end{tikzpicture}
\end{center}
\end{document}