% Student Number: 2240357068
% Student Name: Baturay KAFKAS
% EEE @ Hacettepe University

% Last update: HH:MM DD/MM/YY

% My resource collection: https://github.com/ProBaturay/uni
% Circuits were set up in LTspice.

\documentclass{article}

\usepackage{comment}
\usepackage{graphicx}
\usepackage[top=25mm, bottom=25mm, left=25mm, right=25mm]{geometry}
\usepackage{amsmath}
\usepackage{moresize}
\usepackage{float}
\usepackage{fancyhdr}
\usepackage{booktabs}
\usepackage{fancyhdr}

\pagestyle{fancy}
\fancyhf{}
\fancyhead[R]{Baturay KAFKAS 2240357068 Electrical \& Electronics Engineering}

\rfoot{\thepage}
\renewcommand{\headrulewidth}{0pt} 
\renewcommand{\footrulewidth}{0pt}

\setcounter{page}{16}

\begin{document}
\large

{\textit{This part of the experiment is prepared with Online LaTeX Editor Overleaf, and the circuits are drawn in LTspice. Visit the website for the code here:}}

{\textbf{https://www.overleaf.com/read/ttjwtvcwwgyq\#14ff29}}
\vspace{4mm}
\hrule
\vspace{4mm}
{\Large \textbf{2. PRELIMINARY WORK}}

\vspace{4mm}

{\textbf{2.1} Explain why the adjusted value of $\text{R}_2$ is equal to the internal resistance of the basic meter in Fig. 4.}

\vspace{4mm}

{\textbf{Answer}: When the switch is open, the current is that the ammeter measures when it deflects full-scale. After the switch is closed, the current splits equally and the ammeter deflects half-scale. From Kirchhoff's Voltage Law, the voltage across the ammeter and R$_2$ must be equal. Since they have the same magnitude of current, from Ohm's Law, they have equal magnitudes of resistance. Therefore, the resistance of the ammeter equals the resistor's.}

%\vspace{8mm}

%{\textbf{Answer}: When the switch is open, the magnitude of the current, let it be I$_\text{1}$, is the current that the ammeter measures when it deflects full-scale. We find:}

%\vspace{8mm}

%{V$_\text{sup} = \text{I}_\text{1}\cdot (\text{R}_\text{1} + \text{R}_\text{m})$}

%\vspace{8mm}

%{When the switch is closed, the magnitude of the current, let it be I$_\text{2}$, is half I$_\text{1}$ because the ammeter deflects half-scale. Now that R$_2$ and R$_\text{m}$ are parallel as well.}

%\vspace{8mm}

%{I$\displaystyle_2 = \frac{\text{I}_1}{2}$}

%\vspace{4mm}

%{$\displaystyle\text{R}_\text{p} = \text{R}_2 \ // \ \text{R}_\text{m} = \frac{\text{R}_2 \cdot \text{R}_\text{m}}{\text{R}_2 + \text{R}_\text{m}}$}

%\vspace{4mm}

%{V$\displaystyle_\text{sup} = \text{I}_\text{2}\cdot \Big(\text{R}_\text{1} + \frac{\text{R}_2 \cdot\text{R}_\text{m}}{\text{R}_2 + \text{R}_\text{m}} \Big) = \text{I}_\text{2} \cdot \Big[\frac{\text{R}_1(\text{R}_2 + \text{R}_\text{m}) + \text{R}_2 \cdot \text{R}_\text{m}}{\text{R}_2 + \text{R}_\text{m}}\Big]$}

%\vspace{8mm}

%{Since the source provides constant voltage,}

%\vspace{8mm}

%{V$\displaystyle_\text{sup} = \text{I}_\text{1}\cdot (\text{R}_\text{1} + \text{R}_\text{m}) = \text{I}_\text{2} \cdot \Big[\frac{\text{R}_1(\text{R}_2 + \text{R}_\text{m}) + \text{R}_2 \cdot \text{R}_\text{m}}{\text{R}_2 + \text{R}_\text{m}}\Big]$}

%\vspace{4mm}

%{$\displaystyle2(\text{R}_\text{1} + \text{R}_\text{m}) = \Big[\frac{\text{R}_1(\text{R}_2 + \text{R}_\text{m}) + \text{R}_2 \cdot \text{R}_\text{m}}{\text{R}_2 + \text{R}_\text{m}}\Big]$}

%\vspace{4mm}

%{$\displaystyle \text{R}_\text{m}\text{R}_\text{1}\text{R}_\text{2} + 2\text{R}_\text{2}\text{R}_\text{m}^2 = \text{R}_\text{2} + \text{R}_\text{m}$}

%\vspace{4mm}

%{$\displaystyle \text{R}_\text{m}\text{R}_\text{2}(\text{R}_\text{1} + 2\text{R}_\text{m}) = \text{R}_\text{2} + \text{R}_\text{m}$}

%\vspace{4mm}

%{$\displaystyle \text{R}_\text{1} + 2\text{R}_\text{m} = \frac{\text{R}_\text{2} + \text{R}_\text{m}}{\text{R}_\text{2} \cdot \text{R}_\text{m}}$}

%\vspace{8mm}

%{Let's use the equations we found for V$\displaystyle_\text{sup}$.}

%\vspace{8mm}

%{$\displaystyle \text{I}_\text{1}\cdot (\text{R}_\text{1} + \text{R}_\text{m}) = \text{I}_\text{2}\cdot \Big(\text{R}_\text{1} + \frac{\text{R}_2+\text{R}_\text{m}}{\text{R}_2 \cdot \text{R}_\text{m}} \Big) $}

%\vspace{4mm}

%{$\displaystyle \text{I}_\text{1}\cdot (\text{R}_\text{1} + \text{R}_\text{m}) = \text{I}_\text{2}\cdot(2\text{R}_\text{1} + \text{I}_\text{2}) $}

\vspace{8mm}

{\textbf{2.2} Assume that we are given a basic meter having an internal resistance of 50 $\Omega$ and a full scale deflection of 2 mA}

{\quad a) Design an ammeter having a full scale deflection of 20 mA.}

{\quad b) Design a voltmeter to measure a full scale voltage of 20 V.}

\vspace{4mm}

{\textbf{Answer}:}

\vspace{4mm}

{a)}

\begin{figure}[H]
    \centering
    \includegraphics[width=0.5\linewidth]{Draft1.png}
\end{figure}

\vspace{4mm}

{$\text{I}_1 = 20\text{ mA} - \text{I}_1 = 20\text{ mA} - 2\text{ mA} = 18\text{ mA}$}

\vspace{4mm}

{$\displaystyle\text{I}_1\cdot\text{R}_1 = \text{I}_2\cdot\text{R}_2 \ \rightarrow \ \text{R}_1 = \frac{\text{I}_2\cdot\text{R}_2}{\text{I}_1} = \frac{2\text{ mA}\cdot\text{50 }\Omega}{18\text{ mA}} = \frac{50}{9}\ \Omega \approx 5.56 \ \Omega$}

\vspace{4mm}

{$\displaystyle \boxed{\text{I}_1 = 18 \text{ mA}, \ \text{R}_1 = 5.56\ \Omega}$}

\vspace{4mm}

\newpage

{b)}

\begin{figure}[H]
    \centering
    \includegraphics[width=0.5\linewidth]{Draft2.png}
\end{figure}

{$\displaystyle \text{V}_\text{max} = \text{I}_3\cdot {\text{R}_3} + \text{I}_3\cdot{\text{R}_4} = \text{I}_3\cdot ({\text{R}_3} + {\text{R}_4})$}

\vspace{4mm}

{$\displaystyle \text{R}_\text{4} = \frac{\text{V}_\text{max}}{\text{I}_3} - {\text{R}_3} = \frac{20 \ \text{V}}{2\text{ mA}} - {50 \ \Omega} = 9950 \ \Omega \ \rightarrow \ \boxed{\text{R}_\text{4} =  9950 \ \Omega}$}

\vspace{4mm}

\vspace{8mm}

{\textbf{2.3} Consider the circuit diagram in Fig. 3 and $\text{V}_{\text{DC}}$ = 3 V}

{\quad a) When a-b are short-circuited, determine the value of $\text{R}_\text{v}$ for full-scale deflection.}

{\quad b) Find the value of the unknown resistor $\text{R}_\text{x}$, if the meter current is 0.33 mA.}

\vspace{4mm}

{\textbf{Answer}:}

\vspace{4mm}

{a) $\displaystyle \text{R}_\text{v} = \text{R}_\text{T} - \text{R}_\text{m} = \frac{3 \ \text{V}}{1\text{ mA}} - 100 \ \Omega = 2900 \ \Omega\ \rightarrow \ \boxed{\text{R}_\text{v} =  2900 \ \Omega}$}

\vspace{4mm}

{b) $\displaystyle \text{R}_\text{x} = \text{R}_\text{T}-  \text{R}_\text{v} - \text{R}_\text{m} = \frac{3 \ \text{V}}{0.33\text{ mA}} - 2900 \ \Omega - 100 \ \Omega = 6000 \ \Omega \ \rightarrow \ \boxed{\text{R}_\text{x} =  6000 \ \Omega}$}

\vspace{8mm}

{\textbf{2.4} For the given circuit in Fig. 7, the voltage across the element $\text{R}_{\text{L}}$ is required to be 
measured using two voltmeters whose internal resistances are 300 k$\Omega$ and 6 M$\Omega$.}

\begin{figure}[H]
    \centering
    \includegraphics[width=0.5\linewidth]{msedge_IqjiFNlP82.png}
\end{figure}

{\quad a) Calculate the correct value.}

{\quad b) Calculate the values measured by each voltmeter and corresponding errors.}

\vspace{4mm}

{\textbf{Answer}:}

\vspace{4mm}

{a) $\displaystyle \text{V}_{\text{R}_\text{L}} = \frac{75}{75 + 150}\cdot 300 = 100\text{ V} \ \rightarrow \ \boxed{\text{V}_{\text{R}_\text{L}} = 100\text{ V}}$}

\vspace{8mm}

{b) With 300 k$\Omega$,}

\vspace{4mm}

{$\displaystyle \text{R}_{\text{ab}} = 75 \text{ k}\Omega\ //\ 300\text{ k}\Omega = 60\text{ k}\Omega\ $}

\vspace{4mm}

{$\displaystyle \text{V}_{\text{ab}} = \frac{60 \text{ k}\Omega}{60 \text{ k}\Omega + 150 \text{ k}\Omega} \cdot 300 \text{ V} = \frac{600}{7} \text{ V} \approx 85.71 \text{ V} $}

\vspace{4mm}

{Error: $\displaystyle \text{V}_{\text{R}_\text{L}} - \text{V}_{\text{ab}} \approx 14.29 \text{ V} $}

\vspace{16mm}

{With 6 M$\Omega$,}

\vspace{4mm}

{$\displaystyle \text{R}_{\text{ab}} = 75 \text{ k}\Omega\ //\ 6\text{ M}\Omega \approx 74.07\text{ k}\Omega\ $}

\vspace{4mm}

{$\displaystyle \text{V}_{\text{ab}} = \frac{74.07 \text{ k}\Omega}{74.07 \text{ k}\Omega + 150 \text{ k}\Omega} \cdot 300 \text{ V} \approx 99.17 \text{ V} $}

\vspace{4mm}

{Error: $\displaystyle \text{V}_{\text{R}_\text{L}} - \text{V}_{\text{ab}} \approx 0.83 \text{ V} $}

\vspace{8mm}

%% Example table:

%%\begin{center}
 %%   \large
 %%   \begin{tabular}{ |c| c c| } 
 %%   \hline
 %%         & 1 & 2\\
 %%       \hline
 %%       a)& 3 & 4 \\
 %%       \hline
 %%       b)& 5 & 6 \\
 %%       \hline
 %%   \end{tabular}
%%\end{center} %% Rearrange the table.

\end{document}