% Student Number: 2240357068
% Student Name: Baturay KAFKAS
% EEE @ Hacettepe University

% Last update: 17:45 28/04/25

% My resource collection: https://github.com/kfksbtry/uni

\documentclass{article}

\usepackage{comment}
\usepackage{graphicx}
\usepackage[top=25mm, bottom=25mm, left=25mm, right=25mm]{geometry}
\usepackage{amsmath}
\usepackage{moresize}
\usepackage{float}
\usepackage{fancyhdr}
\usepackage{booktabs}

\pagestyle{fancy}
\fancyhf{} 
\fancyhead[R]{Baturay KAFKAS 2240357068 Electrical \& Electronics Engineering}

\rfoot{\thepage}
\renewcommand{\headrulewidth}{0pt} 
\renewcommand{\footrulewidth}{0pt}

\setcounter{page}{42}

\begin{document}

\large

{\textit{This part of the experiment is prepared with Online LaTeX Editor Overleaf. Visit the website for the code here:}}

{\textbf{https://www.overleaf.com/read/xhmfsdvjwgmb\#13b5c5}}
\vspace{4mm}
\hrule
\vspace{4mm}
{\Large \textbf{1. PRELIMINARY WORK}}

\vspace{4mm} 

{\textbf{1.1} Explain how to test a semiconductor diode whether it is defective or not, by means of 
an ohmmeter?}

\vspace{4mm}

{\textbf{Answer}: Since the ohmmeter has its own battery inside, we may perform a test by applying voltage across the diode. At first, we connect the positive probe to the anode and the negative probe to the cathode. If the diode is not defective, we should obtain a low-resistance reading. On the other hand, we change the direction of the diode. In this case, we ought to obtain a high-resistance reading. Despite the direction of the diode, if a low-resistance reading is obtained, then the circuit is short-circuited, and vice versa for high-resistance readings (open-circuited). Hence, a defective semiconductor diode.}

\vspace{8mm}

{\textbf{1.2} For the circuit given in Fig. 1, D$_1$ is a diode, V$_1$=5V, R$_1$=1k$\Omega$, and R$_{\text{pot}}$=1k$\Omega$. Find the voltage V$_{\text{D}}$ and current I$_{\text{D}}$.}

\begin{figure}[H]
    \centering
    \includegraphics[width=0.5\linewidth]{msedge_uyI54WChBG.png}
\end{figure}

\vspace{4mm}

{\textbf{Answer}: We assume that the diode has ideal characteristics. Therefore, the voltage across R$_\text{pot}$ becomes zero; all the current passes through the diode, meaning the circuit is short-circuited.}

\vspace{4mm}

{V$_1 = \text{I}_\text{D} \cdot \text{R}_1$}

\vspace{4mm}

{I$\displaystyle_\text{D} = \frac{\text{V}_\text{1}}{\text{R}_1} = \frac{5\text{V}}{1\text{k}\Omega} = 5$mA}

\vspace{4mm}

{$\boxed{\text{I}_\text{D} = 5\text{mA} \ \ \ \text{V}_\text{D} = 0 }$}

\vspace{8mm}

\newpage

{\textbf{1.3} For the circuit given in Fig. 2, D$_1$ is a diode, V$_1$=5V, R$_1$=1k$\Omega$ ,and R$_{\text{pot}}$=1k$\Omega$. Find the voltage V$_{\text{D}}$ and current I$_{\text{D}}$.}

\begin{figure}[H]
    \centering
    \includegraphics[width=0.5\linewidth]{msedge_9dd0mQR9Cb.png}
\end{figure}

\vspace{4mm}

{\textbf{Answer}: We assume that the diode has ideal characteristics. Therefore, there will be no current flow through the diode; all the current passes through the resistor, meaning the circuit is open-circuited. Let I be the current through R$_1$.}

\vspace{4mm}

{V$_1 = \text{I} \cdot (\text{R}_1 + \text{R}_\text{pot})$}

\vspace{4mm}

{$\text{I}\displaystyle = \frac{\text{V}_\text{1}}{\text{R}_1 + \text{R}_\text{pot}} = \frac{5\text{V}}{1\text{k}\Omega + 1\text{k}\Omega} = 2.5$mA}

\vspace{4mm}

{V$\displaystyle_\text{D} = 5\text{V}\cdot\frac{1}{1+1} = 2.5$V}

\vspace{4mm}

{$\boxed{\text{I}_\text{D} = 0 \ \ \ \text{V}_\text{D} = 2.5\text{V}}$}

\end{document}