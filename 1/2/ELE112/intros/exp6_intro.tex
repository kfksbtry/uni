% Student Number: 2240357068
% Student Name: Baturay KAFKAS
% EEE @ Hacettepe University

% Last update: 01:32 05/18/25

% My resource collection: https://github.com/kfksbtry/uni

\documentclass{article}

\usepackage{comment}
\usepackage{graphicx}
\usepackage[top=25mm, bottom=25mm, left=25mm, right=25mm]{geometry}
\usepackage{amsmath}
\usepackage{moresize}
\usepackage{float}
\usepackage{fancyhdr}
\usepackage{booktabs}

\pagestyle{fancy}
\fancyhf{}
\fancyhead[R]{Baturay KAFKAS 2240357068 Electrical \& Electronics Engineering}

\rfoot{\thepage}
\renewcommand{\headrulewidth}{0pt} 
\renewcommand{\footrulewidth}{0pt}

\setcounter{page}{47}

\begin{document}

\large

{\small \textbf{HACETTEPE UNIVERSITY}}

{\small \textbf{DEPARTMENT OF ELECTRICAL \& ELECTRONICS ENGINEERING}} 

{\small \textbf{ELE 112 INTRODUCTION TO ELECTRICAL ENGINEERING LABORATORY}} 

\vspace{4mm}

{\small \textbf{EXPERIMENT \#6}}

\vspace{4mm}

{\small \textbf{EXAMINING BASIC CHARACTERISTICS OF A TRANSISTOR IN DC CIRCUITS}}

\vspace{4mm}

{\textbf{Objective:}}
{To examine DC analysis of bipolar junction transistors (BJT).}

\vspace{4mm}

{\textbf{Theory:} If we join together two individual diodes back-to-back, this will give us two PN-junctions connected together in series that share a common P or N terminal. The fusion of these two diodes produces a three layer, two junction, three terminal device forming the basis of a Bipolar Junction Transistor, or BJT for short.}

\begin{figure}[H]
    \centering
    \includegraphics[width=0.35\linewidth]{msedge_27kfW02rhX.png}
    \includegraphics[width=0.35\linewidth]{msedge_5CRo0bHtQW.png}
\end{figure}
\begin{figure}[H]
    \centering
    \includegraphics[width=0.25\linewidth]{msedge_aoyQrMuHYu.png}
    \includegraphics[width=0.25\linewidth]{msedge_Y5Ut5MLC57.png}
\end{figure}

{The BJT physical construction block diagrams are shown on Figure 1-a). The three terminals of the BJT are called Base (B), Collector (C) and Emitter (E). The BJT is fabricated with three separately doped regions. The npn device has one p region between two n regions and the pnp device has one n region between two p regions. The BJT has two junctions (boundaries between the n and the p regions). These junctions are similar to the junctions we saw in the diodes and thus they may be forward biased or reverse biased. By relating these junctions to a diode model the pnp and npn BJTs may be modeled as shown on Figure 1-b). As a result of different possible states of each junction (forward or reverse bias) the BJT will have different modes of operation which are given below:}

\begin{center}
    \large
    \begin{tabular}{ |c| c |c| } 
    \hline
        \textbf{Mode of Operation} & \textbf{Emitter-Base Junction} & \textbf{Collector Base Junction} \\
        \hline
        \textbf{Cut-Off}& Reverse & Reverse \\
        \hline
        \textbf{Active}& Forward & Reverse \\
        \hline
        \textbf{Saturation}& Forward & Forward \\
        \hline
    \end{tabular}
\end{center}

{The transistor’s ability to change between these states enables it to have two basic 
functions: “switching” (digital electronics) or “amplification” (analogue electronics). 
Then bipolar transistors have the ability to operate within three different regions:}

\vspace{4mm}

{Active Region – the transistor operates as an amplifier and I$_\text{C}$ = $\beta$ ×I$_\text{B}$.}

{Saturation – the transistor is operating as a switch and V$_\text{CE}$ = V$_\text{CE, SAT}$.}

{Cut-off – the transistor is “Fully-OFF” operating as a switch and I$_\text{B}$ = I$_\text{C}$ = 0.}

\vspace{4mm}

{For a detailed background please refer to the \textbf{ELE110 Introduction to Electrical 
Engineering} course notes.}

\end{document}