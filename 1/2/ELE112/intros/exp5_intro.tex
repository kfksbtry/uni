% Student Number: 2240357068
% Student Name: Baturay KAFKAS
% EEE @ Hacettepe University

% Last update: HH:MM DD/MM/YY

% My resource collection: https://github.com/kfksbtry/uni
% Circuits were set up in LTspice.

\documentclass{article}

\usepackage{comment}
\usepackage{graphicx}
\usepackage[top=25mm, bottom=25mm, left=25mm, right=25mm]{geometry}
\usepackage{amsmath}
\usepackage{moresize}
\usepackage{float}
\usepackage{fancyhdr}
\usepackage{booktabs}

\pagestyle{fancy}
\fancyhf{}
\fancyhead[R]{Baturay KAFKAS 2240357068 Electrical \& Electronics Engineering}

\rfoot{\thepage}
\renewcommand{\headrulewidth}{0pt} 
\renewcommand{\footrulewidth}{0pt}

\setcounter{page}{44}

\begin{document}

\large

{\small \textbf{HACETTEPE UNIVERSITY}}

{\small \textbf{DEPARTMENT OF ELECTRICAL \& ELECTRONICS ENGINEERING}} 

{\small \textbf{ELE 112 INTRODUCTION TO ELECTRICAL ENGINEERING LABORATORY}} 

\vspace{4mm}

{\small \textbf{EXPERIMENT \#5}}

\vspace{4mm}

{\small \textbf{EXAMINING BASIC CHARACTERISTICS OF A DIODE IN DC CIRCUITS}}

\vspace{4mm}

{\textbf{Objective:}}
{To perform diode testing, and examine the forward bias and reverse bias operations of 
diodes.}

\vspace{4mm}

{\textbf{Theory:}}
{The details are given in the ELE110 Introduction to Electrical Engineering course 
notes.}

\vspace{4mm}

{\Large \textbf{2. EXPERIMENTAL WORK}}

\vspace{4mm}

{\textbf{2.1} Perform diode testing procedure and determine if the given diodes are functioning properly or not.}

\vspace{4mm}

{\textbf{2.2} Set up the circuit in \textit{Fig. 1}. Adjust the potentiometer till you get 0V, 0.3V, 0.5V, 0.7V, 1V, 1.5V, 2V and 2.5V voltage values between point a and point b, respectively. Measure the value of the voltage V$_\text{D}$, and the current I$_\text{D}$. After measurement of V$_\text{D}$ and I$_\text{D}$ values, plot I$_\text{D}$-V$_\text{D}$ curve of the diode. And try to determine the forward bias voltage of the diode.}

\vspace{4mm}

{\textbf{2.3} Set up the circuit in \textit{Fig. 2}. Adjust the potentiometer till you get 0V, 0.3V, 0.5V, 0.7V, 1V, 1.5V, 2V and 2.5V voltage values between point a and point b, respectively, measure the value of the voltage V$_\text{D}$, and the current I$_\text{D}$.}

\vspace{4mm}

{\textbf{2.4} Replace the basic diode (1N4007) with the LED and repeat parts 2.2 and 2.3.}

\vspace{4mm}

{\Large \textbf{3. RESULTS AND CONCLUSION}}

\vspace{4mm}

{\textbf{3.1} Using the results in 2.1 and 2.2, find the internal resistance of the diode for each input voltage value. Does the resistance change at each voltage level?}

{\textbf{3.2} Using the results in 2.3, can you consider the circuit in Fig. 2 as open-circuit?}

{\textbf{3.3} Repeat parts 3.1 and 3.2 for the results found in 2.4.}

{\textbf{3.4} Compare the theoretical and practical results, and comment on them briefly.}

\vspace{4mm}

{\Large \textbf{EQUIPMENTS AND COMPONENTS}}

\vspace{4mm}

{DC battery}

{AVO meter}

{Resistors: 1k$\Omega$ (\#1), 1k$\Omega$ pot}

{Diodes: 1N4007 (\#1), LED (\#1)}

\end{document}