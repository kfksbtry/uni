% Student Number: 2240357068
% Student Name: Baturay KAFKAS
% EEE @ Hacettepe University

% Last update: HH:MM DD/MM/YY

% My resource collection: https://github.com/ProBaturay/uni
% Circuits were set up in LTspice.

\documentclass{article}

\usepackage{comment}
\usepackage{graphicx}
\usepackage[top=25mm, bottom=25mm, left=25mm, right=25mm]{geometry}
\usepackage{amsmath}
\usepackage{moresize}
\usepackage{float}
\usepackage{fancyhdr}
\usepackage{booktabs}

\pagestyle{fancy}
\fancyhf{} 

\rfoot{\thepage}
\renewcommand{\headrulewidth}{0pt} 
\renewcommand{\footrulewidth}{0pt}



%%\setcounter{page}{[NUM]} Enter the next integer after the last page number used.



\begin{document}

\large

{\textit{This part of the experiment is prepared with Online LaTeX Editor Overleaf, and the circuits are drawn in LTspice. Visit the website for the code here:}}

{\textbf{https://www.overleaf.com/read/gqxmprgkvgfd\#1879d9}}
\vspace{4mm}
\hrule
\vspace{4mm}
{\Large \textbf{1. PRELIMINARY WORK}}

\vspace{4mm}

{\textbf{1.1} For the circuit given in Figure 1, find the dissipated and generated powers.}

\vspace{4mm}

\begin{figure}[H]
    \centering
    \includegraphics[width=0.5\linewidth]{msedge_loW7HJp5v7.png}
\end{figure}

{\textbf{Answer}: TBD.}

\vspace{8mm}

{\textbf{1.2} Show that the total power generated by the sources is equal to the total power 
dissipated by the resistors.}

\vspace{4mm}

{\textbf{Answer}: TBD.}

\vspace{8mm}

%% Example table:

%\begin{center}
 %   \large
%    \begin{tabular}{ |c| c c| } 
 %   \hline
  %        & 1 & 2\\
  %      \hline
  %      a)& 3 & 4 \\
  %      \hline
 %       b)& 5 & 6 \\
 %       \hline
 %   \end{tabular}
%\end{center} %% Rearrange the table.

\end{document}
