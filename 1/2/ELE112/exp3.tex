% Student Number: 2240357068
% Student Name: Baturay KAFKAS
% EEE @ Hacettepe University

% Last update: 21:48 22/03/25

% My resource collection: https://github.com/ProBaturay/uni
% Circuits were set up in LTspice.

\documentclass{article}

\usepackage{comment}
\usepackage{graphicx}
\usepackage[top=25mm, bottom=25mm, left=25mm, right=25mm]{geometry}
\usepackage{amsmath}
\usepackage{moresize}
\usepackage{float}
\usepackage{fancyhdr}
\usepackage{booktabs}

\pagestyle{fancy}
\fancyhf{}
\fancyhead[R]{Baturay KAFKAS 2240357068 Electrical \& Electronics Engineering}

\rfoot{\thepage}
\renewcommand{\headrulewidth}{0pt} 
\renewcommand{\footrulewidth}{0pt}

\setcounter{page}{25}

\begin{document}

\large

{\textit{This part of the experiment is prepared with Online LaTeX Editor Overleaf. Visit the website for the code here:}}

{\textbf{https://www.overleaf.com/read/gqxmprgkvgfd\#1879d9}}
\vspace{4mm}
\hrule
\vspace{4mm}
{\Large \textbf{1. PRELIMINARY WORK}}

\vspace{4mm}

{\textbf{1.1} For the circuit given in Figure 1, find the dissipated and generated powers.}

\vspace{4mm}

\begin{figure}[H]
    \centering
    \includegraphics[width=0.5\linewidth]{msedge_loW7HJp5v7.png}
\end{figure}

\vspace{4mm}

{\textbf{Answer}: First, name some points and resistors for ease. After that, let's use the node-voltage method.}

\vspace{4mm}

\begin{figure}[H]
    \centering
    \includegraphics[width=0.5\linewidth]{q0u91995IC.png}
\end{figure}

\vspace{4mm}

{For node a:}

\vspace{4mm}

{$\displaystyle 10\text{mA}-\frac{\text{V}_{\text{a}}}{1\text{k}\Omega}-\frac{\text{V}_{\text{a}}}{1\text{k}\Omega} + \frac{\text{V}_{\text{c}}-\text{V}_{\text{a}}}{1\text{k}\Omega} = 0 \ (1)$}

\vspace{8mm}

{We also know that $\text{V}_\text{c} = 12\text{V} \ (2).$}

\vspace{4mm}

{Compare (1) and (2), we get $\displaystyle \text{V}_\text{a} = \frac{22}{3}\text{V}$.}

\vspace{4mm}

{Now, we can find the generated and dissipated power.}

\vspace{4mm}

{$\displaystyle \boxed{\text{P}_{\text{R}_3} = \frac{\Big(\displaystyle12-\frac{22}{3}\Big)^2 \text{V}^2}{1 \text{k} \Omega} = \frac{196}{9}\text{mW}}$}

\vspace{4mm}

{$\displaystyle \boxed{\text{P}_{\text{R}_1} = \frac{\Big(\displaystyle\frac{22}{3}\Big)^2 \text{V}^2}{1 \text{k} \Omega} = \frac{484}{9}\text{mW}}$}

\vspace{4mm}

{$\displaystyle \boxed{\text{P}_{\text{R}_2} = \frac{\Big(\displaystyle\frac{22}{3}\Big)^2 \text{V}^2}{1 \text{k} \Omega} = \frac{484}{9}\text{mW}}$}

\vspace{4mm}

{$\displaystyle \boxed{\text{P}_{10\text{mA}} = -\Big(\frac{22}{3}\text{V}\Big)(10\text{mA}) = -\frac{220}{3}\text{mW}}$}

\vspace{8mm}

{Find the power generated by the voltage source.}

\vspace{4mm}

{$\displaystyle \text{V}_{\text{c}} - \text{V}_\text{a} = \text{V}_{\text{R}_3} \ \rightarrow \ \text{V}_{\text{R}_3} = \frac{14}{3}\text{V}$}

\vspace{4mm}

{$\displaystyle \text{I}_{\text{R}_3} = \frac{\text{V}_{\text{R}_3}}{\text{R}_3} = \frac{14}{3}\text{mA}$}

\vspace{4mm}

{$\displaystyle \boxed{\text{P}_{12\text{V}} = -(12\text{V})\Big(\frac{14}{3}\text{mA}\Big) = -56\text{mW}}$}

\vspace{8mm}

{\textbf{1.2} Show that the total power generated by the sources is equal to the total power 
dissipated by the resistors.}

\vspace{4mm}

{\textbf{Answer}:}

\vspace{4mm}

{$\displaystyle \text{P}_\text{generated} = \Big({-\frac{220}{3}}- 56\Big)\text{mW} = -\frac{388}{9}\text{mW}$}

\vspace{4mm}

{$\displaystyle \text{P}_\text{dissipated} = \Big({\frac{196}{9}} + \frac{484}{9}+ \frac{484}{9}\Big)\text{mW} = \frac{388}{9}\text{mW}$}

\vspace{4mm}

{$\displaystyle \boxed{|\text{P}_\text{dissipated}| = |\text{P}_\text{generated}| = \frac{388}{9}\text{mW}}$}

\vspace{8mm}

\end{document}
