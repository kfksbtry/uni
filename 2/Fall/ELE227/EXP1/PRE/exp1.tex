%% Comply with the comment where two consecutive percent symbols are written and then delete that comment.
%%
% Student Number: 2240357068
% Student Name: Baturay KAFKAS
% EEE @ Hacettepe University

% Last update: HH:MM DD/MM/YY

% My resource collection: https://github.com/kfksbtry/uni
% Circuits were set up in LTspice.

\documentclass{article}

\usepackage{comment}
\usepackage{graphicx}
\usepackage[top=25mm, bottom=25mm, left=25mm, right=25mm]{geometry}
\usepackage{amsmath}
\usepackage{moresize}
\usepackage{float}
\usepackage{fancyhdr}
\usepackage{booktabs}
\usepackage[T1]{fontenc}
\usepackage{upquote}

\usepackage{lipsum} %% Delete this line later.

\pagestyle{fancy}
\fancyhf{}
\fancyhead[R]{Baturay KAFKAS 2240357068 Electrical \& Electronics Engineering}

\rfoot{\thepage}
\renewcommand{\headrulewidth}{0pt} 
\renewcommand{\footrulewidth}{0pt}



%%\setcounter{page}{[NUM]} Enter the next integer after the last page number used.



\begin{document}

\large %% Default font size

%%------------------------------------%%
{\Large \textbf{EXPERIMENT 1 PRELIMINARY WORK}} %%

\hfill

\noindent \textbf{Q1:} Read the remainders in the backward direction and add the necessary zeros to the left.
\[
\begin{array}{c||c}
    {\textbf{a}.\quad\begin{array}{l|r}
    41& \\20&1\\10&0\\5&0\\2&1\\1&0\\0&1
    \end{array}\implies\boxed{(41)_{10}=(00101001)_2}}\quad&\quad{ \textbf{b}.\quad\begin{array}{l|r}
    79& \\39&1\\19&1\\9&1\\4&1\\2&0\\1&0\\0&1
    \end{array}\implies\boxed{(79)_{10}=(01001111)_2}}
\end{array}\]

\[
\begin{array}{c||c}
    {\textbf{c}.\quad\begin{array}{l|r}
    108& \\54&0\\27&0\\13&1\\6&1\\3&0\\1&1\\0&1
    \end{array}\implies\boxed{(108)_{10}=(01101100)_2}}\:&\:     {\textbf{d}.\quad\begin{array}{l|r}
    127& \\63&1\\31&1\\15&1\\7&1\\3&1\\1&1\\0&1
    \end{array}\implies\boxed{(127)_{10}=(01111111)_2}}
\end{array}\]

\[
\begin{array}{c}
    {\textbf{e}.\quad\begin{array}{l|r}
    240& \\120&0\\60&0\\30&0\\15&0\\7&1\\3&1\\1&1\\0&1
    \end{array}\implies\boxed{(240)_{10}=(11110000)_2}}
\end{array}\]

\hfill

\noindent \textbf{Q2:}

\begin{center}
    \large
    \begin{tabular}{|c|c|c|} 
    \hline
        & Decimal form & Hexadecimal form\\
        \hline
        \textbf{a.} 01001010 & $2^6+2^3+2^1=\boxed{74}$ & $\underbrace{0100}_{4}\,\underbrace{1010}_{\text{A}}=\boxed{4\text{A}}$ \\
        \hline
        \textbf{b.} 01110101 & $2^6+2^5+2^4+2^2+2^0=\boxed{117}$  & $\underbrace{0111}_{7}\,\underbrace{0101}_{5}=\boxed{75}$ \\
        \hline
        \textbf{c.} 11110010 & $2^7+2^6+2^5+2^4+2^1=\boxed{242}$ & $\underbrace{1111}_{\text{F}}\,\underbrace{0010}_{2}=\boxed{\text{F}2}$ \\
        \hline
        \textbf{d.} 00101110 & $2^5+2^3+2^2+2^1=\boxed{46}$  & $\underbrace{0010}_{2}\,\underbrace{1110}_{\text{E}}=\boxed{2\text{E}}$  \\
        \hline
        \textbf{e.} 11010000 & $2^7+2^6+2^4=\boxed{208}$ & $\underbrace{1101}_{\text{D}}\,\underbrace{0000}_{0}=\boxed{\text{D}0}$ \\
        \hline
    \end{tabular}
\end{center}

%% Example figure:

%%\begin{figure}[H]
%%    \centering
%%    \includegraphics[width=NUM\linewidth]{Draft1.png} Enter the width you'd like.
%%\end{figure}

\newpage

\noindent \textbf{Q3: a.} \textit{Hand-drawn circuit and RTL schematic}:

\begin{figure}[H]
    \centering
    \includegraphics[width=0.4\linewidth]{WhatsApp Image 2025-11-01 at 23.18.23.jpeg}
\end{figure}

\begin{figure}[H]
    \centering
    \includegraphics[width=0.6\linewidth]{Resim1_grayscale.png}
\end{figure}

\newpage

\noindent \textit{VHDL}:

\vspace{1mm}
\hrule

\begin{verbatim}
library IEEE;
use IEEE.STD_LOGIC_1164.ALL;

entity EXP1_PRE_Q3 is
Port ( A, B, C : in  STD_LOGIC;
       F       : out STD_LOGIC);
end EXP1_PRE_Q3;

architecture Behavioral of EXP1_PRE_Q3 is
    signal m2, m3, m4, m7 : STD_LOGIC;

begin
    m2 <= not A and B and not C;
    m3 <= not A and B and C;
    m4 <= A and not B and not C;
    m7 <= A and B and C;
    
    F <= m2 or m3 or m4 or m7;
end Behavioral;
\end{verbatim}

\hrule

\hfill

\hfill

\noindent \textbf{b.} \textit{Test bench}

\vspace{1mm}
\hrule

\begin{verbatim}
stim_proc: process
begin		
    wait for 100 ns;	
    A <= '0'; B <= '1'; C <= '0';
    wait for 100 ns;	
    A <= '1'; B <= '1'; C <= '0';
    wait for 100 ns;	
    A <= '1'; B <= '0'; C <= '0';
    wait for 100 ns;	
    A <= '1'; B <= '0'; C <= '1';
    wait;
end process;
\end{verbatim}

\hrule

\hfill

\begin{figure}[H]
    \centering
    \includegraphics[width=1\linewidth]{Resim2_grayscale.png}
\end{figure}

\newpage

\noindent \textbf{Q4: a.}
\[
\begin{array}{|ccc|ccc|}
\hline x&y&z&A&B&C\\\hline
0&0&0&0&0&1\\
0&0&1&0&1&0\\
0&1&0&0&1&1\\
0&1&1&1&0&0\\
1&0&0&0&1&1\\
1&0&1&1&0&0\\
1&1&0&1&0&1\\
1&1&1&1&1&0
\\\hline
\end{array}\implies\text{Collect the minterms}\implies\begin{array}{l}
A=x'yz+xy'z+xyz'+xyz\\
B=x'y'z+x'yz'+xy'z'+xyz\\
C=x'y'z'+x'yz'+xy'z'+xyz'\\
\end{array}\]

\hfill

\noindent \textbf{b.}

\[A:\:\begin{array}{|c|c|c|c|c|}
\hline _x\backslash^{yz}&00&01&11&10\\
\hline 0&0&0&1&0\\
\hline 1&0&1&1&1\\\hline
\end{array}\,,\quad
B:\:\begin{array}{|c|c|c|c|c|}
\hline _x\backslash^{yz}&00&01&11&10\\
\hline 0&0&1&0&1\\
\hline 1&1&0&1&0\\\hline
\end{array}\,,\quad
C:\:\begin{array}{|c|c|c|c|c|}
\hline _x\backslash^{yz}&00&01&11&10\\
\hline 0&1&0&0&1\\
\hline 1&1&0&0&1\\\hline
\end{array}\]

\[A=xz+yz+xy,\qquad\qquad\qquad B=x \oplus y \oplus z,\qquad\qquad\qquad\qquad\quad C=z'\]

\hfill

\noindent \textbf{c.} \textit{Hand-drawn circuit and RTL schematic}:

\begin{figure}[H]
    \centering
    \includegraphics[width=0.55\linewidth]{WhatsApp Image 2025-11-02 at 02.47.38.jpeg}
\end{figure}

\newpage

\begin{figure}[H]
    \centering
    \includegraphics[width=0.9\linewidth]{Resim3_grayscale.png}
\end{figure}

\noindent \textit{VHDL}:

\vspace{1mm}
\hrule

\begin{verbatim}
library IEEE;
use IEEE.STD_LOGIC_1164.ALL;

entity EXP1_PRE_Q4 is
    Port ( x, y, z : in  STD_LOGIC;
           A, B, C : out STD_LOGIC);
end EXP1_PRE_Q4;

architecture Behavioral of EXP1_PRE_Q4 is

begin
    A <= (x and y) or (x and z) or (y and z);
    B <= x xor y xor z;
    C <= not z;
end Behavioral;
\end{verbatim}

\hrule

\newpage

\noindent \textbf{d.} \textit{Test bench}

\vspace{1mm}
\hrule

\hfill

\begin{verbatim}
stim_proc: process
begin		
    wait for 100 ns;	
    x <= '0'; y <= '1'; z <= '0';
    wait for 100 ns;	
    x <= '1'; y <= '1'; z <= '0';
    wait for 100 ns;	
    x <= '1'; y <= '0'; z <= '0';
    wait for 100 ns;	
    x <= '1'; y <= '0'; z <= '1';
    wait;
end process;
\end{verbatim}

\hrule

\begin{figure}[H]
    \centering
    \includegraphics[width=1\linewidth]{Resim4_grayscale.png}
\end{figure}

\end{document}