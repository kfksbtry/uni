% Student Number: 2240357068
% Student Name: Baturay KAFKAS
% EEE @ Hacettepe University

% Last update: HH:MM DD/MM/YY

% My resource collection: https://github.com/kfksbtry/uni
% Circuits were set up in LTspice.

\documentclass{article}

\usepackage{comment}
\usepackage{graphicx}
\usepackage[top=25mm, bottom=25mm, left=25mm, right=25mm]{geometry}
\usepackage{amsmath}
\usepackage{moresize}
\usepackage{float}
\usepackage{fancyhdr}
\usepackage{booktabs}
\usepackage[T1]{fontenc}
\usepackage{upquote}
\usepackage{parskip}
\usepackage{xcolor}
\usepackage{adjustbox}
\usepackage{tikz}
\usepackage{eso-pic}
\usepackage{textcomp}
\usetikzlibrary{matrix,calc, positioning, shapes, arrows}

\pagestyle{fancy}
\fancyhf{}
\fancyhead[R]{Baturay KAFKAS 2240357068 Electrical \& Electronics Engineering}

\rfoot{\thepage}
\renewcommand{\headrulewidth}{0pt} 
\renewcommand{\footrulewidth}{0pt}

\definecolor{imp1}{RGB}{200, 200, 200}
\definecolor{imp2}{RGB}{170, 170, 170}
\definecolor{imp3}{RGB}{140, 140, 140}
\definecolor{imp4}{RGB}{110, 110, 110}
\definecolor{imp5}{RGB}{80, 80, 80}

%%% K-MAP

%isolated term
%#1 - Optional. Space between node and grouping line. Default=0
%#2 - node
%#3 - filling color
\newcommand{\implicantsol}[3][0]{
    \draw[rounded corners=3pt, fill=#3, opacity=0.3] ($(#2.north west)+(135:#1)$) rectangle ($(#2.south east)+(-45:#1)$);
    }


%internal group
%#1 - Optional. Space between node and grouping line. Default=0
%#2 - top left node
%#3 - bottom right node
%#4 - filling color
\newcommand{\implicant}[4][0]{
    \draw[rounded corners=3pt, fill=#4, opacity=0.3] ($(#2.north west)+(135:#1)$) rectangle ($(#3.south east)+(-45:#1)$);
    }

%group lateral borders
%#1 - Optional. Space between node and grouping line. Default=0
%#2 - top left node
%#3 - bottom right node
%#4 - filling color
\newcommand{\implicantcostats}[4][0]{
    \draw[rounded corners=3pt, fill=#4, opacity=0.3] ($(rf.east |- #2.north)+(90:#1)$)-| ($(#2.east)+(0:#1)$) |- ($(rf.east |- #3.south)+(-90:#1)$);
    \draw[rounded corners=3pt, fill=#4, opacity=0.3] ($(cf.west |- #2.north)+(90:#1)$) -| ($(#3.west)+(180:#1)$) |- ($(cf.west |- #3.south)+(-90:#1)$);
}

%group top-bottom borders
%#1 - Optional. Space between node and grouping line. Default=0
%#2 - top left node
%#3 - bottom right node
%#4 - filling color
\newcommand{\implicantdaltbaix}[4][0]{
    \draw[rounded corners=3pt, fill=#4, opacity=0.3] ($(cf.south -| #2.west)+(180:#1)$) |- ($(#2.south)+(-90:#1)$) -| ($(cf.south -| #3.east)+(0:#1)$);
    \draw[rounded corners=3pt, fill=#4, opacity=0.3] ($(rf.north -| #2.west)+(180:#1)$) |- ($(#3.north)+(90:#1)$) -| ($(rf.north -| #3.east)+(0:#1)$);
}

%group corners
%#1 - Optional. Space between node and grouping line. Default=0
%#2 - filling color
\newcommand{\implicantcantons}[2][0]{
    \draw[rounded corners=3pt, opacity=.3] ($(rf.east |- 0.south)+(-90:#1)$) -| ($(0.east |- cf.south)+(0:#1)$);
    \draw[rounded corners=3pt, opacity=.3] ($(rf.east |- 8.north)+(90:#1)$) -| ($(8.east |- rf.north)+(0:#1)$);
    \draw[rounded corners=3pt, opacity=.3] ($(cf.west |- 2.south)+(-90:#1)$) -| ($(2.west |- cf.south)+(180:#1)$);
    \draw[rounded corners=3pt, opacity=.3] ($(cf.west |- 10.north)+(90:#1)$) -| ($(10.west |- rf.north)+(180:#1)$);
    \fill[rounded corners=3pt, fill=#2, opacity=.3] ($(rf.east |- 0.south)+(-90:#1)$) -|  ($(0.east |- cf.south)+(0:#1)$) [sharp corners] ($(rf.east |- 0.south)+(-90:#1)$) |-  ($(0.east |- cf.south)+(0:#1)$) ;
    \fill[rounded corners=3pt, fill=#2, opacity=.3] ($(rf.east |- 8.north)+(90:#1)$) -| ($(8.east |- rf.north)+(0:#1)$) [sharp corners] ($(rf.east |- 8.north)+(90:#1)$) |- ($(8.east |- rf.north)+(0:#1)$) ;
    \fill[rounded corners=3pt, fill=#2, opacity=.3] ($(cf.west |- 2.south)+(-90:#1)$) -| ($(2.west |- cf.south)+(180:#1)$) [sharp corners]($(cf.west |- 2.south)+(-90:#1)$) |- ($(2.west |- cf.south)+(180:#1)$) ;
    \fill[rounded corners=3pt, fill=#2, opacity=.3] ($(cf.west |- 10.north)+(90:#1)$) -| ($(10.west |- rf.north)+(180:#1)$) [sharp corners] ($(cf.west |- 10.north)+(90:#1)$) |- ($(10.west |- rf.north)+(180:#1)$) ;
}

%Empty Karnaugh map 4x4
\newenvironment{Karnaugh1}%
{
\begin{tikzpicture}[baseline=(current bounding box.north),scale=0.8]
\draw (0,0) grid (4,4);
\draw (0,4) -- node [pos=0.9,above right,anchor=south west] {\small A(1)A(0)} node [pos=0.9,below left,anchor=north east] {\small A(3)A(2)} ++(135:1);
%
\matrix (mapa) [matrix of nodes,
        column sep={0.8cm,between origins},
        row sep={0.8cm,between origins},
        every node/.style={minimum size=0.3mm},
        anchor=8.center,
        ampersand replacement=\&] at (0.5,0.5)
{
                       \& |(c00)| 00         \& |(c01)| 01         \& |(c11)| 11         \& |(c10)| 10         \& |(cf)| \phantom{00} \\
|(r00)| 00             \& |(0)|  \phantom{0} \& |(1)|  \phantom{0} \& |(3)|  \phantom{0} \& |(2)|  \phantom{0} \&                     \\
|(r01)| 01             \& |(4)|  \phantom{0} \& |(5)|  \phantom{0} \& |(7)|  \phantom{0} \& |(6)|  \phantom{0} \&                     \\
|(r11)| 11             \& |(12)| \phantom{0} \& |(13)| \phantom{0} \& |(15)| \phantom{0} \& |(14)| \phantom{0} \&                     \\
|(r10)| 10             \& |(8)|  \phantom{0} \& |(9)|  \phantom{0} \& |(11)| \phantom{0} \& |(10)| \phantom{0} \&                     \\
|(rf) | \phantom{00}   \&                    \&                    \&                    \&                    \&                     \\
};
}%
{
\end{tikzpicture}
}

\newenvironment{Karnaugh2}%
{
\begin{tikzpicture}[baseline=(current bounding box.north),scale=0.8]
\draw (0,0) grid (4,4);
\draw (0,4) -- node [pos=0.9,above right,anchor=south west] {\small CD} node [pos=0.9,below left,anchor=north east] {\small AB} ++(135:1);
%
\matrix (mapa) [matrix of nodes,
        column sep={0.8cm,between origins},
        row sep={0.8cm,between origins},
        every node/.style={minimum size=0.3mm},
        anchor=8.center,
        ampersand replacement=\&] at (0.5,0.5)
{
                       \& |(c00)| 00         \& |(c01)| 01         \& |(c11)| 11         \& |(c10)| 10         \& |(cf)| \phantom{00} \\
|(r00)| 00             \& |(0)|  \phantom{0} \& |(1)|  \phantom{0} \& |(3)|  \phantom{0} \& |(2)|  \phantom{0} \&                     \\
|(r01)| 01             \& |(4)|  \phantom{0} \& |(5)|  \phantom{0} \& |(7)|  \phantom{0} \& |(6)|  \phantom{0} \&                     \\
|(r11)| 11             \& |(12)| \phantom{0} \& |(13)| \phantom{0} \& |(15)| \phantom{0} \& |(14)| \phantom{0} \&                     \\
|(r10)| 10             \& |(8)|  \phantom{0} \& |(9)|  \phantom{0} \& |(11)| \phantom{0} \& |(10)| \phantom{0} \&                     \\
|(rf) | \phantom{00}   \&                    \&                    \&                    \&                    \&                     \\
};
}%
{
\end{tikzpicture}
}

%Empty Karnaugh map 2x4
\newenvironment{Karnaughvuit}%
{
\begin{tikzpicture}[baseline=(current bounding box.north),scale=0.8]
\draw (0,0) grid (4,2);
\draw (0,2) -- node [pos=0.85,above right,anchor=south west] {\small JK} node [pos=0.85,below left,anchor=north east] {\small Q} ++(135:1);
%
\matrix (mapa) [matrix of nodes,
        column sep={0.8cm,between origins},
        row sep={0.8cm,between origins},
        every node/.style={minimum size=0.3mm},
        anchor=4.center,
        ampersand replacement=\&] at (0.5,0.5)
{
                      \& |(c00)| 00         \& |(c01)| 01         \& |(c11)| 11         \& |(c10)| 10         \& |(cf)| \phantom{00} \\
|(r00)| 0             \& |(0)|  \phantom{0} \& |(1)|  \phantom{0} \& |(3)|  \phantom{0} \& |(2)|  \phantom{0} \&                     \\
|(r01)| 1             \& |(4)|  \phantom{0} \& |(5)|  \phantom{0} \& |(7)|  \phantom{0} \& |(6)|  \phantom{0} \&                     \\
|(rf) | \phantom{00}  \&                    \&                    \&                    \&                    \&                     \\
};
}%
{
\end{tikzpicture}
}

\newenvironment{Karnaughvuit2}%
{
\begin{tikzpicture}[baseline=(current bounding box.north),scale=0.8]
\draw (0,0) grid (4,2);
\draw (0,2) -- node [pos=0.85,above right,anchor=south west] {\small D1(t)\,D2(t)} node [pos=0.85,below left,anchor=north east] {\small x} ++(135:1);
%
\matrix (mapa) [matrix of nodes,
        column sep={0.8cm,between origins},
        row sep={0.8cm,between origins},
        every node/.style={minimum size=0.3mm},
        anchor=4.center,
        ampersand replacement=\&] at (0.5,0.5)
{
                      \& |(c00)| 00         \& |(c01)| 01         \& |(c11)| 11         \& |(c10)| 10         \& |(cf)| \phantom{00} \\
|(r00)| 0             \& |(0)|  \phantom{0} \& |(1)|  \phantom{0} \& |(3)|  \phantom{0} \& |(2)|  \phantom{0} \&                     \\
|(r01)| 1             \& |(4)|  \phantom{0} \& |(5)|  \phantom{0} \& |(7)|  \phantom{0} \& |(6)|  \phantom{0} \&                     \\
|(rf) | \phantom{00}  \&                    \&                    \&                    \&                    \&                     \\
};
}%
{
\end{tikzpicture}
}


\newenvironment{Karnaughvuit3}%
{
\begin{tikzpicture}[baseline=(current bounding box.north),scale=0.8]
\draw (0,0) grid (4,2);
\draw (0,2) -- node [pos=0.85,above right,anchor=south west] {\small Data\,parityin} node [pos=0.85,below left,anchor=north east] {\small T(t)} ++(135:1);
%
\matrix (mapa) [matrix of nodes,
        column sep={0.8cm,between origins},
        row sep={0.8cm,between origins},
        every node/.style={minimum size=0.3mm},
        anchor=4.center,
        ampersand replacement=\&] at (0.5,0.5)
{
                      \& |(c00)| 00         \& |(c01)| 01         \& |(c11)| 11         \& |(c10)| 10         \& |(cf)| \phantom{00} \\
|(r00)| 0             \& |(0)|  \phantom{0} \& |(1)|  \phantom{0} \& |(3)|  \phantom{0} \& |(2)|  \phantom{0} \&                     \\
|(r01)| 1             \& |(4)|  \phantom{0} \& |(5)|  \phantom{0} \& |(7)|  \phantom{0} \& |(6)|  \phantom{0} \&                     \\
|(rf) | \phantom{00}  \&                    \&                    \&                    \&                    \&                     \\
};
}%
{
\end{tikzpicture}
}


%Empty Karnaugh map 2x2
\newenvironment{Karnaughquatre}%
{
\begin{tikzpicture}[baseline=(current bounding box.north),scale=0.8]
\draw (0,0) grid (2,2);
\draw (0,2) -- node [pos=0.9,above right,anchor=south west] {\small Data} node [pos=0.9,below left,anchor=north east] {\small T(t)} ++(135:1);
%
\matrix (mapa) [matrix of nodes,
        column sep={0.8cm,between origins},
        row sep={0.8cm,between origins},
        every node/.style={minimum size=0.3mm},
        anchor=2.center,
        ampersand replacement=\&] at (0.5,0.5)
{
          \& |(c00)| 0          \& |(c01)| 1  \\
|(r00)| 0 \& |(0)|  \phantom{0} \& |(1)|  \phantom{0} \\
|(r01)| 1 \& |(2)|  \phantom{0} \& |(3)|  \phantom{0} \\
};
}%
{
\end{tikzpicture}
}

%Defines 8 or 16 values (0,1,X)
\newcommand{\contingut}[1]{%
\foreach \x [count=\xi from 0]  in {#1}
     \path (\xi) node {\x};
}

%Places 1 in listed positions
\newcommand{\minterms}[1]{%
    \foreach \x in {#1}
        \path (\x) node {1};
}

%Places 0 in listed positions
\newcommand{\maxterms}[1]{%
    \foreach \x in {#1}
        \path (\x) node {0};
}

%Places X in listed positions
\newcommand{\indeterminats}[1]{%
    \foreach \x in {#1}
        \path (\x) node {X};
}

%%%

\begin{document}

\large %% Default font size

%%------------------------------------%%
{\Large \textbf{EXPERIMENT 4 PRELIMINARY WORK}} %%

\vspace{1em}

\textbf{Q1:} \textbf{1. Analytical Solution}

\textit{Hand-drawn circuits and RTL schematics}

\begin{figure}[H]
    \centering
    \includegraphics[width=0.6\linewidth]{1_hdc1_g.jpg}
\end{figure}
\begin{figure}[H]
    \centering
    \includegraphics[width=0.5\linewidth]{1_hdc2_g.jpg}
\end{figure}
\begin{figure}[H]
    \centering
    \includegraphics[width=0.75\linewidth]{1_hdc3_g.jpg}
\end{figure}
\begin{figure}[H]
    \centering
    \includegraphics[width=0.6\linewidth]{1_hdc4_g.jpg}
\end{figure}
\newpage
\begin{figure}[H]
    \centering
    \includegraphics[width=0.72\linewidth]{1_a_rtl_i_g.png}
\end{figure}

\vspace{-2em}

\begin{figure}[H]
    \centering
    \includegraphics[width=0.72\linewidth]{1_b_rtl_i_g.png}
\end{figure}

\vspace{-2em}

\begin{figure}[H]
    \centering
    \includegraphics[width=0.72\linewidth]{1_c_rtl_i_g.png}
\end{figure}

\vspace{-2em}

\begin{figure}[H]
    \centering
    \includegraphics[width=0.72\linewidth]{1_d_rtl_i_g.png}
\end{figure}

\textit{CircuitLab}

\begin{center}\textbf{a.} Sr Latch with NAND Gates\end{center}
\begin{figure}[H]
    \centering
    \includegraphics[width=0.75\linewidth]{c1.png}
\end{figure}

\begin{center}Graph of Input S\_sr\end{center}
\begin{figure}[H]
    \centering
    \includegraphics[width=0.7\linewidth]{graph_S_sr_g.png}
\end{figure}

\begin{center}Graph of Input En\_sr\end{center}
\begin{figure}[H]
    \centering
    \includegraphics[width=0.7\linewidth]{graph_En_sr_g.png}
\end{figure}

\newpage

\begin{center}Graph of Input R\_sr\end{center}
\begin{figure}[H]
    \centering
    \includegraphics[width=0.7\linewidth]{graph_R_sr_g.png}
\end{figure}

\begin{center}Graph of Output Q\_sr\end{center}
\begin{figure}[H]
    \centering
    \includegraphics[width=0.7\linewidth]{graph_Q_sr_g.png}
\end{figure}

\begin{center}Graph of Output Qn\_sr\end{center}
\begin{figure}[H]
    \centering
    \includegraphics[width=0.7\linewidth]{graph_Qn_sr_g.png}
\end{figure}

\newpage

\begin{center}\textbf{b.} D Latch\end{center}
\begin{figure}[H]
    \centering
    \includegraphics[width=0.75\linewidth]{c2.png}
\end{figure}

\begin{center}Graph of Input D\_d\end{center}
\begin{figure}[H]
    \centering
    \includegraphics[width=0.7\linewidth]{graph_D_d_g.png}
\end{figure}

\begin{center}Graph of Input En\_d\end{center}
\begin{figure}[H]
    \centering
    \includegraphics[width=0.7\linewidth]{graph_En_d_g.png}
\end{figure}

\newpage

\begin{center}Graph of Output Q\_d\end{center}
\begin{figure}[H]
    \centering
    \includegraphics[width=0.7\linewidth]{graph_Q_d_g.png}
\end{figure}

\begin{center}Graph of Output Qn\_d\end{center}
\begin{figure}[H]
    \centering
    \includegraphics[width=0.7\linewidth]{graph_Qn_d_g.png}
\end{figure}

\begin{center}\textbf{c.} Positive-Edge-Triggered D Flip-Flop\end{center}
\begin{figure}[H]
    \centering
    \includegraphics[width=0.75\linewidth]{c3.png}
\end{figure}

\newpage

\begin{center}Graph of Input D\_dff\end{center}
\begin{figure}[H]
    \centering
    \includegraphics[width=0.7\linewidth]{graph_D_dff_g.png}
\end{figure}

\begin{center}Graph of Input Clk\_dff\end{center}
\begin{figure}[H]
    \centering
    \includegraphics[width=0.7\linewidth]{graph_Clk_dff_g.png}
\end{figure}

\begin{center}Graph of Output Q\_dff\end{center}
\begin{figure}[H]
    \centering
    \includegraphics[width=0.7\linewidth]{graph_Q_dff_g.png}
\end{figure}

\newpage

\begin{center}\textbf{d.} Positive-Edge-Triggered T Flip-Flop\end{center}
\begin{figure}[H]
    \centering
    \includegraphics[width=0.75\linewidth]{c4.png}
\end{figure}

\begin{center}Graph of Input T\_tff\end{center}
\begin{figure}[H]
    \centering
    \includegraphics[width=0.7\linewidth]{graph_T_tff_g.png}
\end{figure}

\begin{center}Graph of Input Clk\_tff\end{center}
\begin{figure}[H]
    \centering
    \includegraphics[width=0.7\linewidth]{graph_Clk_tff_g.png}
\end{figure}

\newpage

\begin{center}Graph of Output Q\_tff\end{center}
\begin{figure}[H]
    \centering
    \includegraphics[width=0.7\linewidth]{graph_Q_tff_g.png}
\end{figure}

\begin{center}Graph of Output Qn\_tff\end{center}
\begin{figure}[H]
    \centering
    \includegraphics[width=0.7\linewidth]{graph_Qn_tff_g.png}
\end{figure}

\textbf{2. Codes}

\textit{VHDL - \textbf{a}. SR Latch with NAND Gates}

\vspace{1mm}
\hrule

\begin{verbatim}
library ieee;
use ieee.std_logic_1164.all;
entity EXP4_PRE_Q1_SRLATCH is
    port ( S_sr, R_sr, En_sr : in  std_logic; Q_sr, Qn_sr : out std_logic);
end EXP4_PRE_Q1_SRLATCH;
architecture Behavioral of EXP4_PRE_Q1_SRLATCH is
signal temp_Q, temp_Qn : STD_LOGIC;
begin
	 temp_Q <= (S_sr nand En_sr) nand temp_Qn;
	 temp_Qn <= (R_sr nand En_sr) nand temp_Q;
	 Q_sr <= temp_Q; Qn_sr <= temp_Qn;
end Behavioral;
\end{verbatim}
\vspace{1mm}
\hrule

\newpage

\textit{VHDL - \textbf{b.} D Latch}

\vspace{1mm}
\hrule

\begin{verbatim}
library IEEE;
use IEEE.STD_LOGIC_1164.ALL;
entity EXP4_PRE_Q1_DLATCH is
    port ( D_d, En_d : in  std_logic; Q_d, Qn_d : out std_logic);
end EXP4_PRE_Q1_DLATCH;
architecture Behavioral of EXP4_PRE_Q1_DLATCH is
component EXP4_PRE_Q1_SRLATCH
    port ( S_sr, R_sr, En_sr : in  std_logic; Q_sr, Qn_sr : out std_logic);
end component;
signal notD_d : std_logic;
begin
    notD_d <= not D_d;
    sr_latch : EXP4_PRE_Q1_SRLATCH port map (D_d, notD_d, En_d, Q_d, Qn_d);
end Behavioral;
\end{verbatim}
\vspace{1mm}
\hrule

\vspace{1em}

\textit{VHDL - \textbf{c}. Positive-Edge-Triggered D Flip-Flop}

\vspace{1mm}
\hrule

\begin{verbatim}
library IEEE;
use IEEE.STD_LOGIC_1164.ALL;
entity EXP4_PRE_Q1_DFLIPFLOP is
    port ( D_dff, Clk_dff : in  std_logic;
           Q_dff, Qn_dff : out std_logic);
end EXP4_PRE_Q1_DFLIPFLOP;
architecture Behavioral of EXP4_PRE_Q1_DFLIPFLOP is
component EXP4_PRE_Q1_DLATCH
    port ( D_d, En_d : in  std_logic; Q_d, Qn_d : out std_logic);
end component;
signal notClk_dff, Q1_dff, Q1n_dff : std_logic;
begin
  notClk_dff <= not Clk_dff;
  latch1 : EXP4_PRE_Q1_DLATCH port map (D_dff, notClk_dff, Q1_dff, Q1n_dff);
  latch2 : EXP4_PRE_Q1_DLATCH port map (Q1_dff, Clk_dff, Q_dff, Qn_dff);
end Behavioral;
\end{verbatim}
\vspace{1mm}
\hrule

\vspace{1em}

\textit{VHDL - \textbf{d.} JK Flip-Flop with Reset State}

\vspace{1mm}
\hrule

\begin{verbatim}
library IEEE;
use IEEE.STD_LOGIC_1164.ALL;
entity EXP4_PRE_Q1_JKFF is
    Port ( J_jkff, K_jkff, Reset_jkff, Clock_jkff : in STD_LOGIC;
           Q_jkff, Qn_jkff : out STD_LOGIC);
end EXP4_PRE_Q1_JKFF;
architecture Behavioral of EXP4_PRE_Q1_JKFF is
signal tempQ: std_logic;
begin
    process (Reset_jkff, Clock_jkff)
    begin
        if Clock_jkff'event and Clock_jkff = '1' then
            if Reset_jkff = '1' then tempQ <= '0';
                elsif (J_jkff='0' and K_jkff='0') then tempQ <= tempQ;
                elsif (J_jkff='0' and K_jkff='1') then tempQ <= '0';
                elsif (J_jkff='1' and K_jkff='0') then tempQ <= '1';
                elsif (J_jkff='1' and K_jkff='1') then tempQ <= not (tempQ);
            end if;
        end if;
    end process;
    Q_jkff <= tempQ; Qn_jkff <= not tempQ;
end Behavioral;
\end{verbatim}
\vspace{1mm}
\hrule


\vspace{1em}

\textit{VHDL - \textbf{d.} Positive-Edge-Triggered T Flip-Flop}

\vspace{1mm}
\hrule

\begin{verbatim}
library IEEE;
use IEEE.STD_LOGIC_1164.ALL;
entity EXP4_PRE_Q1_TFLIPFLOP is
    Port ( T_tff, Reset_tff, Clk_tff : in std_logic;
           Q_tff, Qn_tff : out std_logic);
end EXP4_PRE_Q1_TFLIPFLOP;
architecture Behavioral of EXP4_PRE_Q1_TFLIPFLOP is
component EXP4_PRE_Q1_JKFF
    Port ( J_jkff, K_jkff, Reset_jkff, Clock_jkff : in STD_LOGIC;
           Q_jkff, Qn_jkff : out STD_LOGIC);
end component;
signal notClk_tff : std_logic;
begin
    jkff : EXP4_PRE_Q1_JKFF port map(T_tff, T_tff, Reset_tff,
                                     Clk_tff, Q_tff, Qn_tff);
end Behavioral;
\end{verbatim}
\vspace{1mm}
\hrule

\vspace{1em}

\textbf{3. Results}

\textit{Test bench - \textbf{a.} SR Latch with NAND Gates}

\vspace{1mm}
\hrule

\begin{verbatim}
stim_proc: process
begin		
    En_sr <= '0'; S_sr <= '0'; R_sr <= '0'; wait for 100 ns;
    En_sr <= '0'; S_sr <= '0'; R_sr <= '1'; wait for 100 ns;
    En_sr <= '0'; S_sr <= '1'; R_sr <= '0'; wait for 100 ns;
    En_sr <= '0'; S_sr <= '1'; R_sr <= '1'; wait for 100 ns;
    En_sr <= '1'; S_sr <= '0'; R_sr <= '0'; wait for 100 ns;
    En_sr <= '1'; S_sr <= '0'; R_sr <= '1'; wait for 100 ns;
    En_sr <= '1'; S_sr <= '1'; R_sr <= '0'; wait for 100 ns;
    En_sr <= '1'; S_sr <= '1'; R_sr <= '1'; wait for 100 ns;
end process;
\end{verbatim}

\vspace{1mm}
\hrule

\begin{figure}[H]
    \centering
    \includegraphics[width=1\linewidth]{1_a_graph_i_g.png}
\end{figure}

\textit{Test bench - \textbf{b.} D Latch}

\vspace{1mm}
\hrule

\begin{verbatim}
stim_proc: process
begin		
    En_d <= '0'; D_d <= '0'; wait for 100 ns;
    En_d <= '0'; D_d <= '1'; wait for 100 ns;
    En_d <= '1'; D_d <= '0'; wait for 100 ns;
    En_d <= '1'; D_d <= '1'; wait for 100 ns;
end process;
\end{verbatim}

\vspace{1mm}
\hrule

\begin{figure}[H]
    \centering
    \includegraphics[width=1\linewidth]{1_b_graph_i_g.png}
\end{figure}

\textit{Test bench - \textbf{c.} Positive-Edge-Triggered D Flip-Flop}

\vspace{1mm}
\hrule

\begin{verbatim}
stim_proc: process
begin		
    D_dff <= '0'; wait for Clk_dff_period*2;
    D_dff <= '1'; wait for Clk_dff_period*2;
    D_dff <= '0'; wait for Clk_dff_period*2;
    D_dff <= '1'; wait for Clk_dff_period*2;
end process;
\end{verbatim}

\vspace{1mm}
\hrule

\begin{figure}[H]
    \centering
    \includegraphics[width=0.65\linewidth]{1_c_graph_i_g.png}
\end{figure}

\textit{Test bench - \textbf{d.} Positive-Edge-Triggered T Flip-Flop}

\vspace{1mm}
\hrule

\begin{verbatim}
stim_proc: process
begin		
    Reset_tff <= '1'; T_tff <= '0'; wait for Clk_tff_period*2;
    Reset_tff <= '0'; T_tff <= '0'; wait for Clk_tff_period*2;
    T_tff <= '1'; wait for Clk_tff_period*2;
    T_tff <= '0'; wait for Clk_tff_period*2;
end process;
\end{verbatim}

\vspace{1mm}
\hrule

\begin{figure}[H]
    \centering
    \includegraphics[width=0.75\linewidth]{1_d_graph_i_g.png}
\end{figure}

\textbf{Q2:} \textbf{1. Analytical Solution}

\begin{minipage}{0.57\textwidth}
\centering
\renewcommand{\arraystretch}{1.2}
\begin{tabular}{|c|c|c|c|c|c|}
\hline
\multicolumn{2}{|c|}{Present State} & Input & \multicolumn{2}{c|}{Next State} & Output \\ \hline
D1(t) & D2(t) & x & D1(t+1) & D2(t+1) & y \\\hline
0 & 0 & 0 & 0 & 0 & 0 \\
0 & 0 & 1 & 0 & 1 & 0 \\
0 & 1 & 0 & 1 & 0 & 0 \\
0 & 1 & 1 & 0 & 1 & 0 \\ 
1 & 0 & 0 & 0 & 0 & 0 \\
1 & 0 & 1 & 1 & 1 & 0 \\
1 & 1 & 0 & 1 & 0 & 0 \\
1 & 1 & 1 & 0 & 1 & 1 \\\hline
\end{tabular}

\vspace{0.5em}

State Table for Mealy Sequence Detector

\vspace{0.5em}

$\text{y}=\text{D1}(\text{t})\,\text{D2}(\text{t})\,\text{x}$
\end{minipage}
\begin{minipage}{0.45\textwidth}
\centering
\begin{Karnaughvuit2}
\contingut{0,1,0,1,0,0,1,0}
\implicant{1}{3}{imp1}
\implicant{6}{6}{imp2}
\end{Karnaughvuit2}

\vspace{-1em}

{\small
$\text{D1}(\text{t}+1)=\text{x}'\,\text{D2}(\text{t})+\text{x}\,\text{D1}(\text{t})\,\text{D2}'(\text{t})$}

\begin{Karnaughvuit2}
\contingut{0,0,0,0,1,1,1,1}
\implicant{4}{6}{imp1}
\end{Karnaughvuit2}

\vspace{-1em}

{\small
$\text{D2}(\text{t}+1)=\text{x}$}

\end{minipage}

\vspace{1em}
The characteristic equation of a D flip-flop is $\text{D}=\text{Q}(\text{t}+1)$. Therefore, we may use 2 D flip-flops for $\text{D1}(\text{t}+1)$ and $\text{D2}(\text{t}+1)$ and treat the next states as the inputs of the D flip-flops.

\textit{Hand-drawn circuit and RTL schematic}

\begin{figure}[H]
    \centering
    \includegraphics[width=1\linewidth]{2_hdc_g.jpg}
\end{figure}

\begin{figure}[H]
    \centering
    \includegraphics[width=1\linewidth]{2_rtl_i_g.png}
\end{figure}

\vspace{1em}

\textbf{2. Codes}

\textit{VHDL - D Flip-Flop}

\vspace{1mm}
\hrule

\begin{verbatim}
library IEEE;
use IEEE.STD_LOGIC_1164.ALL;

entity EXP4_PRE_Q2_DFLIPFLOP is
    Port ( D_dff, Reset_dff, Clock_dff : in STD_LOGIC;
           Q_dff, Qn_dff : out STD_LOGIC);
end EXP4_PRE_Q2_DFLIPFLOP;

architecture Behavioral of EXP4_PRE_Q2_DFLIPFLOP is
signal tempQ : STD_LOGIC;
begin
    process (Reset_dff, Clock_dff)
    begin
        if Clock_dff'event and Clock_dff = '1' then
            if Reset_dff = '1' then tempQ <= '0';
            else tempQ <= D_dff;
            end if;
        end if;
    end process;
    Q_dff <= tempQ;
    Qn_dff <= not tempQ;
end Behavioral;
\end{verbatim}

\vspace{1mm}
\hrule

\newpage

\textit{VHDL - Detector}

\vspace{1mm}
\hrule

\begin{verbatim}
library IEEE;
use IEEE.STD_LOGIC_1164.ALL;

entity EXP4_PRE_Q2_DETECTOR is
    Port ( x, res, clk : in STD_LOGIC;
           y : out STD_LOGIC);
end EXP4_PRE_Q2_DETECTOR;

architecture Behavioral of EXP4_PRE_Q2_DETECTOR is
component EXP4_PRE_Q2_DFLIPFLOP
    Port ( D_dff, Reset_dff, Clock_dff : in STD_LOGIC;
           Q_dff, Qn_dff : out STD_LOGIC);
end component;
signal D1, D2, Q1, Q1n, Q2, Q2n : STD_LOGIC;
begin
    D1 <= (not x and Q2) or (x and Q1 and Q2n) ;
    D2 <= x;
    dff1 : EXP4_PRE_Q2_DFLIPFLOP port map(D1, res, clk, Q1, Q1n);
    dff2 : EXP4_PRE_Q2_DFLIPFLOP port map(D2, res, clk, Q2, Q2n);
    y <= Q1 and Q2 and x;
end Behavioral;
\end{verbatim}

\vspace{1mm}
\hrule

\vspace{1em}

\textbf{3. Results}

\textit{Test bench}

\vspace{1mm}
\hrule

\begin{verbatim}
stim_proc: process
begin
    res <= '1'; wait for clk_period;
    x <= '0'; res <= '0'; wait for clk_period;
    x <= '1'; wait for clk_period;
    x <= '1'; wait for clk_period;
    x <= '0'; wait for clk_period;
    x <= '0'; wait for clk_period;
    x <= '1'; wait for clk_period;
    x <= '0'; wait for clk_period;
    x <= '1'; wait for clk_period;
    x <= '1'; wait for clk_period;
    x <= '1'; wait for clk_period;
    x <= '0'; wait for clk_period
end process;
\end{verbatim}

\vspace{1mm}
\hrule

\begin{figure}[H]
    \centering
    \includegraphics[width=1\linewidth]{2_graph_i_g.png}
\end{figure}

\newpage

\textbf{Q3:} \textbf{1. Analytical Solution}

\begin{minipage}{1\textwidth}
\centering
\renewcommand{\arraystretch}{1.2}
\begin{tabular}{|c|c|c|c|c|}
\hline
Present State & \multicolumn{2}{|c|}{Input} & Next State & Output \\ \hline
T(t) & Data & parityin & T$(\text{t}+1)$ & paritychk \\\hline
0 & 0 & 0 & 0 & 1 \\
0 & 0 & 1 & 0 & 0 \\
0 & 1 & 0 & 1 & 1 \\
0 & 1 & 1 & 1 & 0 \\
1 & 0 & 0 & 0 & 0 \\
1 & 0 & 1 & 0 & 1 \\
1 & 1 & 0 & 1 & 0 \\
1 & 1 & 1 & 1 & 1 \\\hline
\end{tabular}

\vspace{0.5em}

State Table for Parity Generator \& Checker
\end{minipage}

\begin{minipage}{0.5\textwidth}
\centering
\begin{Karnaughvuit3}
    \contingut{0,0,1,1,0,0,1,1}
    \implicant{3}{7}{imp1}
\end{Karnaughvuit3}

\vspace{-1em}

{\small
$\text{T}(\text{t}+1)=\text{Data}$}
\end{minipage}\begin{minipage}{0.5\textwidth}
\centering
\begin{Karnaughvuit3}
    \contingut{1,0,1,0,0,1,0,1}
    \implicant{3}{7}{imp1}
\end{Karnaughvuit3}

\vspace{-1em}

{\small
$\text{paritychk}=\text{T}'(\text{t})\,\text{parityin}'+\text{T}(\text{t})\,\text{parityin}$
$\text{paritychk}=(\text{T}(\text{t})\oplus\text{parityin})'$}
\end{minipage}

\vspace{1em}

\textit{Hand-drawn circuit and RTL schematic}

\begin{figure}[H]
    \centering
    \includegraphics[width=1\linewidth]{3_hdc_g.jpg}
\end{figure}

\begin{figure}[H]
    \centering
    \includegraphics[width=1\linewidth]{3_rtl_i_g.png}
\end{figure}

\textbf{2. Codes}

\textit{VHDL - T Flip-Flop}

\vspace{1mm}
\hrule

\begin{verbatim}
library IEEE;
use IEEE.STD_LOGIC_1164.ALL;
entity EXP4_PRE_Q3_TFLIPFLOP is
    port ( T_tff, Reset_tff, Clk_tff : in std_logic;
           Q_tff, Qn_tff : out std_logic);
end EXP4_PRE_Q3_TFLIPFLOP;
architecture Behavioral of EXP4_PRE_Q3_TFLIPFLOP is
signal tempQ : std_logic;
begin
    process (Reset_tff, Clk_tff)
    begin
        if Clk_tff'event and Clk_tff = '1' then
            if Reset_tff = '1' then tempQ <= '0';
            elsif T_tff = '1' then tempQ <= not tempQ;
            end if;
        end if;
    end process;
    Q_tff <= tempQ; Qn_tff <= not tempQ;
end Behavioral;
\end{verbatim}

\vspace{1mm}
\hrule

\newpage

\textit{VHDL - Parity Generator \& Checker}

\vspace{1mm}
\hrule

\begin{verbatim}
library IEEE;
use IEEE.STD_LOGIC_1164.ALL;

entity EXP4_PRE_Q3_PARITY is
    Port ( data, parityin, reset, clk : in STD_LOGIC;
	        parityout, paritychk : out STD_LOGIC);
end EXP4_PRE_Q3_PARITY;

architecture Behavioral of EXP4_PRE_Q3_PARITY is

component EXP4_PRE_Q3_TFLIPFLOP
    port ( T_tff, Reset_tff, Clk_tff : in std_logic;
           Q_tff, Qn_tff : out std_logic);
end component;

signal tempQ, tempQn: STD_LOGIC; 

begin
    tff : EXP4_PRE_Q3_TFLIPFLOP port map (data, reset, clk, tempQ, tempQn);
    parityout <= tempQ;
    paritychk <= tempQ xnor parityin;
end Behavioral;
\end{verbatim}

\vspace{1mm}
\hrule

\vspace{1em}

\textbf{3. Results}

\textit{Test bench }

\vspace{1mm}
\hrule

\begin{verbatim}
stim_proc: process
begin
    -- Data 1
    reset <= '1'; parityin <= '1'; wait for clk_period;
    reset <= '0'; data <= '1'; wait for clk_period;
    data <= '0'; wait for clk_period;
    data <= '1'; wait for clk_period;
    data <= '1'; wait for clk_period;
    data <= '0'; wait for clk_period;
    data <= '1'; wait for clk_period;
    data <= '0'; wait for clk_period;
    data <= '1'; wait for clk_period;
    data <= '0'; wait for clk_period;
    data <= '1'; wait for clk_period;
    data <= '1'; wait for clk_period;
    data <= '0'; wait for clk_period;


    
    -- Data 2
    reset <= '1'; parityin <= '1'; wait for clk_period;
    reset <= '0'; data <= '1'; wait for clk_period;
    data <= '1'; wait for clk_period;
    data <= '1'; wait for clk_period;
    data <= '1'; wait for clk_period;
    data <= '0'; wait for clk_period;
    data <= '0'; wait for clk_period;
    data <= '1'; wait for clk_period;
    data <= '0'; wait for clk_period;
    data <= '1'; wait for clk_period;
    data <= '1'; wait for clk_period;
    data <= '0'; wait for clk_period;
    data <= '1'; wait for clk_period;
    
    -- Data 3
    reset <= '1'; parityin <= '0'; wait for clk_period;
    reset <= '0'; data <= '0'; wait for clk_period;
    data <= '0'; wait for clk_period;
    data <= '0'; wait for clk_period;
    data <= '0'; wait for clk_period;
    data <= '0'; wait for clk_period;
    data <= '0'; wait for clk_period;
    data <= '0'; wait for clk_period;
    data <= '0'; wait for clk_period;
    data <= '1'; wait for clk_period;
    data <= '0'; wait for clk_period;
    data <= '1'; wait for clk_period;
    data <= '0'; wait for clk_period;
    
    -- Data 4
    reset <= '1'; parityin <= '0'; wait for clk_period;
    reset <= '0'; data <= '1'; wait for clk_period;
    data <= '1'; wait for clk_period;
    data <= '1'; wait for clk_period;
    data <= '1'; wait for clk_period;
    data <= '1'; wait for clk_period;
    data <= '1'; wait for clk_period;
    data <= '1'; wait for clk_period;
    data <= '1'; wait for clk_period;
    data <= '1'; wait for clk_period;
    data <= '1'; wait for clk_period;
    data <= '1'; wait for clk_period;
    data <= '1'; wait for clk_period;
end process;
\end{verbatim}

\vspace{1mm}
\hrule

\newpage

\begin{center}\textit{Data 1}\end{center}
\begin{figure}[H]
    \centering
    \includegraphics[width=1\linewidth]{3_graph1_i_g.png}
\end{figure}

\vspace{1em}

\begin{center}\textit{Data 2}\end{center}
\begin{figure}[H]
    \centering
    \includegraphics[width=1\linewidth]{3_graph2_i_g.png}
\end{figure}

\vspace{1em}

\begin{center}\textit{Data 3}\end{center}
\begin{figure}[H]
    \centering
    \includegraphics[width=1\linewidth]{3_graph3_i_g.png}
\end{figure}

\vspace{1em}

\begin{center}\textit{Data 4}\end{center}
\begin{figure}[H]
    \centering
    \includegraphics[width=1\linewidth]{3_graph4_i_g.png}
\end{figure}

\newpage

\textbf{Q4:} \textbf{1. Analytical Solution}

\begin{minipage}{0.5\textwidth}\hspace{0.75em}
\begin{center}
\renewcommand{\arraystretch}{1.2}
\begin{tabular}{|c|c|c|c|c|}
    \hline J & K & Q(t) & Q(t+1) & D \\\hline
    0 & 0 & 0 & 0 & 0 \\
    0 & 0 & 1 & 1 & 1 \\
    0 & 1 & 0 & 0 & 0 \\
    0 & 1 & 1 & 0 & 0 \\
    1 & 0 & 0 & 1 & 1 \\
    1 & 0 & 1 & 1 & 1 \\
    1 & 1 & 0 & 1 & 1 \\
    1 & 1 & 1 & 0 & 0 \\\hline
\end{tabular}

\vspace{1em}

Truth Table for D Flip-Flop

Excitation Table for JK Flip-Flop

\end{center}

\end{minipage}\begin{minipage}{0.5\textwidth}\hspace{0.75em}
\begin{center}
\begin{Karnaughvuit}
\contingut{0,0,1,1,1,0,1,0}
\implicant{3}{2}{imp1}
\implicantcostats{4}{6}{imp2}
\end{Karnaughvuit}

\vspace{-1em}

$\text{D}=\text{QK}'+\text{Q}'\text{J}$

\end{center}
\end{minipage}

\vspace{1em}

\textit{Hand-drawn circuit and RTL schematic}

\begin{figure}[H]
    \centering
    \includegraphics[width=0.9\linewidth]{4_hdc_g.jpg}
\end{figure}

\vspace{-0.5em}

\begin{figure}[H]
    \centering
    \includegraphics[width=0.9\linewidth]{4_rtl_i_g.png}
\end{figure}

\newpage

\textbf{2. Codes}


\textit{VHDL - D Flip-Flop}

\vspace{1mm}
\hrule

\begin{verbatim}
library IEEE;
use IEEE.STD_LOGIC_1164.ALL;
entity EXP4_PRE_Q4_DFLIPFLOP is
    Port ( D_dff, Reset_dff, Clock_dff : in STD_LOGIC;
           Q_dff, Qn_dff : out STD_LOGIC);
end EXP4_PRE_Q4_DFLIPFLOP;

architecture Behavioral of EXP4_PRE_Q4_DFLIPFLOP is
signal tempQ : STD_LOGIC;
begin
    process (Reset_dff, Clock_dff)
    begin
        if Clock_dff'event and Clock_dff = '1' then
            if Reset_dff = '1' then tempQ <= '0';
            else tempQ <= D_dff;
            end if;
        end if;
    end process;
    Q_dff <= tempQ; Qn_dff <= not tempQ;
end Behavioral;
\end{verbatim}

\vspace{1mm}
\hrule

\vspace{1em}

\textit{VHDL - Blackbox}

\vspace{1mm}
\hrule

\begin{verbatim}
library IEEE;
use IEEE.STD_LOGIC_1164.ALL;
entity EXP4_PRE_Q4_BLACKBOX is
    Port ( J_jkff, K_jkff, Reset_jkff, Clock_jkff : in STD_LOGIC;
           Q_jkff, Qn_jkff : out STD_LOGIC);
end EXP4_PRE_Q4_BLACKBOX;

architecture Behavioral of EXP4_PRE_Q4_BLACKBOX is
component EXP4_PRE_Q4_DFLIPFLOP
    Port ( D_dff, Reset_dff, Clock_dff : in STD_LOGIC;
           Q_dff, Qn_dff : out STD_LOGIC);
end component;
signal tempQ, tempQn, D : STD_LOGIC;
begin
    D <= (tempQ and not K_jkff) or (tempQn and J_jkff);
    dff : EXP4_PRE_Q4_DFLIPFLOP port map (D, Reset_jkff, Clock_jkff,
	                                         tempQ, tempQn);
    Q_jkff <= tempQ;
    Qn_jkff <= tempQn;
end Behavioral;
\end{verbatim}

\vspace{1mm}
\hrule

\vspace{1em}

\textbf{3. Results}

\textit{Test bench}

\vspace{1mm}
\hrule

\begin{verbatim}
stim_proc: process
begin
  Reset_jkff <= '1'; wait for Clock_jkff_period;
  J_jkff <= '0'; K_jkff <= '0'; Reset_jkff<='0'; wait for Clock_jkff_period;
  J_jkff <= '0'; K_jkff <= '1'; wait for Clock_jkff_period;
  J_jkff <= '1'; K_jkff <= '0'; wait for Clock_jkff_period;
  J_jkff <= '1'; K_jkff <= '1'; wait for Clock_jkff_period*2;
  J_jkff <= '0'; K_jkff <= '1'; wait for Clock_jkff_period;
  J_jkff <= '0'; K_jkff <= '0'; wait for Clock_jkff_period;
end process;
\end{verbatim}

\vspace{1mm}
\hrule

\begin{figure}[H]
    \centering
    \includegraphics[width=0.85\linewidth]{4_graph_i_g.png}
\end{figure}

\end{document}