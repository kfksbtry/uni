% Student Number: 2240357068
% Student Name: Baturay KAFKAS
% EEE @ Hacettepe University

% My resource collection: https://github.com/kfksbtry/uni
% Circuits were set up in LTspice.

\documentclass{article}

\usepackage{graphicx}
\usepackage[top=25mm, bottom=25mm, left=25mm, right=25mm]{geometry}
\usepackage{amsmath}
\usepackage{moresize}
\usepackage{parskip}
\usepackage{float}
\usepackage{fancyhdr}
\usepackage{booktabs}
\usepackage{pgfplots}
\pgfplotsset{compat=1.18}
\usepackage{tikz}

\pagestyle{fancy}
\fancyhf{}
\fancyhead[R]{Baturay KAFKAS 2240357068 Electrical \& Electronics Engineering}

\rfoot{\thepage}
\renewcommand{\headrulewidth}{0pt} 
\renewcommand{\footrulewidth}{0pt}

\begin{document}

\large
\textit{This part of the experiment is prepared with Online LaTeX Editor Overleaf. Visit the website for the source here:}

\textbf{https://www.overleaf.com/read/xymdvxntjkyk\#386e99}

\hrule

\vspace{1em}

\textbf{1. EXPERIMENT 6 - PRELIMINARY WORK}

\textbf{1.1} Find the roots of the characteristic equation that governs the transient behavior of the voltage shown in \textit{Fig. 1} if $R=300\:\Omega,\:L=50\:\text{mH}$, and $ C = 0.2\:\mu\text{F}$. Will the response be overdamped, underdamped, or critically damped? What value of $R$ causes the response to be critically damped?

\begin{figure}[H]
    \centering
    \includegraphics[width=0.35\linewidth]{chrome_4BzGWESKpi.png}
\end{figure}

\textbf{Answer}: Let $I_R, I_L,$ and $I_C$ be the currents through the resistor, inductor, and capacitor, respectively, directed downward. By KCL,

\[I_R+I_L+I_C=0\implies\frac{V}{R}+\frac{1}{L}\int_0^tV(\tau)\,d\tau+I_L(0)+C\frac{dV}{dt}=0.\]

Differentiate each side with respect to $t$.

\[\frac1R\frac{dV}{dt}+\frac1LV+C\frac{d^2V}{dt^2}=0\implies\frac{d^2V}{dt^2}+\frac1{RL}\frac{dV}{dt}+\frac{1}{LC}V=0\]

This is a second-order homogeneous ordinary differential equation with constant coefficients. Therefore, we can extract the characteristic equation.

\[s^2+\frac{1}{RC}s+\frac1{LC}=0\implies s_{1,2}=\frac1{2RC}\pm\sqrt{\frac1{4R^2C^2}-\frac{1}{LC}}=\alpha\pm\sqrt{\alpha^2-\omega_0^2}\]

$\alpha$ is the neper frequency and $\omega_0$ is the resonant frequency.

\[\alpha=\frac1{2RC}=\frac1{2\cdot300\cdot0.2\cdot10^{-6}}=8333\:\frac{\text{rad}}{\text{s}},\quad\omega_0=\frac1{\sqrt{LC}}=\frac1{\sqrt{50\cdot10^{-3}\cdot0.2\cdot10^{-6}}}=10^4\:\frac{\text{rad}}{\text{s}}\]

Then the roots are

\[\boxed{s_{1,2}=8333\pm\sqrt{8333^2-10000^2}=8333\pm j\,5528}.\]

Since we have two distinct complex roots, the response is $\boxed{\text{underdamped}}$.

For the response to be critically damped, set $\alpha=\omega_0$.

\[\alpha=\omega_0\implies\frac1{2RC}=\frac1{\sqrt{LC}}\implies R=\frac{\sqrt{LC}}{2C}=\frac{\sqrt{50\cdot10^{-3}\cdot0.2\cdot10^{-6}}}{2\cdot0.2\cdot10^{-6}}=250\:\Omega\]

For critically damped response, $\boxed{R_{\text{critical}}=250\:\Omega}$.

\vspace{1em}

\textbf{1.2} Suggest a method to observe and measure the current on the inductor in \textit{Fig. 1} using the oscilloscope.

\textbf{Answer}: We may insert a shunt resistor with low resistance in series with the inductor, and then observe the voltage waveform on the oscilloscope. The ratio of the instantaneous resistor voltage and the resistance gives us the current through the inductor. The circuit is slightly modified, but this yields an approximate inductor current.

\vspace{1em}

\textbf{1.3} No energy is stored in the 100 mH inductor or the $0.4\:\mu$F capacitor when the switch in the circuit shown in \textit{Fig. 2} is closed. Find $V_C(t)$ for $t \geq 0$. Will the response be overdamped, underdamped, or critically damped? What value of $R$ causes the response to be critically damped?

\begin{figure}[H]
    \centering
    \includegraphics[width=0.5\linewidth]{chrome_9pxJ4ySTvj.png}
\end{figure}

\textbf{Answer}: Let $I$ be the current in the circuit. By KVL,

\begin{equation}-48+280\,I+\frac1{0.4\cdot10^{-6}}\int_0^tI(\tau)\,d\tau+V_C(0)+0.1\,\frac{dI}{dt}=0\end{equation}

Differentiate each side with respect to $t$.

\[-280\,\frac{dI}{dt}+25\cdot10^5I+0.1\,\frac{d^2I}{dt^2}=0\implies\frac{d^2I}{dt^2}-2800\,\frac{dI}{dt}+25\cdot10^6\,I=0\]

This is a second-order homogeneous ordinary differential equation with constant coefficients. Therefore, we can extract the characteristic equation.

\[s^2-2800s+25\cdot10^6=0\]

\[s^2+\frac{R}{L}s+\frac1{LC}=0\implies s_{1,2}=\frac R{2L}\pm\sqrt{\frac{R^2}{4L^2}-\frac{1}{LC}}=\alpha\pm\sqrt{\alpha^2-\omega_0^2}\]

\[\alpha=\frac R{2L}=\frac{280}{2\cdot0.1}=1400\:\frac{\text{rad}}{\text{s}},\quad\omega_0=\frac1{\sqrt{LC}}=\frac1{\sqrt{0.1\cdot0.4\cdot10^{-6}}}=500\:\frac{\text{rad}}{\text{s}}\]

Then the roots are

\[s_{1,2}=1400\pm\sqrt{1400^2-5000^2}=1400\pm j\,4800.\]

Since we have two distinct complex roots, the response is $\boxed{\text{underdamped}}$. The current in the circuit can be expressed as follows.

\begin{equation}I(t)=e^{-1400t}[c_1\cos(4800t)+c_2\sin(4800t)]\end{equation}

It is given that no energy is stored in the inductor and capacitor initially. That is,

\[\frac12CV^2_C(0)=0\implies V_C(0)=0,\qquad\frac12LI^2(0^+)=0\implies I(0^+)=0.\]

Therefore,

\[I(0^+)=e^{-1400\cdot0}[c_1\cos(4800\cdot0)+c_2\sin(4800\cdot0)]=c_1=0.\]

We have the relationship between current through and voltage across the inductor

\[V_L=L\frac{dI}{dt}\implies\frac{dI}{dt}=\frac{V_L}{L}=\frac{-V_S-V_C-V_R}{L},\]

where $V_S$ is the voltage supplied. Solve $(1)$ for $\dfrac{dI}{dt}$.

\[\frac{dI}{dt}=\frac1{0.1}\left[-48-\left(\frac1{0.4\cdot10^{-6}}\int_0^tI(\tau)\,d\tau+V_C(0)\right)-280\,I\right]\]

At $t=0$,

\[\left.\frac{dI}{dt}\right|_{t=0}=-480-0-\underbrace{10V_C(0)}_0-\underbrace{2800\,I(0^+)}_0=-480.\]

Differentiate (2) with respect to $t$ and evaluate the resulting expression at $t=0$.

\begin{align*}\dfrac{dI}{dt}&=-1400e^{-1400t}[c_1\cos(4800t)+c_2\sin(4800t)]\\&\quad+e^{-1400t}[-4800c_1\sin(4800t)+4800c_2\cos(4800t)]\\\\&=e^{-1400t}[(4800c_2-1400c_1)\cos(4800t)+(-1400c_2-4800c_1)\sin(4800t)]\end{align*}

\begin{align*}\left.\frac{dI}{dt}\right|_{t=0}&=e^{-1400\cdot0}[(4800c_2-1400\cdot0)\cos(4800\cdot0)+(-1400c_2-4800\cdot0)\sin(4800\cdot0)]\\\\&=-4800c_2=-480\implies c_2=\frac1{10}\end{align*}

The current in the circuit is

\[I(t)=\frac1{10}e^{-1400t}\sin(4800t).\]

The voltage across the capacitor is then

\begin{align*}V_C(t)&=\frac1C\int_0^tI(\tau)\,d\tau+V_C(0)=\frac1{0.4\cdot10^{-6}}\int_0^t\frac1{10}e^{-1400\tau}\sin(4800\tau)\,d\tau+0\\\\&=25\cdot10^4\int_0^te^{-1400\tau}\sin(4800\tau)\,d\tau.\end{align*}

Use integration by parts.

\[\left.\begin{array}{c}
u=e^{-1400\tau}\implies du=-1400e^{-1400\tau}\,d\tau\\[0.5em]
dv=\sin(4800\tau)\,d\tau\implies v=-\dfrac1{4800}\cos(4800\tau)
\end{array}\right\}\implies\int_0^t u\,dv=uv\,\Bigg|_0^t-\int_0^t v\,du\]
\[V_C(t)=25\cdot10^4\left(-\frac1{4800}e^{-1400\tau}\cos(4800\tau)\,\Bigg|_0^t-\frac{1400}{4800}\int_0^t e^{-1400\tau}\cos(4800\tau)\,d\tau\right)\]

Apply integration by parts once again.

\[\left.\begin{array}{c}
w=e^{-1400\tau}\implies dw=-1400e^{-1400\tau}\,d\tau\\[0.5em]
dz=\cos(4800\tau)\,d\tau\implies z=\dfrac1{4800}\sin(4800\tau)
\end{array}\right\}\implies\int_0^t w\,dz=wz\,\Bigg|_0^t-\int_0^t z\,dw\]
\begin{align*}
V_C(t)&=25\cdot10^4\Biggl\{-\frac{e^{-1400\tau}}{4800}\cos(4800\tau)\,\Bigg|_0^t-\frac7{24}\Biggl[\frac{e^{-1400\tau}}{4800}\sin(4800\tau)\,\Bigg|_0^t\\&\qquad\qquad\quad-\Biggl(-\frac{1400}{4800}\underbrace{\int_0^te^{-1400\tau}\sin(4800\tau)\,d\tau}_{V_C(t)/(25\cdot10^4)}\Biggr)\Biggr]\Biggr\}
\end{align*}

\[V_C(t)=-e^{-1400t}\left[\frac{625}{12}\cos(4800t)+\frac{4375}{288}\sin(4800t)\right]-\frac{49}{576}V_C(t)+c_3\]

\[V_C(t)=\frac{576}{625}c_3-e^{-1400t}\left[48\cos(4800t)+14\sin(4800t)\right]\text{ V},\quad t\geq0\]

We have $V_C(0)=0$. Therefore,

\[V_C(0)=\frac{576}{625}c_3-48=0\implies c_3=\frac{625}{12}\]
\[\implies \boxed{V_C(t)=48-e^{-1400t}\left[48\cos(4800t)+14\sin(4800t)\right]\text{ V},\quad t\geq0}.\]

For the response to be critically damped, set $\alpha=\omega_0$.

\[\alpha=\omega_0\implies\frac R{2L}=\frac1{\sqrt{LC}}\implies R=\frac{2L}{\sqrt{LC}}=\frac{2\cdot0.1}{\sqrt{0.1\cdot0.4\cdot10^{-6}}}=1000\:\Omega\]

For critically damped response, $\boxed{R_{\text{critical}}=1000\:\Omega}$.

\end{document}